\documentclass[conference, 10pt, final, letterpaper, twocolumn]{IEEEtran}
\usepackage[utf8]{inputenc}
\usepackage{stmaryrd}
\usepackage{amsfonts}
\usepackage{mathrsfs,amsmath}
\usepackage{amssymb}
\usepackage{makeidx}
\usepackage{mhchem}
\usepackage{multirow}
\usepackage{algorithmic} 
\usepackage{bm}
\usepackage{setspace}
\usepackage{amsthm}
\usepackage{empheq}
\usepackage{booktabs}
\usepackage{authblk}
\usepackage[ruled,linesnumbered,vlined]{algorithm2e}  
\usepackage{color}
\usepackage{enumerate}
\usepackage{makecell}
\usepackage[left=0.65in, right=0.65in, top=0.7in, bottom = 0.98in]{geometry}
\usepackage{tabularx,booktabs}
\usepackage{ifpdf}
 \ifpdf
 \else
 \fi
\usepackage{cite}
\usepackage{threeparttable}
\ifCLASSINFOpdf
   \usepackage[pdftex]{graphicx}
\else
\fi
\usepackage{amsmath}
\usepackage[percent]{overpic}
\renewcommand{\algorithmicrequire}{\textbf{Input:}} 
\renewcommand{\algorithmicensure}{\textbf{Output:}} 
\usepackage{array} 
\usepackage{tikz,pgfplots,filecontents}
\usepackage{rotating}
\usetikzlibrary{spy} %
\pgfplotsset{width=8.6cm, height=6cm, compat=1.9}

\usepackage{physics}
\usepackage{amsmath}
\usepackage{tikz}
\usepackage{mathdots}
\usepackage{yhmath}
\usepackage{cancel}
\usepackage{color}
\usepackage{siunitx}
\usepackage{array}
\usepackage{multirow}
\usepackage{amssymb}
\usepackage{gensymb}
\usepackage{tabularx}
\usepackage{extarrows}
\usepackage{booktabs}
\usetikzlibrary{fadings}
\usetikzlibrary{patterns}
\usetikzlibrary{shadows.blur}
\usetikzlibrary{shapes}

\usetikzlibrary{calc}
\usepackage{pgfplotstable}
 
\ifCLASSOPTIONcompsoc 
  \usepackage[caption=false,font=normalsize,labelfont=sf,textfont=sf]{subfig}
\else
  \usepackage[caption=false,font=footnotesize]{subfig}
\fi

\usepackage{tikz-network}
\usetikzlibrary{shapes}
 
\usepackage{url}
\usepackage{hyperref}


\makeatletter
\let\NAT@parse\undefined
\makeatother

\newtheorem{theorem}{Theorem}
\newtheorem{lemma}[theorem]{Lemma}
\newtheorem{proposition}[theorem]{Proposition}
\newtheorem{definition}{Definition}
\newtheorem{corollary}[theorem]{Corollary}
 \newtheorem{remark}[theorem]{Remark}
\newtheorem*{pf}{Proof}
 
\newcommand{\tabincell}[2]{\begin{tabular}{@{}#1@{}}#2\end{tabular}}
\hyphenation{}
\IEEEoverridecommandlockouts   
\renewcommand\thepage{}
\allowdisplaybreaks

\pagenumbering{arabic}

\pgfplotsset{every axis legend/.style={%
cells={anchor=west},
inner xsep=3pt,inner ysep=2pt,nodes={inner sep=0.8pt,text depth=0.15em},
anchor=north east,%
shape=rectangle,%
fill=white,%
draw=black,
at={(0.98,0.98)},
font=\footnotesize,
%line width=1pt,
%tick style={line width=0.8pt}
}}

\pgfplotsset{every axis/.append style={line width=0.6pt,tick style={line width=0.8pt}}}
%\setlength{\abovecaptionskip}{-0pt} 

%---------------------------------------------------------------------------------------------------


\begin{document}

\title{Less Carbon Footprint in Edge Computing by Joint Task Offloading and Energy Sharing}
% \title{Less Carbon Footprint in Edge Computing: The Case of Joint Task Offloading and Battery Sharing}

 
\author{Zhanwei Yu\textsuperscript{1}, Yi Zhao\textsuperscript{1}, Tao Deng\textsuperscript{2}, Lei You\textsuperscript{3}, and Di Yuan\textsuperscript{1}}
\affil{\textsuperscript{1}Department of Information Technology, Uppsala University, Sweden
\authorcr \textsuperscript{2}School of Computer Science and Technology, Soochow University, China
\authorcr \textsuperscript{3}Department of Engineering Technology, Technical University of Denmark, Denmark
\authorcr {\em \textsuperscript{1}\{zhanwei.yu; yi.zhao; di.yuan\}@it.uu.se, \textsuperscript{2}dengtao@suda.edu.cn, \textsuperscript{3}lei.you@pm.me}}

% \author{Zhanwei Yu and Di Yuan}
% \affil{Department of Information Technology, Uppsala University, Sweden
% \authorcr {\em \{zhanwei.yu; di.yuan\}@it.uu.se}}

\renewcommand*{\Affilfont}{\small}

\maketitle

\begin{abstract}
In sprite the state-of-the-art, significantly reducing carbon footprint (CF) in communications systems remains urgent. We address this challenge in the context of edge computing. The carbon intensity of electricity supply largely varies spatially as well as temporally. This, together with energy sharing via a battery management system (BMS), justifies the potential of CF-oriented task offloading, by redistributing the computational tasks in time and space. In this paper, we consider optimal task scheduling and offloading, as well as battery charging and discharging to minimize the total CF. We formulate this CF minimization problem as an integer linear programming model. However, we demonstrate that, via a graph-based reformulation, the problem can be cast as a minimum-cost flow problem. This finding reveals that global optimum can be admitted in polynomial time. Numerical results using real-world data show that optimization can reduce up to $83.3\%$ of the total CF.
\end{abstract}

\begin{IEEEkeywords}
carbon footprint, scheduling, edge computing.
\end{IEEEkeywords}


\section{Introduction}
The current carbon footprint (CF) in information and communications technology (ICT) sector could be as high as $2.1\%$ to $3.9\%$ of the total figure \cite{https://doi.org/10.48550/arxiv.2102.02622}. This is more than the $2\%$ baseline in Europe. The authors of \cite{perrons2021digital} list four potential measures that can significantly help reduce the CF due to ICT, one of which is edge computing. 

The edge servers consume substantially less energy than conventional cloud data centers, though they need to be densely deployed. For green edge computing, research has been conducted for better design of networks \cite{mao2017survey}, including dynamic right-sizing, geographical load balancing, and the use of renewable energy. However, most works with respect to green edge computing consider energy consumption or efficiency as the objective function, rather than CF explicitly. 

Lower energy consumption contributes to reducing CF, however, they are not equivalent. For example, in an edge computing network consisting of uniform servers and task distribution, considering only energy consumption would render task offloading useless (which itself would cost energy). However, as the CF intensity differs by power source over time and space, offloading does help. We show a US domestic carbon intensity (CI) data set and another data set for three counties in Europe in Table \ref{tab:data} \cite{ElectricityMaps}. As we can see, there are large differences between spaces and time. For example, the CI in Oregon is five times higher than that in Washington. Also, The CI in Oregon at 24:00 is 31.7\% less than that at 08:00. The variation is due to the difference in the availability of various power sources over time and space. This motivates us to use the spatial and temporal information of CI to reduce CF by optimal task offloading and scheduling.  




\begin{table}[h]
    \caption{\label{tab:data}Two real-world data sets of carbon intensity.}
    \begin{center}
        \begin{threeparttable}[b]
            \begin{tabular}{*{5}{llrrr}}
                \toprule
                \midrule
                 \multicolumn{2}{c}{\multirow{2}{*}{\bf Region}} & \multicolumn{3}{c}{\bf Carbon intensity\tnote{1}}\\
                 \cmidrule(lr){3-5}
                \multicolumn{2}{c}{} & 08:00 & 16:00 & 24:00 \\
                \midrule
                 \multirow{3}{*}{\begin{turn}{90} USA \end{turn}} & Washington & 110 & 95  & 95\\
                 & Oregon & 605 & 579 & 413\\
                 & California & 325 & 292  & 238\\
                \midrule
                 \multirow{3}{*}{\begin{turn}{90} Europe \end{turn}} & Sweden & 24 & 26 & 25\\
                 & Germany & 375 & 285 & 381\\
                 & Poland & 593 & 573 & 547\\
                \midrule
                \bottomrule
            \end{tabular}
            
            \begin{tablenotes}
            	\footnotesize
            	\item $^1$Unit: g\ce{CO2}eq/kWh.
            \end{tablenotes}
        \end{threeparttable}
    \end{center}
\end{table}



Of the current literature, we highlight the following contributions. The studies in \cite{van2012distributed, Rad2022Carbon, do2015proximal, aldossary2021towards, ahvar2021deca, yang2022carbon} have addressed CF in edge computing or fog computing. The authors of \cite{van2012distributed} provide a model that can estimate the CF in distributed data centers. Their results show that the total CF of a set of distributed small data centers is less than that from a big data center with equivalent compute capability. The authors of \cite{Rad2022Carbon} provide a Lyapunov-based algorithm for a distributed data center to minimize electricity cost subject to CF limit. In \cite{do2015proximal}, the authors consider minimizing the CF for video streaming in fog computing networks. In addition, the studies in \cite{aldossary2021towards} and \cite{ahvar2021deca} provide application placement methods to minimize the CF in fog computing networks. The authors of \cite{yang2022carbon} examine task scheduling policies to minimize the CF of edge computing networks via the drift-plus-penalty methodology in Lyapunov optimization. Table \ref{tab:difference} lists the features of our paper and the related works \cite{van2012distributed, Rad2022Carbon, do2015proximal, aldossary2021towards, ahvar2021deca, yang2022carbon}. Note that the works in \cite{van2012distributed, Rad2022Carbon, do2015proximal, aldossary2021towards, ahvar2021deca} only consider optimizing spatial or temporal dimension instead of joint spatio-temporal optimization. 


\begin{table*}[h] % add energy sharing
    \caption{\label{tab:difference}The difference among our paper and the related works.}
    \begin{center}
        \begin{threeparttable}[b]
            \begin{tabular}{*{7}{lccccc}}
                \toprule
                \midrule
                {\bf Work(s)} & \makecell[c]{\bf Renewable\\ \bf energy} & \makecell[c]{\bf Energy\\ \bf storage} &\makecell[c]{\bf Energy\\ \bf sharing} & \makecell[c]{\bf Spatio-temporal\\ \bf optimization} & \makecell[c]{\bf Task\\ \bf offloading}\\
                \midrule
                 \cite{van2012distributed} & \checkmark & & & &\\
                 \cite{Rad2022Carbon, do2015proximal, aldossary2021towards, ahvar2021deca} & & & & & \checkmark \\ 
                 \cite{yang2022carbon} & & & & \checkmark & \checkmark \\
                 Our work & \checkmark & \checkmark & \checkmark & \checkmark& \checkmark\\
                \bottomrule
            \end{tabular}
        \end{threeparttable}
    \end{center}
\end{table*}

To the best of our knowledge, the work in \cite{yang2022carbon} is the closest to ours, yet the differences are significant. The system model in our paper considers battery charging on top of task offloading and scheduling, making our scenario more comprehensive. To be more exact, our paper considers task scheduling and offloading, as well as energy sharing with battery charging and discharging to minimize the CF, utilizing temporal and spatial information of CI. The main contributions of this paper are summarized below.
\begin{itemize}
    \item We consider an edge computing network with renewable energy sources and batteries, and model the CF for the resulting scenario. We consider task offloading and scheduling, as well as battery charging and discharging with energy sharing, with the objective of CF minimization. 
    \item Due to the discrete nature of task offloadling, this CF minimization problem leads to an integer linear programming (ILP) model. However, we reveal that the structure of the problem admits a reformulation using minimum-cost flow problem, implying that its global optimum can be computed in polynomial time. Thus our optimization approach is highly scalable. 
    \item We use real-world carbon intensity data to evaluate the performance. The numerical results show that, by optimal CF-aware task offloading and scheduling along with energy sharing, our optimization approach can reduce up to $83.3\%$ of the total CF.
\end{itemize}

\section{System Model and Problem Formulation}

\begin{figure}[t]
	\begin{center}



    \tikzset{every picture/.style={line width=0.75pt}} %set default line width to 0.75pt        
    
    \begin{tikzpicture}[x=0.75pt,y=0.75pt,yscale=-1,xscale=1]
    %uncomment if require: \path (0,446); %set diagram left start at 0, and has height of 446
    
        %Shape: Ellipse [id:dp23689960204771032] 
        \draw  [color={rgb, 255:red, 255; green, 255; blue, 255 }  ,draw opacity=1 ][fill={rgb, 255:red, 0; green, 0; blue, 0 }  ,fill opacity=1 ] (155.38,17.72) .. controls (155.38,14.63) and (157.89,12.13) .. (160.97,12.13) .. controls (164.06,12.13) and (166.56,14.63) .. (166.56,17.72) .. controls (166.56,20.81) and (164.06,23.31) .. (160.97,23.31) .. controls (157.89,23.31) and (155.38,20.81) .. (155.38,17.72) -- cycle ;
        %Rounded Rect [id:dp7350954357882564] 
        \draw   (167.98,17.67) .. controls (167.98,17.66) and (167.99,17.65) .. (168,17.65) -- (170.69,17.65) .. controls (170.7,17.65) and (170.71,17.66) .. (170.71,17.67) -- (170.71,17.72) .. controls (170.71,17.73) and (170.7,17.74) .. (170.69,17.74) -- (168,17.74) .. controls (167.99,17.74) and (167.98,17.73) .. (167.98,17.72) -- cycle ;
        %Rounded Rect [id:dp5817858075714595] 
        \draw   (152.56,12.82) .. controls (152.57,12.81) and (152.58,12.8) .. (152.59,12.81) -- (154.92,14.16) .. controls (154.93,14.16) and (154.93,14.17) .. (154.93,14.18) -- (154.9,14.23) .. controls (154.89,14.24) and (154.88,14.24) .. (154.87,14.24) -- (152.54,12.89) .. controls (152.53,12.88) and (152.53,12.87) .. (152.53,12.86) -- cycle ;
        %Rounded Rect [id:dp562001902201841] 
        \draw   (155.9,9.37) .. controls (155.91,9.36) and (155.92,9.36) .. (155.92,9.37) -- (157.27,11.71) .. controls (157.27,11.71) and (157.27,11.72) .. (157.26,11.73) -- (157.22,11.76) .. controls (157.21,11.76) and (157.2,11.76) .. (157.19,11.75) -- (155.84,9.42) .. controls (155.84,9.41) and (155.84,9.4) .. (155.85,9.39) -- cycle ;
        %Rounded Rect [id:dp24994446560754513] 
        \draw   (160.96,8.01) .. controls (160.97,8.01) and (160.97,8.01) .. (160.97,8.02) -- (160.97,10.72) .. controls (160.97,10.73) and (160.97,10.74) .. (160.96,10.74) -- (160.9,10.74) .. controls (160.89,10.74) and (160.88,10.73) .. (160.88,10.72) -- (160.88,8.02) .. controls (160.88,8.01) and (160.89,8.01) .. (160.9,8.01) -- cycle ;
        %Rounded Rect [id:dp41080675695046875] 
        \draw   (151.26,17.65) .. controls (151.26,17.64) and (151.27,17.63) .. (151.28,17.63) -- (153.97,17.63) .. controls (153.98,17.63) and (153.99,17.64) .. (153.99,17.65) -- (153.99,17.7) .. controls (153.99,17.71) and (153.98,17.72) .. (153.97,17.72) -- (151.28,17.72) .. controls (151.27,17.72) and (151.26,17.71) .. (151.26,17.7) -- cycle ;
        %Rounded Rect [id:dp2589224672963768] 
        \draw   (164.51,23.63) .. controls (164.52,23.63) and (164.53,23.63) .. (164.54,23.64) -- (165.88,25.97) .. controls (165.89,25.98) and (165.89,25.99) .. (165.88,26) -- (165.83,26.02) .. controls (165.82,26.03) and (165.81,26.03) .. (165.8,26.02) -- (164.46,23.68) .. controls (164.45,23.68) and (164.46,23.66) .. (164.46,23.66) -- cycle ;
        %Rounded Rect [id:dp52965196080424] 
        \draw   (161.05,24.71) .. controls (161.06,24.71) and (161.06,24.72) .. (161.06,24.73) -- (161.06,27.42) .. controls (161.06,27.43) and (161.06,27.44) .. (161.05,27.44) -- (160.99,27.44) .. controls (160.98,27.44) and (160.97,27.43) .. (160.97,27.42) -- (160.97,24.73) .. controls (160.97,24.72) and (160.98,24.71) .. (160.99,24.71) -- cycle ;
        %Rounded Rect [id:dp4402538190695453] 
        \draw   (167.1,21.03) .. controls (167.1,21.02) and (167.11,21.01) .. (167.12,21.02) -- (169.46,22.37) .. controls (169.46,22.37) and (169.47,22.38) .. (169.46,22.39) -- (169.43,22.44) .. controls (169.43,22.45) and (169.42,22.45) .. (169.41,22.45) -- (167.08,21.1) .. controls (167.07,21.09) and (167.07,21.08) .. (167.07,21.07) -- cycle ;
        %Rounded Rect [id:dp6996007283347523] 
        \draw   (165.88,9.36) .. controls (165.89,9.36) and (165.89,9.37) .. (165.88,9.38) -- (164.54,11.71) .. controls (164.53,11.72) and (164.52,11.73) .. (164.51,11.72) -- (164.46,11.69) .. controls (164.46,11.69) and (164.45,11.68) .. (164.46,11.67) -- (165.8,9.34) .. controls (165.81,9.33) and (165.82,9.32) .. (165.83,9.33) -- cycle ;
        %Rounded Rect [id:dp670060683107357] 
        \draw   (169.46,12.74) .. controls (169.47,12.75) and (169.46,12.76) .. (169.46,12.77) -- (167.12,14.11) .. controls (167.11,14.12) and (167.1,14.12) .. (167.1,14.11) -- (167.07,14.06) .. controls (167.07,14.05) and (167.07,14.04) .. (167.08,14.03) -- (169.41,12.69) .. controls (169.42,12.68) and (169.43,12.69) .. (169.43,12.69) -- cycle ;
        %Rounded Rect [id:dp17906894780647287] 
        \draw   (154.93,20.95) .. controls (154.93,20.96) and (154.93,20.97) .. (154.92,20.98) -- (152.59,22.32) .. controls (152.58,22.33) and (152.57,22.33) .. (152.56,22.32) -- (152.53,22.27) .. controls (152.53,22.26) and (152.53,22.25) .. (152.54,22.24) -- (154.87,20.9) .. controls (154.88,20.89) and (154.89,20.9) .. (154.9,20.9) -- cycle ;
        %Rounded Rect [id:dp7163585653714504] 
        \draw   (157.13,23.62) .. controls (157.14,23.63) and (157.14,23.64) .. (157.13,23.65) -- (155.79,25.98) .. controls (155.78,25.99) and (155.77,25.99) .. (155.76,25.99) -- (155.72,25.96) .. controls (155.71,25.95) and (155.7,25.94) .. (155.71,25.94) -- (157.06,23.6) .. controls (157.06,23.59) and (157.07,23.59) .. (157.08,23.6) -- cycle ;
        
        %Shape: Trapezoid [id:dp39839281755362443] 
        \draw  [color={rgb, 255:red, 255; green, 255; blue, 255 }  ,draw opacity=1 ][fill={rgb, 255:red, 0; green, 0; blue, 0 }  ,fill opacity=1 ] (133.8,34.68) -- (137.2,17.61) -- (160.28,17.61) -- (163.68,34.68) -- cycle ;
        %Straight Lines [id:da8047810174415737] 
        \draw [color={rgb, 255:red, 255; green, 255; blue, 255 }  ,draw opacity=1 ][fill={rgb, 255:red, 0; green, 0; blue, 0 }  ,fill opacity=1 ]   (135.05,29.17) -- (162.64,29.12) ;
        %Straight Lines [id:da8343675725090776] 
        \draw [color={rgb, 255:red, 255; green, 255; blue, 255 }  ,draw opacity=1 ][fill={rgb, 255:red, 0; green, 0; blue, 0 }  ,fill opacity=1 ]   (136,23.22) -- (161.09,23.22) ;
        %Straight Lines [id:da6746426481591967] 
        \draw [color={rgb, 255:red, 255; green, 255; blue, 255 }  ,draw opacity=1 ][fill={rgb, 255:red, 0; green, 0; blue, 0 }  ,fill opacity=1 ]   (152.28,18.1) -- (154.51,34.94) ;
        %Straight Lines [id:da546227329864996] 
        \draw [color={rgb, 255:red, 255; green, 255; blue, 255 }  ,draw opacity=1 ][fill={rgb, 255:red, 0; green, 0; blue, 0 }  ,fill opacity=1 ]   (145.19,17.94) -- (143.19,35.06) ;
        
        
        %Shape: Can [id:dp9005011599408153] 
        \draw  [color={rgb, 255:red, 255; green, 255; blue, 255 }  ,draw opacity=1 ][fill={rgb, 255:red, 0; green, 0; blue, 0 }  ,fill opacity=1 ][line width=1.5]  (170.53,55.02) -- (170.53,62.12) .. controls (170.53,64.08) and (160.74,65.67) .. (148.67,65.67) .. controls (136.6,65.67) and (126.82,64.08) .. (126.82,62.12) -- (126.82,55.02) .. controls (126.82,53.06) and (136.6,51.47) .. (148.67,51.47) .. controls (160.74,51.47) and (170.53,53.06) .. (170.53,55.02) .. controls (170.53,56.98) and (160.74,58.57) .. (148.67,58.57) .. controls (136.6,58.57) and (126.82,56.98) .. (126.82,55.02) ;
        %Shape: Ellipse [id:dp627980002725933] 
        \draw  [fill={rgb, 255:red, 255; green, 255; blue, 255 }  ,fill opacity=1 ] (147.75,61.72) .. controls (147.75,61.13) and (148.23,60.65) .. (148.82,60.65) .. controls (149.41,60.65) and (149.89,61.13) .. (149.89,61.72) .. controls (149.89,62.31) and (149.41,62.79) .. (148.82,62.79) .. controls (148.23,62.79) and (147.75,62.31) .. (147.75,61.72) -- cycle ;
        %Shape: Ellipse [id:dp32022930297328456] 
        \draw  [fill={rgb, 255:red, 255; green, 255; blue, 255 }  ,fill opacity=1 ] (151.55,61.72) .. controls (151.55,61.13) and (152.02,60.65) .. (152.62,60.65) .. controls (153.21,60.65) and (153.68,61.13) .. (153.68,61.72) .. controls (153.68,62.31) and (153.21,62.79) .. (152.62,62.79) .. controls (152.02,62.79) and (151.55,62.31) .. (151.55,61.72) -- cycle ;
        %Shape: Ellipse [id:dp09464548947851847] 
        \draw  [fill={rgb, 255:red, 255; green, 255; blue, 255 }  ,fill opacity=1 ] (143.96,61.72) .. controls (143.96,61.13) and (144.44,60.65) .. (145.03,60.65) .. controls (145.62,60.65) and (146.1,61.13) .. (146.1,61.72) .. controls (146.1,62.31) and (145.62,62.79) .. (145.03,62.79) .. controls (144.44,62.79) and (143.96,62.31) .. (143.96,61.72) -- cycle ;
        
        %Shape: Rectangle [id:dp35865913094459034] 
        \draw  [fill={rgb, 255:red, 0; green, 0; blue, 0 }  ,fill opacity=1 ] (136.18,83.97) -- (162.32,83.97) -- (162.32,96.73) -- (136.18,96.73) -- cycle ;
        %Shape: Cross [id:dp9459558747780203] 
        \draw  [fill={rgb, 255:red, 255; green, 255; blue, 255 }  ,fill opacity=1 ] (143.64,85.38) -- (140.99,85.38) -- (140.99,87.25) -- (139.12,87.25) -- (139.12,89.9) -- (140.99,89.9) -- (140.99,91.76) -- (143.64,91.76) -- (143.64,89.9) -- (145.51,89.9) -- (145.51,87.25) -- (143.64,87.25) -- cycle ;
        %Shape: Rectangle [id:dp8159450856522705] 
        \draw  [fill={rgb, 255:red, 255; green, 255; blue, 255 }  ,fill opacity=1 ] (152.74,87.11) -- (159.17,87.11) -- (159.17,89.98) -- (152.74,89.98) -- cycle ;
        %Shape: Rectangle [id:dp9559476182094357] 
        \draw  [fill={rgb, 255:red, 0; green, 0; blue, 0 }  ,fill opacity=1 ] (140.83,80.67) -- (144.1,80.67) -- (144.1,81.66) -- (140.83,81.66) -- cycle ;
        %Shape: Rectangle [id:dp06489231659707517] 
        \draw  [fill={rgb, 255:red, 0; green, 0; blue, 0 }  ,fill opacity=1 ] (154.7,80.67) -- (157.96,80.67) -- (157.96,81.66) -- (154.7,81.66) -- cycle ;
        
        %Shape: Trapezoid [id:dp5856244719833053] 
        \draw   (215.52,64.5) -- (217.25,53.93) -- (229.25,53.93) -- (230.98,64.5) -- cycle ;
        %Shape: Trapezoid [id:dp44335621565613703] 
        \draw   (217.25,53.93) -- (218.6,45.09) -- (227.9,45.09) -- (229.25,53.93) -- cycle ;
        %Shape: Trapezoid [id:dp2690273077646286] 
        \draw   (218.6,45.09) -- (218.6,37.3) -- (227.9,37.3) -- (227.9,45.09) -- cycle ;
        %Shape: Trapezoid [id:dp789123773409603] 
        \draw   (218.6,37.3) -- (218.6,27.17) -- (227.9,27.17) -- (227.9,37.3) -- cycle ;
        %Straight Lines [id:da828316850046003] 
        \draw    (217.25,53.93) -- (230.98,64.5) ;
        %Straight Lines [id:da591569368482918] 
        \draw    (215.52,64.5) -- (229.25,53.93) ;
        %Straight Lines [id:da7192792870901867] 
        \draw    (217.25,53.93) -- (227.9,45.09) ;
        %Straight Lines [id:da25942426045741995] 
        \draw    (229.25,53.93) -- (218.6,45.09) ;
        %Straight Lines [id:da998802426261975] 
        \draw    (227.9,45.09) -- (218.6,37.3) ;
        %Straight Lines [id:da38916705477465796] 
        \draw    (218.6,45.09) -- (227.9,37.3) ;
        %Straight Lines [id:da26657305293991884] 
        \draw    (218.6,37.3) -- (227.9,27.17) ;
        %Straight Lines [id:da31720929903327444] 
        \draw    (227.9,37.3) -- (218.6,27.17) ;
        %Shape: Right Triangle [id:dp46306371951470315] 
        \draw   (227.9,30.97) -- (240.97,37.3) -- (227.9,37.3) -- cycle ;
        %Straight Lines [id:da5846990799110421] 
        \draw    (227.9,37.3) -- (232.18,33.03) ;
        %Straight Lines [id:da9958530803374206] 
        \draw    (237.52,37.43) -- (237.52,39.57) ;
        
        %Shape: Right Triangle [id:dp09576900324503912] 
        \draw   (218.6,30.97) -- (205.53,37.3) -- (218.6,37.3) -- cycle ;
        %Straight Lines [id:da10297154407621023] 
        \draw    (214.45,33.3) -- (218.6,37.3) ;
        %Straight Lines [id:da9008160700393986] 
        \draw    (209.12,37.43) -- (209.12,39.57) ;
        
        %Straight Lines [id:da3670426702577094] 
        \draw [color={rgb, 255:red, 155; green, 155; blue, 155 }  ,draw opacity=1 ]   (224,89.97) -- (224,69.11) ;
        %Straight Lines [id:da3081807630858884] 
        \draw [color={rgb, 255:red, 155; green, 155; blue, 155 }  ,draw opacity=1 ]   (98.67,24.9) -- (132.5,24.9) ;
        %Straight Lines [id:da9003194521232087] 
        \draw [color={rgb, 255:red, 155; green, 155; blue, 155 }  ,draw opacity=1 ]   (98.67,24.9) -- (98.67,90.4) ;
        %Straight Lines [id:da9853775896451256] 
        \draw [color={rgb, 255:red, 155; green, 155; blue, 155 }  ,draw opacity=1 ]   (98.67,90.4) -- (130.5,90.4) ;
        \draw [shift={(132.5,90.4)}, rotate = 180] [fill={rgb, 255:red, 155; green, 155; blue, 155 }  ,fill opacity=1 ][line width=0.08]  [draw opacity=0] (8.4,-2.1) -- (0,0) -- (8.4,2.1) -- cycle    ;
        %Straight Lines [id:da9683252488589866] 
        \draw [color={rgb, 255:red, 155; green, 155; blue, 155 }  ,draw opacity=1 ]   (148.86,79.97) -- (148.86,68.26) ;
        \draw [shift={(148.86,66.26)}, rotate = 90] [fill={rgb, 255:red, 155; green, 155; blue, 155 }  ,fill opacity=1 ][line width=0.08]  [draw opacity=0] (8.4,-2.1) -- (0,0) -- (8.4,2.1) -- cycle    ;
        %Straight Lines [id:da6427138667449821] 
        \draw [color={rgb, 255:red, 155; green, 155; blue, 155 }  ,draw opacity=1 ]   (148.67,36.26) -- (148.67,47.97) ;
        \draw [shift={(148.67,49.97)}, rotate = 270] [fill={rgb, 255:red, 155; green, 155; blue, 155 }  ,fill opacity=1 ][line width=0.08]  [draw opacity=0] (8.4,-2.1) -- (0,0) -- (8.4,2.1) -- cycle    ;
        %Straight Lines [id:da8819546339421043] 
        \draw [color={rgb, 255:red, 155; green, 155; blue, 155 }  ,draw opacity=1 ]   (210,58.26) -- (174.57,58.26) ;
        \draw [shift={(172.57,58.26)}, rotate = 360] [fill={rgb, 255:red, 155; green, 155; blue, 155 }  ,fill opacity=1 ][line width=0.08]  [draw opacity=0] (8.4,-2.1) -- (0,0) -- (8.4,2.1) -- cycle    ;
        %Straight Lines [id:da12696545096661116] 
        \draw [color={rgb, 255:red, 155; green, 155; blue, 155 }  ,draw opacity=1 ]   (224,89.97) -- (167.43,89.97) ;
        \draw [shift={(165.43,89.97)}, rotate = 360] [fill={rgb, 255:red, 155; green, 155; blue, 155 }  ,fill opacity=1 ][line width=0.08]  [draw opacity=0] (8.4,-2.1) -- (0,0) -- (8.4,2.1) -- cycle    ;
        
        %Shape: Ellipse [id:dp046440233113860696] 
        \draw  [color={rgb, 255:red, 255; green, 255; blue, 255 }  ,draw opacity=1 ][fill={rgb, 255:red, 0; green, 0; blue, 0 }  ,fill opacity=1 ] (92.32,225.59) .. controls (92.32,222.5) and (94.82,220) .. (97.91,220) .. controls (100.99,220) and (103.5,222.5) .. (103.5,225.59) .. controls (103.5,228.68) and (100.99,231.18) .. (97.91,231.18) .. controls (94.82,231.18) and (92.32,228.68) .. (92.32,225.59) -- cycle ;
        %Rounded Rect [id:dp21673028923709814] 
        \draw   (104.91,225.53) .. controls (104.91,225.52) and (104.92,225.52) .. (104.93,225.52) -- (107.62,225.52) .. controls (107.63,225.52) and (107.64,225.52) .. (107.64,225.53) -- (107.64,225.59) .. controls (107.64,225.6) and (107.63,225.61) .. (107.62,225.61) -- (104.93,225.61) .. controls (104.92,225.61) and (104.91,225.6) .. (104.91,225.59) -- cycle ;
        %Rounded Rect [id:dp9417842391430391] 
        \draw   (89.5,220.68) .. controls (89.5,220.67) and (89.51,220.67) .. (89.52,220.68) -- (91.85,222.02) .. controls (91.86,222.03) and (91.86,222.04) .. (91.86,222.05) -- (91.83,222.1) .. controls (91.83,222.1) and (91.82,222.11) .. (91.81,222.1) -- (89.47,220.75) .. controls (89.47,220.75) and (89.46,220.74) .. (89.47,220.73) -- cycle ;
        %Rounded Rect [id:dp4967499642080284] 
        \draw   (92.83,217.23) .. controls (92.84,217.23) and (92.85,217.23) .. (92.86,217.24) -- (94.2,219.57) .. controls (94.21,219.58) and (94.2,219.59) .. (94.2,219.6) -- (94.15,219.62) .. controls (94.14,219.63) and (94.13,219.63) .. (94.12,219.62) -- (92.78,217.28) .. controls (92.77,217.28) and (92.78,217.26) .. (92.78,217.26) -- cycle ;
        %Rounded Rect [id:dp42182766171645003] 
        \draw   (97.89,215.87) .. controls (97.9,215.87) and (97.91,215.88) .. (97.91,215.89) -- (97.91,218.58) .. controls (97.91,218.59) and (97.9,218.6) .. (97.89,218.6) -- (97.83,218.6) .. controls (97.82,218.6) and (97.82,218.59) .. (97.82,218.58) -- (97.82,215.89) .. controls (97.82,215.88) and (97.82,215.87) .. (97.83,215.87) -- cycle ;
        %Rounded Rect [id:dp7372953483874027] 
        \draw   (88.19,225.52) .. controls (88.19,225.51) and (88.2,225.5) .. (88.21,225.5) -- (90.9,225.5) .. controls (90.91,225.5) and (90.92,225.51) .. (90.92,225.52) -- (90.92,225.57) .. controls (90.92,225.58) and (90.91,225.59) .. (90.9,225.59) -- (88.21,225.59) .. controls (88.2,225.59) and (88.19,225.58) .. (88.19,225.57) -- cycle ;
        %Rounded Rect [id:dp016554811734768915] 
        \draw   (101.44,231.5) .. controls (101.45,231.49) and (101.46,231.5) .. (101.47,231.51) -- (102.82,233.84) .. controls (102.82,233.85) and (102.82,233.86) .. (102.81,233.86) -- (102.76,233.89) .. controls (102.75,233.9) and (102.74,233.89) .. (102.74,233.88) -- (101.39,231.55) .. controls (101.39,231.54) and (101.39,231.53) .. (101.4,231.53) -- cycle ;
        %Rounded Rect [id:dp3333806907098691] 
        \draw   (97.98,232.57) .. controls (97.99,232.57) and (98,232.58) .. (98,232.59) -- (98,235.29) .. controls (98,235.3) and (97.99,235.3) .. (97.98,235.3) -- (97.92,235.3) .. controls (97.91,235.3) and (97.91,235.3) .. (97.91,235.29) -- (97.91,232.59) .. controls (97.91,232.58) and (97.91,232.57) .. (97.92,232.57) -- cycle ;
        %Rounded Rect [id:dp4045437830291596] 
        \draw   (104.03,228.89) .. controls (104.04,228.88) and (104.05,228.88) .. (104.06,228.89) -- (106.39,230.23) .. controls (106.4,230.24) and (106.4,230.25) .. (106.4,230.26) -- (106.37,230.31) .. controls (106.36,230.31) and (106.35,230.32) .. (106.34,230.31) -- (104.01,228.96) .. controls (104,228.96) and (104,228.95) .. (104,228.94) -- cycle ;
        %Rounded Rect [id:dp18140925314320766] 
        \draw   (102.81,217.22) .. controls (102.82,217.23) and (102.82,217.24) .. (102.82,217.25) -- (101.47,219.58) .. controls (101.46,219.59) and (101.45,219.59) .. (101.44,219.59) -- (101.4,219.56) .. controls (101.39,219.56) and (101.39,219.54) .. (101.39,219.54) -- (102.74,217.2) .. controls (102.74,217.19) and (102.75,217.19) .. (102.76,217.2) -- cycle ;
        %Rounded Rect [id:dp6960166673215535] 
        \draw   (106.4,220.61) .. controls (106.4,220.62) and (106.4,220.63) .. (106.39,220.63) -- (104.06,221.98) .. controls (104.05,221.98) and (104.04,221.98) .. (104.03,221.97) -- (104,221.93) .. controls (104,221.92) and (104,221.91) .. (104.01,221.9) -- (106.34,220.55) .. controls (106.35,220.55) and (106.36,220.55) .. (106.37,220.56) -- cycle ;
        %Rounded Rect [id:dp34185776666936607] 
        \draw   (91.86,228.82) .. controls (91.86,228.83) and (91.86,228.84) .. (91.85,228.84) -- (89.52,230.19) .. controls (89.51,230.19) and (89.5,230.19) .. (89.5,230.18) -- (89.47,230.14) .. controls (89.46,230.13) and (89.47,230.12) .. (89.47,230.11) -- (91.81,228.76) .. controls (91.82,228.76) and (91.83,228.76) .. (91.83,228.77) -- cycle ;
        %Rounded Rect [id:dp5556414628284645] 
        \draw   (94.06,231.49) .. controls (94.07,231.49) and (94.07,231.51) .. (94.07,231.51) -- (92.72,233.85) .. controls (92.72,233.86) and (92.71,233.86) .. (92.7,233.85) -- (92.65,233.83) .. controls (92.64,233.82) and (92.64,233.81) .. (92.64,233.8) -- (93.99,231.47) .. controls (93.99,231.46) and (94.01,231.46) .. (94.01,231.46) -- cycle ;
        
        %Shape: Trapezoid [id:dp9078658261484358] 
        \draw  [color={rgb, 255:red, 255; green, 255; blue, 255 }  ,draw opacity=1 ][fill={rgb, 255:red, 0; green, 0; blue, 0 }  ,fill opacity=1 ] (70.73,242.55) -- (74.13,225.48) -- (97.21,225.48) -- (100.61,242.55) -- cycle ;
        %Straight Lines [id:da6538397825350515] 
        \draw [color={rgb, 255:red, 255; green, 255; blue, 255 }  ,draw opacity=1 ][fill={rgb, 255:red, 0; green, 0; blue, 0 }  ,fill opacity=1 ]   (71.98,237.04) -- (99.58,236.99) ;
        %Straight Lines [id:da5435454566868603] 
        \draw [color={rgb, 255:red, 255; green, 255; blue, 255 }  ,draw opacity=1 ][fill={rgb, 255:red, 0; green, 0; blue, 0 }  ,fill opacity=1 ]   (72.94,231.09) -- (98.02,231.09) ;
        %Straight Lines [id:da24497164177888142] 
        \draw [color={rgb, 255:red, 255; green, 255; blue, 255 }  ,draw opacity=1 ][fill={rgb, 255:red, 0; green, 0; blue, 0 }  ,fill opacity=1 ]   (89.21,225.97) -- (91.45,242.81) ;
        %Straight Lines [id:da32161055867429855] 
        \draw [color={rgb, 255:red, 255; green, 255; blue, 255 }  ,draw opacity=1 ][fill={rgb, 255:red, 0; green, 0; blue, 0 }  ,fill opacity=1 ]   (82.13,225.81) -- (80.13,242.93) ;
        
        
        %Shape: Can [id:dp4496626078631891] 
        \draw  [color={rgb, 255:red, 255; green, 255; blue, 255 }  ,draw opacity=1 ][fill={rgb, 255:red, 0; green, 0; blue, 0 }  ,fill opacity=1 ][line width=1.5]  (105.06,196.49) -- (105.06,203.59) .. controls (105.06,205.55) and (95.28,207.14) .. (83.21,207.14) .. controls (71.14,207.14) and (61.35,205.55) .. (61.35,203.59) -- (61.35,196.49) .. controls (61.35,194.53) and (71.14,192.94) .. (83.21,192.94) .. controls (95.28,192.94) and (105.06,194.53) .. (105.06,196.49) .. controls (105.06,198.45) and (95.28,200.04) .. (83.21,200.04) .. controls (71.14,200.04) and (61.35,198.45) .. (61.35,196.49) ;
        %Shape: Ellipse [id:dp6589498088673775] 
        \draw  [fill={rgb, 255:red, 255; green, 255; blue, 255 }  ,fill opacity=1 ] (82.29,203.19) .. controls (82.29,202.6) and (82.76,202.12) .. (83.36,202.12) .. controls (83.95,202.12) and (84.43,202.6) .. (84.43,203.19) .. controls (84.43,203.78) and (83.95,204.26) .. (83.36,204.26) .. controls (82.76,204.26) and (82.29,203.78) .. (82.29,203.19) -- cycle ;
        %Shape: Ellipse [id:dp09849771978532496] 
        \draw  [fill={rgb, 255:red, 255; green, 255; blue, 255 }  ,fill opacity=1 ] (86.08,203.19) .. controls (86.08,202.6) and (86.56,202.12) .. (87.15,202.12) .. controls (87.74,202.12) and (88.22,202.6) .. (88.22,203.19) .. controls (88.22,203.78) and (87.74,204.26) .. (87.15,204.26) .. controls (86.56,204.26) and (86.08,203.78) .. (86.08,203.19) -- cycle ;
        %Shape: Ellipse [id:dp9620130396250384] 
        \draw  [fill={rgb, 255:red, 255; green, 255; blue, 255 }  ,fill opacity=1 ] (78.49,203.19) .. controls (78.49,202.6) and (78.97,202.12) .. (79.56,202.12) .. controls (80.15,202.12) and (80.63,202.6) .. (80.63,203.19) .. controls (80.63,203.78) and (80.15,204.26) .. (79.56,204.26) .. controls (78.97,204.26) and (78.49,203.78) .. (78.49,203.19) -- cycle ;
        
        %Shape: Rectangle [id:dp8525312211512277] 
        \draw  [fill={rgb, 255:red, 0; green, 0; blue, 0 }  ,fill opacity=1 ] (71.11,161.04) -- (97.25,161.04) -- (97.25,173.8) -- (71.11,173.8) -- cycle ;
        %Shape: Cross [id:dp8060460094339914] 
        \draw  [fill={rgb, 255:red, 255; green, 255; blue, 255 }  ,fill opacity=1 ] (78.57,162.45) -- (75.92,162.45) -- (75.92,164.32) -- (74.06,164.32) -- (74.06,166.96) -- (75.92,166.96) -- (75.92,168.83) -- (78.57,168.83) -- (78.57,166.96) -- (80.44,166.96) -- (80.44,164.32) -- (78.57,164.32) -- cycle ;
        %Shape: Rectangle [id:dp9936314719832327] 
        \draw  [fill={rgb, 255:red, 255; green, 255; blue, 255 }  ,fill opacity=1 ] (87.67,164.18) -- (94.1,164.18) -- (94.1,167.05) -- (87.67,167.05) -- cycle ;
        %Shape: Rectangle [id:dp2616738616393308] 
        \draw  [fill={rgb, 255:red, 0; green, 0; blue, 0 }  ,fill opacity=1 ] (75.77,157.74) -- (79.03,157.74) -- (79.03,158.73) -- (75.77,158.73) -- cycle ;
        %Shape: Rectangle [id:dp6509883524938491] 
        \draw  [fill={rgb, 255:red, 0; green, 0; blue, 0 }  ,fill opacity=1 ] (89.63,157.74) -- (92.9,157.74) -- (92.9,158.73) -- (89.63,158.73) -- cycle ;
        
        %Shape: Trapezoid [id:dp8014050339336294] 
        \draw   (18.18,188.03) -- (19.92,177.46) -- (31.92,177.46) -- (33.65,188.03) -- cycle ;
        %Shape: Trapezoid [id:dp5103731797077524] 
        \draw   (19.92,177.46) -- (21.27,168.62) -- (30.57,168.62) -- (31.92,177.46) -- cycle ;
        %Shape: Trapezoid [id:dp10953227594536252] 
        \draw   (21.27,168.62) -- (21.27,160.83) -- (30.57,160.83) -- (30.57,168.62) -- cycle ;
        %Shape: Trapezoid [id:dp9070958451388693] 
        \draw   (21.27,160.83) -- (21.27,150.7) -- (30.57,150.7) -- (30.57,160.83) -- cycle ;
        %Straight Lines [id:da494003956341327] 
        \draw    (19.92,177.46) -- (33.65,188.03) ;
        %Straight Lines [id:da21997936724937683] 
        \draw    (18.18,188.03) -- (31.92,177.46) ;
        %Straight Lines [id:da1521234963259528] 
        \draw    (19.92,177.46) -- (30.57,168.62) ;
        %Straight Lines [id:da8800065517008677] 
        \draw    (31.92,177.46) -- (21.27,168.62) ;
        %Straight Lines [id:da10132037023465568] 
        \draw    (30.57,168.62) -- (21.27,160.83) ;
        %Straight Lines [id:da8857169272489369] 
        \draw    (21.27,168.62) -- (30.57,160.83) ;
        %Straight Lines [id:da18940911218505718] 
        \draw    (21.27,160.83) -- (30.57,150.7) ;
        %Straight Lines [id:da2737194992684513] 
        \draw    (30.57,160.83) -- (21.27,150.7) ;
        %Shape: Right Triangle [id:dp6435286455283047] 
        \draw   (30.57,154.5) -- (43.63,160.83) -- (30.57,160.83) -- cycle ;
        %Straight Lines [id:da8747959343770428] 
        \draw    (30.57,160.83) -- (34.85,156.57) ;
        %Straight Lines [id:da6386041455977634] 
        \draw    (40.18,160.97) -- (40.18,163.1) ;
        
        %Shape: Right Triangle [id:dp7539180207710741] 
        \draw   (21.27,154.5) -- (8.2,160.83) -- (21.27,160.83) -- cycle ;
        %Straight Lines [id:da807382544873495] 
        \draw    (17.12,156.83) -- (21.27,160.83) ;
        %Straight Lines [id:da0548293598568661] 
        \draw    (11.78,160.97) -- (11.78,163.1) ;
        
        %Straight Lines [id:da0738665156663838] 
        \draw [color={rgb, 255:red, 155; green, 155; blue, 155 }  ,draw opacity=1 ]   (127.33,167.67) -- (127.33,235.87) ;
        %Straight Lines [id:da0862753004176684] 
        \draw [color={rgb, 255:red, 155; green, 155; blue, 155 }  ,draw opacity=1 ]   (127.33,235.87) -- (105.47,235.87) ;
        %Straight Lines [id:da4998499727969492] 
        \draw [color={rgb, 255:red, 155; green, 155; blue, 155 }  ,draw opacity=1 ]   (84.19,222.64) -- (84.19,210.92) ;
        \draw [shift={(84.19,208.92)}, rotate = 90] [fill={rgb, 255:red, 155; green, 155; blue, 155 }  ,fill opacity=1 ][line width=0.08]  [draw opacity=0] (8.4,-2.1) -- (0,0) -- (8.4,2.1) -- cycle    ;
        %Straight Lines [id:da7391398944172984] 
        \draw [color={rgb, 255:red, 155; green, 155; blue, 155 }  ,draw opacity=1 ]   (83.61,177.32) -- (83.61,189.04) ;
        \draw [shift={(83.61,191.04)}, rotate = 270] [fill={rgb, 255:red, 155; green, 155; blue, 155 }  ,fill opacity=1 ][line width=0.08]  [draw opacity=0] (8.4,-2.1) -- (0,0) -- (8.4,2.1) -- cycle    ;
        %Straight Lines [id:da8065620719950461] 
        \draw [color={rgb, 255:red, 155; green, 155; blue, 155 }  ,draw opacity=1 ]   (57.35,201.25) -- (26.2,201.25) ;
        \draw [shift={(59.35,201.25)}, rotate = 180] [fill={rgb, 255:red, 155; green, 155; blue, 155 }  ,fill opacity=1 ][line width=0.08]  [draw opacity=0] (8.4,-2.1) -- (0,0) -- (8.4,2.1) -- cycle    ;
        %Straight Lines [id:da9492164262292073] 
        \draw [color={rgb, 255:red, 155; green, 155; blue, 155 }  ,draw opacity=1 ]   (66.17,167.17) -- (37,167.17) ;
        \draw [shift={(68.17,167.17)}, rotate = 180] [fill={rgb, 255:red, 155; green, 155; blue, 155 }  ,fill opacity=1 ][line width=0.08]  [draw opacity=0] (8.4,-2.1) -- (0,0) -- (8.4,2.1) -- cycle    ;
        %Straight Lines [id:da05111815853712898] 
        \draw [color={rgb, 255:red, 155; green, 155; blue, 155 }  ,draw opacity=1 ]   (127.33,167.67) -- (103.27,167.67) ;
        \draw [shift={(101.27,167.67)}, rotate = 360] [fill={rgb, 255:red, 155; green, 155; blue, 155 }  ,fill opacity=1 ][line width=0.08]  [draw opacity=0] (8.4,-2.1) -- (0,0) -- (8.4,2.1) -- cycle    ;
        %Shape: Can [id:dp06540455485988761] 
        \draw  [color={rgb, 255:red, 255; green, 255; blue, 255 }  ,draw opacity=1 ][fill={rgb, 255:red, 0; green, 0; blue, 0 }  ,fill opacity=1 ][line width=1.5]  (211.47,196.16) -- (211.47,203.25) .. controls (211.47,205.21) and (221.25,206.8) .. (233.32,206.8) .. controls (245.39,206.8) and (255.18,205.21) .. (255.18,203.25) -- (255.18,196.16) .. controls (255.18,194.2) and (245.39,192.61) .. (233.32,192.61) .. controls (221.25,192.61) and (211.47,194.2) .. (211.47,196.16) .. controls (211.47,198.12) and (221.25,199.7) .. (233.32,199.7) .. controls (245.39,199.7) and (255.18,198.12) .. (255.18,196.16) ;
        %Shape: Ellipse [id:dp7703012030788203] 
        \draw  [fill={rgb, 255:red, 255; green, 255; blue, 255 }  ,fill opacity=1 ] (234.25,202.86) .. controls (234.25,202.26) and (233.77,201.79) .. (233.18,201.79) .. controls (232.59,201.79) and (232.11,202.26) .. (232.11,202.86) .. controls (232.11,203.45) and (232.59,203.93) .. (233.18,203.93) .. controls (233.77,203.93) and (234.25,203.45) .. (234.25,202.86) -- cycle ;
        %Shape: Ellipse [id:dp690264748979571] 
        \draw  [fill={rgb, 255:red, 255; green, 255; blue, 255 }  ,fill opacity=1 ] (230.45,202.86) .. controls (230.45,202.26) and (229.97,201.79) .. (229.38,201.79) .. controls (228.79,201.79) and (228.31,202.26) .. (228.31,202.86) .. controls (228.31,203.45) and (228.79,203.93) .. (229.38,203.93) .. controls (229.97,203.93) and (230.45,203.45) .. (230.45,202.86) -- cycle ;
        %Shape: Ellipse [id:dp874524076657855] 
        \draw  [fill={rgb, 255:red, 255; green, 255; blue, 255 }  ,fill opacity=1 ] (238.04,202.86) .. controls (238.04,202.26) and (237.56,201.79) .. (236.97,201.79) .. controls (236.38,201.79) and (235.9,202.26) .. (235.9,202.86) .. controls (235.9,203.45) and (236.38,203.93) .. (236.97,203.93) .. controls (237.56,203.93) and (238.04,203.45) .. (238.04,202.86) -- cycle ;
        
        %Shape: Trapezoid [id:dp09868071238769627] 
        \draw   (298.02,188.03) -- (296.28,177.46) -- (284.28,177.46) -- (282.55,188.03) -- cycle ;
        %Shape: Trapezoid [id:dp034648918462727885] 
        \draw   (296.28,177.46) -- (294.93,168.62) -- (285.63,168.62) -- (284.28,177.46) -- cycle ;
        %Shape: Trapezoid [id:dp810755830233199] 
        \draw   (294.93,168.62) -- (294.93,160.83) -- (285.63,160.83) -- (285.63,168.62) -- cycle ;
        %Shape: Trapezoid [id:dp8211199144861359] 
        \draw   (294.93,160.83) -- (294.93,150.7) -- (285.63,150.7) -- (285.63,160.83) -- cycle ;
        %Straight Lines [id:da02829171462409863] 
        \draw    (296.28,177.46) -- (282.55,188.03) ;
        %Straight Lines [id:da3842553754005844] 
        \draw    (298.02,188.03) -- (284.28,177.46) ;
        %Straight Lines [id:da3249880328420971] 
        \draw    (296.28,177.46) -- (285.63,168.62) ;
        %Straight Lines [id:da3947792446647116] 
        \draw    (284.28,177.46) -- (294.93,168.62) ;
        %Straight Lines [id:da06810762328258235] 
        \draw    (285.63,168.62) -- (294.93,160.83) ;
        %Straight Lines [id:da888796877101043] 
        \draw    (294.93,168.62) -- (285.63,160.83) ;
        %Straight Lines [id:da5122985177480841] 
        \draw    (294.93,160.83) -- (285.63,150.7) ;
        %Straight Lines [id:da5234196061850045] 
        \draw    (285.63,160.83) -- (294.93,150.7) ;
        %Shape: Right Triangle [id:dp7077892858287509] 
        \draw   (285.63,154.5) -- (272.56,160.83) -- (285.63,160.83) -- cycle ;
        %Straight Lines [id:da9718242440072744] 
        \draw    (285.63,160.83) -- (281.35,156.57) ;
        %Straight Lines [id:da7353802852905165] 
        \draw    (276.02,160.97) -- (276.02,163.1) ;
        
        %Shape: Right Triangle [id:dp3984437149644189] 
        \draw   (294.93,154.5) -- (308,160.83) -- (294.93,160.83) -- cycle ;
        %Straight Lines [id:da3442811381013058] 
        \draw    (299.08,156.83) -- (294.93,160.83) ;
        %Straight Lines [id:da7129483859197401] 
        \draw    (304.42,160.97) -- (304.42,163.1) ;
        
        %Straight Lines [id:da14689072222002553] 
        \draw [color={rgb, 255:red, 155; green, 155; blue, 155 }  ,draw opacity=1 ]   (290.53,200.33) -- (290.53,191.67) ;
        %Straight Lines [id:da7704282875494002] 
        \draw [color={rgb, 255:red, 155; green, 155; blue, 155 }  ,draw opacity=1 ]   (189.2,170.03) -- (189.2,235.53) ;
        %Straight Lines [id:da6412519535680179] 
        \draw [color={rgb, 255:red, 155; green, 155; blue, 155 }  ,draw opacity=1 ]   (189.2,235.53) -- (211.07,235.53) ;
        %Straight Lines [id:da22705010839146178] 
        \draw [color={rgb, 255:red, 155; green, 155; blue, 155 }  ,draw opacity=1 ]   (232.34,222.3) -- (232.34,210.59) ;
        \draw [shift={(232.34,208.59)}, rotate = 90] [fill={rgb, 255:red, 155; green, 155; blue, 155 }  ,fill opacity=1 ][line width=0.08]  [draw opacity=0] (8.4,-2.1) -- (0,0) -- (8.4,2.1) -- cycle    ;
        %Straight Lines [id:da5645110675639073] 
        \draw [color={rgb, 255:red, 155; green, 155; blue, 155 }  ,draw opacity=1 ]   (232.92,176.99) -- (232.92,188.7) ;
        \draw [shift={(232.92,190.7)}, rotate = 270] [fill={rgb, 255:red, 155; green, 155; blue, 155 }  ,fill opacity=1 ][line width=0.08]  [draw opacity=0] (8.4,-2.1) -- (0,0) -- (8.4,2.1) -- cycle    ;
        %Straight Lines [id:da5869044347840864] 
        \draw [color={rgb, 255:red, 155; green, 155; blue, 155 }  ,draw opacity=1 ]   (251.18,169.59) -- (279.83,169.59) ;
        \draw [shift={(249.18,169.59)}, rotate = 0] [fill={rgb, 255:red, 155; green, 155; blue, 155 }  ,fill opacity=1 ][line width=0.08]  [draw opacity=0] (8.4,-2.1) -- (0,0) -- (8.4,2.1) -- cycle    ;
        %Straight Lines [id:da8297605304246554] 
        \draw [color={rgb, 255:red, 155; green, 155; blue, 155 }  ,draw opacity=1 ]   (259,200.33) -- (290.53,200.33) ;
        \draw [shift={(257,200.33)}, rotate = 0] [fill={rgb, 255:red, 155; green, 155; blue, 155 }  ,fill opacity=1 ][line width=0.08]  [draw opacity=0] (8.4,-2.1) -- (0,0) -- (8.4,2.1) -- cycle    ;
        %Straight Lines [id:da13925604612830966] 
        \draw [color={rgb, 255:red, 155; green, 155; blue, 155 }  ,draw opacity=1 ]   (189.2,170.03) -- (213.27,170.03) ;
        \draw [shift={(215.27,170.03)}, rotate = 180] [fill={rgb, 255:red, 155; green, 155; blue, 155 }  ,fill opacity=1 ][line width=0.08]  [draw opacity=0] (8.4,-2.1) -- (0,0) -- (8.4,2.1) -- cycle    ;
        %Shape: Ellipse [id:dp5355339043604253] 
        \draw  [color={rgb, 255:red, 255; green, 255; blue, 255 }  ,draw opacity=1 ][fill={rgb, 255:red, 0; green, 0; blue, 0 }  ,fill opacity=1 ] (240.32,225.59) .. controls (240.32,222.5) and (242.82,220) .. (245.91,220) .. controls (248.99,220) and (251.5,222.5) .. (251.5,225.59) .. controls (251.5,228.68) and (248.99,231.18) .. (245.91,231.18) .. controls (242.82,231.18) and (240.32,228.68) .. (240.32,225.59) -- cycle ;
        %Rounded Rect [id:dp04420299162045738] 
        \draw   (252.91,225.53) .. controls (252.91,225.52) and (252.92,225.52) .. (252.93,225.52) -- (255.62,225.52) .. controls (255.63,225.52) and (255.64,225.52) .. (255.64,225.53) -- (255.64,225.59) .. controls (255.64,225.6) and (255.63,225.61) .. (255.62,225.61) -- (252.93,225.61) .. controls (252.92,225.61) and (252.91,225.6) .. (252.91,225.59) -- cycle ;
        %Rounded Rect [id:dp38965807926567364] 
        \draw   (237.5,220.68) .. controls (237.5,220.67) and (237.51,220.67) .. (237.52,220.68) -- (239.85,222.02) .. controls (239.86,222.03) and (239.86,222.04) .. (239.86,222.05) -- (239.83,222.1) .. controls (239.83,222.1) and (239.82,222.11) .. (239.81,222.1) -- (237.47,220.75) .. controls (237.47,220.75) and (237.46,220.74) .. (237.47,220.73) -- cycle ;
        %Rounded Rect [id:dp26300646912581893] 
        \draw   (240.83,217.23) .. controls (240.84,217.23) and (240.85,217.23) .. (240.86,217.24) -- (242.2,219.57) .. controls (242.21,219.58) and (242.2,219.59) .. (242.2,219.6) -- (242.15,219.62) .. controls (242.14,219.63) and (242.13,219.63) .. (242.12,219.62) -- (240.78,217.28) .. controls (240.77,217.28) and (240.78,217.26) .. (240.78,217.26) -- cycle ;
        %Rounded Rect [id:dp15807748254714116] 
        \draw   (245.89,215.87) .. controls (245.9,215.87) and (245.91,215.88) .. (245.91,215.89) -- (245.91,218.58) .. controls (245.91,218.59) and (245.9,218.6) .. (245.89,218.6) -- (245.83,218.6) .. controls (245.82,218.6) and (245.82,218.59) .. (245.82,218.58) -- (245.82,215.89) .. controls (245.82,215.88) and (245.82,215.87) .. (245.83,215.87) -- cycle ;
        %Rounded Rect [id:dp9208312633419158] 
        \draw   (236.19,225.52) .. controls (236.19,225.51) and (236.2,225.5) .. (236.21,225.5) -- (238.9,225.5) .. controls (238.91,225.5) and (238.92,225.51) .. (238.92,225.52) -- (238.92,225.57) .. controls (238.92,225.58) and (238.91,225.59) .. (238.9,225.59) -- (236.21,225.59) .. controls (236.2,225.59) and (236.19,225.58) .. (236.19,225.57) -- cycle ;
        %Rounded Rect [id:dp2058087547263563] 
        \draw   (249.44,231.5) .. controls (249.45,231.49) and (249.46,231.5) .. (249.47,231.51) -- (250.82,233.84) .. controls (250.82,233.85) and (250.82,233.86) .. (250.81,233.86) -- (250.76,233.89) .. controls (250.75,233.9) and (250.74,233.89) .. (250.74,233.88) -- (249.39,231.55) .. controls (249.39,231.54) and (249.39,231.53) .. (249.4,231.53) -- cycle ;
        %Rounded Rect [id:dp5638790499914985] 
        \draw   (245.98,232.57) .. controls (245.99,232.57) and (246,232.58) .. (246,232.59) -- (246,235.29) .. controls (246,235.3) and (245.99,235.3) .. (245.98,235.3) -- (245.92,235.3) .. controls (245.91,235.3) and (245.91,235.3) .. (245.91,235.29) -- (245.91,232.59) .. controls (245.91,232.58) and (245.91,232.57) .. (245.92,232.57) -- cycle ;
        %Rounded Rect [id:dp5175856503004568] 
        \draw   (252.03,228.89) .. controls (252.04,228.88) and (252.05,228.88) .. (252.06,228.89) -- (254.39,230.23) .. controls (254.4,230.24) and (254.4,230.25) .. (254.4,230.26) -- (254.37,230.31) .. controls (254.36,230.31) and (254.35,230.32) .. (254.34,230.31) -- (252.01,228.96) .. controls (252,228.96) and (252,228.95) .. (252,228.94) -- cycle ;
        %Rounded Rect [id:dp7432273228736346] 
        \draw   (250.81,217.22) .. controls (250.82,217.23) and (250.82,217.24) .. (250.82,217.25) -- (249.47,219.58) .. controls (249.46,219.59) and (249.45,219.59) .. (249.44,219.59) -- (249.4,219.56) .. controls (249.39,219.56) and (249.39,219.54) .. (249.39,219.54) -- (250.74,217.2) .. controls (250.74,217.19) and (250.75,217.19) .. (250.76,217.2) -- cycle ;
        %Rounded Rect [id:dp1076932108654638] 
        \draw   (254.4,220.61) .. controls (254.4,220.62) and (254.4,220.63) .. (254.39,220.63) -- (252.06,221.98) .. controls (252.05,221.98) and (252.04,221.98) .. (252.03,221.97) -- (252,221.93) .. controls (252,221.92) and (252,221.91) .. (252.01,221.9) -- (254.34,220.55) .. controls (254.35,220.55) and (254.36,220.55) .. (254.37,220.56) -- cycle ;
        %Rounded Rect [id:dp2788166863414874] 
        \draw   (239.86,228.82) .. controls (239.86,228.83) and (239.86,228.84) .. (239.85,228.84) -- (237.52,230.19) .. controls (237.51,230.19) and (237.5,230.19) .. (237.5,230.18) -- (237.47,230.14) .. controls (237.46,230.13) and (237.47,230.12) .. (237.47,230.11) -- (239.81,228.76) .. controls (239.82,228.76) and (239.83,228.76) .. (239.83,228.77) -- cycle ;
        %Rounded Rect [id:dp06543329925873875] 
        \draw   (242.06,231.49) .. controls (242.07,231.49) and (242.07,231.51) .. (242.07,231.51) -- (240.72,233.85) .. controls (240.72,233.86) and (240.71,233.86) .. (240.7,233.85) -- (240.65,233.83) .. controls (240.64,233.82) and (240.64,233.81) .. (240.64,233.8) -- (241.99,231.47) .. controls (241.99,231.46) and (242.01,231.46) .. (242.01,231.46) -- cycle ;
        
        %Shape: Trapezoid [id:dp1628147856794615] 
        \draw  [color={rgb, 255:red, 255; green, 255; blue, 255 }  ,draw opacity=1 ][fill={rgb, 255:red, 0; green, 0; blue, 0 }  ,fill opacity=1 ] (218.73,242.55) -- (222.13,225.48) -- (245.21,225.48) -- (248.61,242.55) -- cycle ;
        %Straight Lines [id:da06495993283011048] 
        \draw [color={rgb, 255:red, 255; green, 255; blue, 255 }  ,draw opacity=1 ][fill={rgb, 255:red, 0; green, 0; blue, 0 }  ,fill opacity=1 ]   (219.98,237.04) -- (247.58,236.99) ;
        %Straight Lines [id:da23132450634953083] 
        \draw [color={rgb, 255:red, 255; green, 255; blue, 255 }  ,draw opacity=1 ][fill={rgb, 255:red, 0; green, 0; blue, 0 }  ,fill opacity=1 ]   (220.94,231.09) -- (246.02,231.09) ;
        %Straight Lines [id:da047337957569512534] 
        \draw [color={rgb, 255:red, 255; green, 255; blue, 255 }  ,draw opacity=1 ][fill={rgb, 255:red, 0; green, 0; blue, 0 }  ,fill opacity=1 ]   (237.21,225.97) -- (239.45,242.81) ;
        %Straight Lines [id:da911143525690777] 
        \draw [color={rgb, 255:red, 255; green, 255; blue, 255 }  ,draw opacity=1 ][fill={rgb, 255:red, 0; green, 0; blue, 0 }  ,fill opacity=1 ]   (230.13,225.81) -- (228.13,242.93) ;
        
        
        %Shape: Rectangle [id:dp3290889365498213] 
        \draw  [fill={rgb, 255:red, 0; green, 0; blue, 0 }  ,fill opacity=1 ] (219.11,161.04) -- (245.25,161.04) -- (245.25,173.8) -- (219.11,173.8) -- cycle ;
        %Shape: Cross [id:dp9068586922320692] 
        \draw  [fill={rgb, 255:red, 255; green, 255; blue, 255 }  ,fill opacity=1 ] (226.57,162.45) -- (223.92,162.45) -- (223.92,164.32) -- (222.06,164.32) -- (222.06,166.96) -- (223.92,166.96) -- (223.92,168.83) -- (226.57,168.83) -- (226.57,166.96) -- (228.44,166.96) -- (228.44,164.32) -- (226.57,164.32) -- cycle ;
        %Shape: Rectangle [id:dp9992665368326406] 
        \draw  [fill={rgb, 255:red, 255; green, 255; blue, 255 }  ,fill opacity=1 ] (235.67,164.18) -- (242.1,164.18) -- (242.1,167.05) -- (235.67,167.05) -- cycle ;
        %Shape: Rectangle [id:dp8155202601009173] 
        \draw  [fill={rgb, 255:red, 0; green, 0; blue, 0 }  ,fill opacity=1 ] (223.77,157.74) -- (227.03,157.74) -- (227.03,158.73) -- (223.77,158.73) -- cycle ;
        %Shape: Rectangle [id:dp7437450370853582] 
        \draw  [fill={rgb, 255:red, 0; green, 0; blue, 0 }  ,fill opacity=1 ] (237.63,157.74) -- (240.9,157.74) -- (240.9,158.73) -- (237.63,158.73) -- cycle ;
        
        %Shape: Rectangle [id:dp7773542380618799] 
        \draw   (130.57,113.67) -- (182.43,113.67) -- (182.43,139.67) -- (130.57,139.67) -- cycle ;
        %Shape: Arc [id:dp5650806587626209] 
        \draw  [draw opacity=0] (149.13,121.36) .. controls (150.73,118.89) and (153.52,117.25) .. (156.68,117.25) .. controls (161.64,117.25) and (165.66,121.27) .. (165.66,126.22) .. controls (165.66,127.13) and (165.52,128) .. (165.27,128.83) -- (156.68,126.22) -- cycle ; \draw    (149.13,121.36) .. controls (150.73,118.89) and (153.52,117.25) .. (156.68,117.25) .. controls (161.64,117.25) and (165.66,121.27) .. (165.66,126.22) ; \draw [shift={(165.27,128.83)}, rotate = 274.15] [fill={rgb, 255:red, 0; green, 0; blue, 0 }  ][line width=0.08]  [draw opacity=0] (5.36,-2.57) -- (0,0) -- (5.36,2.57) -- (3.56,0) -- cycle    ; 
        %Shape: Arc [id:dp6862645112098824] 
        \draw  [draw opacity=0] (164.23,131.09) .. controls (162.63,133.56) and (159.85,135.2) .. (156.68,135.2) .. controls (151.72,135.2) and (147.7,131.18) .. (147.7,126.22) .. controls (147.7,125.32) and (147.84,124.44) .. (148.09,123.62) -- (156.68,126.22) -- cycle ; \draw    (164.23,131.09) .. controls (162.63,133.56) and (159.85,135.2) .. (156.68,135.2) .. controls (151.72,135.2) and (147.7,131.18) .. (147.7,126.22) ; \draw [shift={(148.09,123.62)}, rotate = 94.15] [fill={rgb, 255:red, 0; green, 0; blue, 0 }  ][line width=0.08]  [draw opacity=0] (5.36,-2.57) -- (0,0) -- (5.36,2.57) -- (3.56,0) -- cycle    ; 
        %Shape: Triangle [id:dp6807873925684846] 
        \draw  [fill={rgb, 255:red, 0; green, 0; blue, 0 }  ,fill opacity=1 ] (156.68,126.22) -- (158.27,126.25) -- (154.49,132.8) -- cycle ;
        %Shape: Triangle [id:dp2517297573327719] 
        \draw  [fill={rgb, 255:red, 0; green, 0; blue, 0 }  ,fill opacity=1 ] (156.68,126.22) -- (155.1,126.2) -- (158.88,119.65) -- cycle ;
        %Straight Lines [id:da2661412932973988] 
        \draw [color={rgb, 255:red, 155; green, 155; blue, 155 }  ,draw opacity=1 ]   (149.67,99.67) -- (149.67,111.53) ;
        %Straight Lines [id:da916646003941181] 
        \draw [color={rgb, 255:red, 155; green, 155; blue, 155 }  ,draw opacity=1 ]   (86.07,133.53) -- (125.67,133.53) ;
        %Straight Lines [id:da2596197330522467] 
        \draw [color={rgb, 255:red, 155; green, 155; blue, 155 }  ,draw opacity=1 ]   (86.07,133.53) -- (86.07,155.33) ;
        %Straight Lines [id:da19220466082958354] 
        \draw [color={rgb, 255:red, 155; green, 155; blue, 155 }  ,draw opacity=1 ]   (186.07,133.53) -- (230.83,133.53) ;
        %Straight Lines [id:da5201478983031171] 
        \draw [color={rgb, 255:red, 155; green, 155; blue, 155 }  ,draw opacity=1 ]   (230.83,133.53) -- (230.83,155) ;
        %Shape: Rectangle [id:dp9278859435215474] 
        \draw  [color={rgb, 255:red, 155; green, 155; blue, 155 }  ,draw opacity=1 ][dash pattern={on 0.84pt off 2.51pt}] (3.67,145.67) -- (133.67,145.67) -- (133.67,248.33) -- (3.67,248.33) -- cycle ;
        %Shape: Rectangle [id:dp6748218140116866] 
        \draw  [color={rgb, 255:red, 155; green, 155; blue, 155 }  ,draw opacity=1 ][dash pattern={on 0.84pt off 2.51pt}] (182,145.02) -- (312.67,145.02) -- (312.67,249.33) -- (182,249.33) -- cycle ;
        %Shape: Rectangle [id:dp12206071137136631] 
        \draw  [color={rgb, 255:red, 155; green, 155; blue, 155 }  ,draw opacity=1 ][dash pattern={on 0.84pt off 2.51pt}] (90.33,4.33) -- (247,4.33) -- (247,104.33) -- (90.33,104.33) -- cycle ;
        %Straight Lines [id:da0966143636817669] 
        \draw [color={rgb, 255:red, 155; green, 155; blue, 155 }  ,draw opacity=1 ]   (26.2,201.25) -- (26.2,190.73) ;
        
        % Text Node
        \draw (142,142) node [anchor=north west][inner sep=0.75pt]   [align=left] {BMS};
        
    
    \end{tikzpicture}

	\end{center}
	\caption{An edge computing network consists of the substructures, each of which includes an edge computing servers, batteries, renewable energy sources, a BMS, and the grids. The arrows represent the directions of energy flow.}\label{fig:system_model}
\end{figure}









\subsection{System Model}
Fig. \ref{fig:system_model} shows $S$ sites (illustrated by dashed rectangles) forming an edge computing network. Let $\mathcal{S} = \{1,2,...,S\}$. Each site consists of an edge computing server, a battery, a local renewable energy source, and the local power grid. The servers have the same specification and computing power. The batteries are connected to each other via a battery management system (BMS), hence the energy in a battery can be shared with the servers on the other sites\footnote{The BMS has been studies widely, see for example \cite{leithon2013online} and \cite{leithon2019task}}. In this paper, we use the index of the site to refer to its components, e.g., server $s$, battery $s$, etc. More specifics of the system model are as follows;
\begin{itemize}
    \item A server can be powered by its local grid, its renewable energy source, and all the batteries. 
    \item A local battery can be charged by the local grid and the local renewable energy source. Additionally, the battery can provide energy to any other server. However, this comes with an energy transfer loss.
\end{itemize}

\subsection{Time Horizon}
We consider a scheduling horizon of $T$ time slots, denoted by $\mathcal{T} = \{1,2,...,T\}$. The set of tasks over the entire time horizon is represented by $\mathcal{N} = \{1, 2, ..., N\}$. In this paper, the tasks are assumed to be of the same type, i.e., the amount of energy to complete them is uniform, denoted by $E$. Without loss of generality, we use $E$ as our energy unit, and the relevant parameters (e.g., $R$-, $I$-, $\alpha$-, $\beta$-parameters, $L$, and $H$ to be introduced next) are all normalized by $E$, i.e., they are given as some multiples of $E$. 

Task $n$ is represented by a tuple $(o_n, d_n, s_n, \mathcal{S}_n)$, where $o_n, d_n \in \mathcal{T}, s_n \in \mathcal{S}, \mathcal{S}_n \subseteq \mathcal{S}$ such that
\begin{itemize}
    \item $o_n$ is the time slot when the task is generated;
    \item $d_n$ is the deadline of completing the task;
    \item $s_n$ is the server that the task is initially associated with;
    \item $\mathcal{S}_n$ represents the set of candidate servers\footnote{Some tasks might not want to be performed in some servers because of, for example, latency due to the distance.} that can perform task $n$.
\end{itemize}
A task is to be completed in some time slot of the interval $[o_n, d_n]$.

Denote by $\pi_{nst}$ a binary variable that is one if and only if task $n$ is completed by server $s$ in time slot $t$. For any task, it needs to be completed once, so we have
\begin{equation}\label{c0}
    \sum_{s \in \mathcal{S}} \sum^{d_n}_{t = o_n} \pi_{nst} = 1, \forall n \in \mathcal{N}.
\end{equation}

In time slot $t$, the amount of energy consumed to task completion by server $s$ is $\sum_{n \in \mathcal{N}} \pi_{nst}$. For the consumed energy, we use variables $x_{st}$, $y_{ss^\prime t}$, and $z_{st}$ to represent the amount of energy (again normalized by $E$) to server $s$ from the local grid, battery $s^\prime$ and the local renewable source, respectively, in time slot $t$. Then we have
\begin{equation}\label{c1}
     x_{st} + \sum_{s^\prime \in S}  y _{ss^\prime t} + z_{st} = \sum_{n \in \mathcal{N}} \pi_{nst}, \forall s \in \mathcal{S}, t \in \mathcal{T}.
\end{equation}
In addition, the maximum number of tasks a server can complete in a time slot can be known by the computing capacity, denoted by $H$. Note that the required energy of a task is one unit, so we have the following constraint:
\begin{equation}\label{c4}
    x_{st}+\sum_{s^\prime \in\mathcal{S}} y_{ss^\prime t} +z_{st} \leq H, \forall s \in \mathcal{S}, t \in \mathcal{T}.
\end{equation}

For charging battery $s$ at the beginning of time slot $t$, we use variables $u_{st}$ and $v_{st}$ to represent the amounts of energy from the local grid and the renewable source, respectively. In addition, denote by variable $w_{st}$ the amount of the remaining energy of battery $s$ at the end of time slot $t$. We use $L$ to present the capacity of the battery. At the beginning of time slot $t$, we have the following battery capacity constraint:
\begin{equation}\label{c2}
    u_{st} + v_{st} + w_{s(t-1)} \leq L, \forall s \in \mathcal{S}, t \in \mathcal{T} \setminus \{1\}.
\end{equation}
At the end of that, the remaining energy $w_{st}$ is calculated by
\begin{equation}\label{c3}
    w_{st} = u_{st} + v_{st} + w_{s(t-1)} -\sum_{s^\prime \in \mathcal{S}} y_{ss^\prime t}, \forall s \in \mathcal{S}, t \in \mathcal{T} \setminus \{1\}.
\end{equation}

We use $R_{st}$ to denote the amount of renewable energy\footnote{Renewable energy can be predicted, see \cite{li2019renewable}.} available from source $s$ in time slot $t$. The renewable energy is used for charging the local battery and supplying to the local server, so we have  
\begin{equation}\label{c5}
    z_{st} + u_{st} \leq R_{st}, \forall s \in \mathcal{S}, t \in \mathcal{T}.
\end{equation}

\subsection{Carbon Footprint and Formulation} \label{subsec:CFA}

In the system model, the CF occurs only in three processes related to the grids, including battery charging, server consumption, and task offloading. Denote by $I_{st}$ the CI of local grid $s$ in time slot $t$. The three types of CF can be calculated as follows.
\begin{enumerate}
    \item {\em Server consumption}: The CF will occur when a server consumes the energy from the local grid to complete task. The total amount of CF of that can be expressed as
    \begin{equation}\label{th}
        Sc(\boldsymbol{x}) = \sum_{t \in \mathcal{T}} \sum_{s\in \mathcal{S}} I_{st}x_{st}.
    \end{equation}
    \item {\em Battery charging}: The total amount of CF in battery charging via the local grid can be calculated by
    \begin{equation}\label{bc}
        Bc(\boldsymbol{u}) = \sum_{t \in \mathcal{T}} \sum_{s\in \mathcal{S}} I_{st}u_{st}.
    \end{equation}
    \item {\em Task offloading}: In this paper, we consider the worst case of the CF in task offloading, i.e., we assume that the energy used for task offloading is all from the grid. We use $\alpha_{s_n s}$ to represent the required amount of the energy for transferring task $n$ from its initial server $s_n$ to $s$ ($\alpha_{s_n s} = 0$ if $s_n = s$). So the total amount of CF of task transfer in the network is given by
    \begin{align}\label{tt}
        Tt(\boldsymbol{\pi}) =  \sum_{n \in \mathcal{N}} \sum_{s \in \mathcal{S}_n}  \sum^{d_n}_{t = o_n} \alpha_{s_n s} I_{st}  \pi_{nst}.
    \end{align}
\end{enumerate}
In addition, there are losses in the energy transfer between the battery to the servers on other sites. We convert this loss to an equivalent CF such that we can analyze the CF more explicitly. Furthermore, we denote the loss per unit of energy from the battery $s^\prime$ to server $s$ by $\beta_{s^\prime s}$ ($\beta_{s^\prime s} = 0$ if $s^\prime = s$). We suppose that local grid $s$ provides energy to make up for the loss, so the total amount of the equivalent CF with respect to loss is 
\begin{equation}\label{lo}
    Lo(\boldsymbol{y}) = \sum_{s \in \mathcal{S}}  \sum_{t \in \mathcal{T}}\sum_{s^\prime \in S_n}  \beta_{s^\prime s} I_{st} y _{ss^\prime t}.
\end{equation}

We consider minimizing the total CF in the edge computing network via offloading and scheduling the tasks, and charging and discharging the batteries. The CF minimization problem can be formulated by integer linear programming (ILP) as follows,
\begin{subequations}\label{formulation}
    \begin{align}
         \mathop{\min}_{\boldsymbol{\pi}, \boldsymbol{x}, \boldsymbol{y}, \boldsymbol{z}, \boldsymbol{u}, \boldsymbol{v}, \boldsymbol{w}} \ &Sc(\boldsymbol{x}) +  Bc(\boldsymbol{u}) + Tt(\boldsymbol{\pi}) + Lo(\boldsymbol{y}) \label{obj}\\
         \text{s.t.} & \quad \eqref{c0} - \eqref{c5},\notag\\
         & \quad \pi\text{-variables} \in \{0,1\},
    \end{align}
\end{subequations}
where the objective function \eqref{obj} is the overall CF of the network.

\section{Problem Solving} 

% \begin{figure*}[t]
    \begin{center}





\tikzset{every picture/.style={line width=0.75pt}} %set default line width to 0.75pt        

\begin{tikzpicture}[x=0.75pt,y=0.75pt,yscale=-1,xscale=1]
%uncomment if require: \path (0,398); %set diagram left start at 0, and has height of 398

%Shape: Circle [id:dp593919605681948] 
\draw  [color={rgb, 255:red, 208; green, 2; blue, 27 }  ,draw opacity=1 ] (55.83,156.5) .. controls (55.83,151.25) and (60.09,147) .. (65.33,147) .. controls (70.58,147) and (74.83,151.25) .. (74.83,156.5) .. controls (74.83,161.75) and (70.58,166) .. (65.33,166) .. controls (60.09,166) and (55.83,161.75) .. (55.83,156.5) -- cycle ;
%Shape: Circle [id:dp4139267441151957] 
\draw  [color={rgb, 255:red, 208; green, 2; blue, 27 }  ,draw opacity=1 ] (101.17,156.5) .. controls (101.17,151.25) and (105.42,147) .. (110.67,147) .. controls (115.91,147) and (120.17,151.25) .. (120.17,156.5) .. controls (120.17,161.75) and (115.91,166) .. (110.67,166) .. controls (105.42,166) and (101.17,161.75) .. (101.17,156.5) -- cycle ;
%Straight Lines [id:da08608125520746612] 
\draw    (74.83,156.5) -- (99.17,156.5) ;
\draw [shift={(101.17,156.5)}, rotate = 180] [fill={rgb, 255:red, 0; green, 0; blue, 0 }  ][line width=0.08]  [draw opacity=0] (7.2,-1.8) -- (0,0) -- (7.2,1.8) -- cycle    ;
%Shape: Circle [id:dp9182179445681371] 
\draw  [color={rgb, 255:red, 208; green, 2; blue, 27 }  ,draw opacity=1 ] (55.8,179.83) .. controls (55.8,174.59) and (60.05,170.33) .. (65.3,170.33) .. controls (70.55,170.33) and (74.8,174.59) .. (74.8,179.83) .. controls (74.8,185.08) and (70.55,189.33) .. (65.3,189.33) .. controls (60.05,189.33) and (55.8,185.08) .. (55.8,179.83) -- cycle ;
%Shape: Circle [id:dp9216202146408428] 
\draw  [color={rgb, 255:red, 208; green, 2; blue, 27 }  ,draw opacity=1 ] (101.13,179.83) .. controls (101.13,174.59) and (105.39,170.33) .. (110.63,170.33) .. controls (115.88,170.33) and (120.13,174.59) .. (120.13,179.83) .. controls (120.13,185.08) and (115.88,189.33) .. (110.63,189.33) .. controls (105.39,189.33) and (101.13,185.08) .. (101.13,179.83) -- cycle ;
%Straight Lines [id:da044823890331016525] 
\draw    (74.8,179.83) -- (99.13,179.83) ;
\draw [shift={(101.13,179.83)}, rotate = 180] [fill={rgb, 255:red, 0; green, 0; blue, 0 }  ][line width=0.08]  [draw opacity=0] (7.2,-1.8) -- (0,0) -- (7.2,1.8) -- cycle    ;

%Shape: Circle [id:dp48267737952322975] 
\draw  [color={rgb, 255:red, 208; green, 2; blue, 27 }  ,draw opacity=1 ] (55.8,216.83) .. controls (55.8,211.59) and (60.05,207.33) .. (65.3,207.33) .. controls (70.55,207.33) and (74.8,211.59) .. (74.8,216.83) .. controls (74.8,222.08) and (70.55,226.33) .. (65.3,226.33) .. controls (60.05,226.33) and (55.8,222.08) .. (55.8,216.83) -- cycle ;
%Shape: Circle [id:dp6311745806556641] 
\draw  [color={rgb, 255:red, 208; green, 2; blue, 27 }  ,draw opacity=1 ] (101.13,216.83) .. controls (101.13,211.59) and (105.39,207.33) .. (110.63,207.33) .. controls (115.88,207.33) and (120.13,211.59) .. (120.13,216.83) .. controls (120.13,222.08) and (115.88,226.33) .. (110.63,226.33) .. controls (105.39,226.33) and (101.13,222.08) .. (101.13,216.83) -- cycle ;
%Straight Lines [id:da07201812063383062] 
\draw    (74.8,216.83) -- (99.13,216.83) ;
\draw [shift={(101.13,216.83)}, rotate = 180] [fill={rgb, 255:red, 0; green, 0; blue, 0 }  ][line width=0.08]  [draw opacity=0] (7.2,-1.8) -- (0,0) -- (7.2,1.8) -- cycle    ;

%Straight Lines [id:da4517027414947885] 
\draw [color={rgb, 255:red, 155; green, 155; blue, 155 }  ,draw opacity=1 ] [dash pattern={on 0.84pt off 2.51pt}]  (243.67,0.75) -- (243.67,315.17) ;
%Shape: Circle [id:dp44439668591732095] 
\draw  [color={rgb, 255:red, 126; green, 211; blue, 33 }  ,draw opacity=1 ] (15,69.5) .. controls (15,64.25) and (19.25,60) .. (24.5,60) .. controls (29.75,60) and (34,64.25) .. (34,69.5) .. controls (34,74.75) and (29.75,79) .. (24.5,79) .. controls (19.25,79) and (15,74.75) .. (15,69.5) -- cycle ;
%Shape: Circle [id:dp9621130956289814] 
\draw  [color={rgb, 255:red, 126; green, 211; blue, 33 }  ,draw opacity=1 ] (15,92.8) .. controls (15,87.55) and (19.25,83.3) .. (24.5,83.3) .. controls (29.75,83.3) and (34,87.55) .. (34,92.8) .. controls (34,98.05) and (29.75,102.3) .. (24.5,102.3) .. controls (19.25,102.3) and (15,98.05) .. (15,92.8) -- cycle ;
%Shape: Circle [id:dp6789068245318313] 
\draw  [color={rgb, 255:red, 126; green, 211; blue, 33 }  ,draw opacity=1 ] (15,129.8) .. controls (15,124.55) and (19.25,120.3) .. (24.5,120.3) .. controls (29.75,120.3) and (34,124.55) .. (34,129.8) .. controls (34,135.05) and (29.75,139.3) .. (24.5,139.3) .. controls (19.25,139.3) and (15,135.05) .. (15,129.8) -- cycle ;
%Straight Lines [id:da9900400553640198] 
\draw    (31.25,76.75) -- (60.96,145.41) ;
\draw [shift={(61.75,147.25)}, rotate = 246.61] [fill={rgb, 255:red, 0; green, 0; blue, 0 }  ][line width=0.08]  [draw opacity=0] (7.2,-1.8) -- (0,0) -- (7.2,1.8) -- cycle    ;
%Straight Lines [id:da6873568632040385] 
\draw    (29.5,101.3) -- (57.5,170.89) ;
\draw [shift={(58.25,172.75)}, rotate = 248.08] [fill={rgb, 255:red, 0; green, 0; blue, 0 }  ][line width=0.08]  [draw opacity=0] (7.2,-1.8) -- (0,0) -- (7.2,1.8) -- cycle    ;
%Straight Lines [id:da43712383842750646] 
\draw    (30.75,137.25) -- (58.51,206.89) ;
\draw [shift={(59.25,208.75)}, rotate = 248.27] [fill={rgb, 255:red, 0; green, 0; blue, 0 }  ][line width=0.08]  [draw opacity=0] (7.2,-1.8) -- (0,0) -- (7.2,1.8) -- cycle    ;
%Curve Lines [id:da6823635267305639] 
\draw    (284.3,389.2) .. controls (184.8,373.48) and (-20.19,264.84) .. (55.59,162.79) ;
\draw [shift={(56.75,161.25)}, rotate = 125.54] [fill={rgb, 255:red, 0; green, 0; blue, 0 }  ][line width=0.08]  [draw opacity=0] (7.2,-1.8) -- (0,0) -- (7.2,1.8) -- cycle    ;
%Shape: Circle [id:dp09915510148134388] 
\draw  [color={rgb, 255:red, 74; green, 144; blue, 226 }  ,draw opacity=1 ] (177,243.5) .. controls (177,238.25) and (181.25,234) .. (186.5,234) .. controls (191.75,234) and (196,238.25) .. (196,243.5) .. controls (196,248.75) and (191.75,253) .. (186.5,253) .. controls (181.25,253) and (177,248.75) .. (177,243.5) -- cycle ;
%Shape: Circle [id:dp2106611699536416] 
\draw  [color={rgb, 255:red, 74; green, 144; blue, 226 }  ,draw opacity=1 ] (177,266.8) .. controls (177,261.55) and (181.25,257.3) .. (186.5,257.3) .. controls (191.75,257.3) and (196,261.55) .. (196,266.8) .. controls (196,272.05) and (191.75,276.3) .. (186.5,276.3) .. controls (181.25,276.3) and (177,272.05) .. (177,266.8) -- cycle ;
%Shape: Circle [id:dp25074856576749416] 
\draw  [color={rgb, 255:red, 74; green, 144; blue, 226 }  ,draw opacity=1 ] (177,303.8) .. controls (177,298.55) and (181.25,294.3) .. (186.5,294.3) .. controls (191.75,294.3) and (196,298.55) .. (196,303.8) .. controls (196,309.05) and (191.75,313.3) .. (186.5,313.3) .. controls (181.25,313.3) and (177,309.05) .. (177,303.8) -- cycle ;
%Straight Lines [id:da748472758265778] 
\draw    (119.6,159.6) -- (178.42,235.35) ;
\draw [shift={(179.64,236.93)}, rotate = 232.17] [fill={rgb, 255:red, 0; green, 0; blue, 0 }  ][line width=0.08]  [draw opacity=0] (7.2,-1.8) -- (0,0) -- (7.2,1.8) -- cycle    ;
%Straight Lines [id:da3965030230559381] 
\draw    (117.8,162.6) -- (178.29,258.1) ;
\draw [shift={(179.36,259.79)}, rotate = 237.65] [fill={rgb, 255:red, 0; green, 0; blue, 0 }  ][line width=0.08]  [draw opacity=0] (7.2,-1.8) -- (0,0) -- (7.2,1.8) -- cycle    ;
%Straight Lines [id:da8255076351126915] 
\draw    (115.6,164.8) -- (181.86,292.58) ;
\draw [shift={(182.79,294.36)}, rotate = 242.59] [fill={rgb, 255:red, 0; green, 0; blue, 0 }  ][line width=0.08]  [draw opacity=0] (7.2,-1.8) -- (0,0) -- (7.2,1.8) -- cycle    ;
%Straight Lines [id:da31689046304044277] 
\draw    (119.2,183.8) -- (175.92,238.68) ;
\draw [shift={(177.36,240.07)}, rotate = 224.06] [fill={rgb, 255:red, 0; green, 0; blue, 0 }  ][line width=0.08]  [draw opacity=0] (7.2,-1.8) -- (0,0) -- (7.2,1.8) -- cycle    ;
%Straight Lines [id:da8323923675685014] 
\draw    (117.2,186.6) -- (175.93,261.92) ;
\draw [shift={(177.16,263.5)}, rotate = 232.06] [fill={rgb, 255:red, 0; green, 0; blue, 0 }  ][line width=0.08]  [draw opacity=0] (7.2,-1.8) -- (0,0) -- (7.2,1.8) -- cycle    ;
%Straight Lines [id:da2609478512157011] 
\draw    (114.6,188.6) -- (178.33,295.5) ;
\draw [shift={(179.36,297.21)}, rotate = 239.2] [fill={rgb, 255:red, 0; green, 0; blue, 0 }  ][line width=0.08]  [draw opacity=0] (7.2,-1.8) -- (0,0) -- (7.2,1.8) -- cycle    ;
%Straight Lines [id:da21284023528449292] 
\draw    (119.6,220.4) -- (175.14,242.75) ;
\draw [shift={(177,243.5)}, rotate = 201.92] [fill={rgb, 255:red, 0; green, 0; blue, 0 }  ][line width=0.08]  [draw opacity=0] (7.2,-1.8) -- (0,0) -- (7.2,1.8) -- cycle    ;
%Straight Lines [id:da5547102757175062] 
\draw    (117.6,223) -- (175.39,265.61) ;
\draw [shift={(177,266.8)}, rotate = 216.4] [fill={rgb, 255:red, 0; green, 0; blue, 0 }  ][line width=0.08]  [draw opacity=0] (7.2,-1.8) -- (0,0) -- (7.2,1.8) -- cycle    ;
%Straight Lines [id:da1210776062190595] 
\draw    (115,225.6) -- (176.07,298.54) ;
\draw [shift={(177.36,300.07)}, rotate = 230.06] [fill={rgb, 255:red, 0; green, 0; blue, 0 }  ][line width=0.08]  [draw opacity=0] (7.2,-1.8) -- (0,0) -- (7.2,1.8) -- cycle    ;
%Shape: Circle [id:dp3111818739060366] 
\draw  [color={rgb, 255:red, 126; green, 211; blue, 33 }  ,draw opacity=1 ] (282.6,383.6) .. controls (282.6,378.35) and (286.85,374.1) .. (292.1,374.1) .. controls (297.35,374.1) and (301.6,378.35) .. (301.6,383.6) .. controls (301.6,388.85) and (297.35,393.1) .. (292.1,393.1) .. controls (286.85,393.1) and (282.6,388.85) .. (282.6,383.6) -- cycle ;
%Straight Lines [id:da7422309951426038] 
\draw    (120.17,156.5) -- (295.17,156.5) ;
\draw [shift={(297.17,156.5)}, rotate = 180] [fill={rgb, 255:red, 0; green, 0; blue, 0 }  ][line width=0.08]  [draw opacity=0] (7.2,-1.8) -- (0,0) -- (7.2,1.8) -- cycle    ;
%Straight Lines [id:da3568304400266695] 
\draw [color={rgb, 255:red, 155; green, 155; blue, 155 }  ,draw opacity=1 ] [dash pattern={on 0.84pt off 2.51pt}]  (10.97,317.6) -- (591,317.6) ;
%Straight Lines [id:da311090791695219] 
\draw    (120.13,179.83) -- (295.13,179.83) ;
\draw [shift={(297.13,179.83)}, rotate = 180] [fill={rgb, 255:red, 0; green, 0; blue, 0 }  ][line width=0.08]  [draw opacity=0] (7.2,-1.8) -- (0,0) -- (7.2,1.8) -- cycle    ;
%Straight Lines [id:da9404033824837681] 
\draw    (120.13,216.83) -- (295.13,216.83) ;
\draw [shift={(297.13,216.83)}, rotate = 180] [fill={rgb, 255:red, 0; green, 0; blue, 0 }  ][line width=0.08]  [draw opacity=0] (7.2,-1.8) -- (0,0) -- (7.2,1.8) -- cycle    ;
%Curve Lines [id:da2362736060765973] 
\draw    (16,73.67) .. controls (-11.44,185.83) and (-10.78,291.5) .. (176.4,246.47) ;
\draw [shift={(176.4,246.47)}, rotate = 167.1] [fill={rgb, 255:red, 0; green, 0; blue, 0 }  ][line width=0.08]  [draw opacity=0] (7.2,-1.8) -- (0,0) -- (7.2,1.8) -- cycle    ;
%Curve Lines [id:da08786040586631416] 
\draw    (17.38,99.63) .. controls (5.56,153.17) and (-16.44,275.17) .. (176.88,268.13) ;
\draw [shift={(176.88,268.13)}, rotate = 179.39] [fill={rgb, 255:red, 0; green, 0; blue, 0 }  ][line width=0.08]  [draw opacity=0] (7.2,-1.8) -- (0,0) -- (7.2,1.8) -- cycle    ;
%Curve Lines [id:da6253702613941365] 
\draw    (18.83,138) .. controls (9.17,209.52) and (61.35,267.05) .. (174.86,301.9) ;
\draw [shift={(176.57,302.43)}, rotate = 197.46] [fill={rgb, 255:red, 0; green, 0; blue, 0 }  ][line width=0.08]  [draw opacity=0] (7.2,-1.8) -- (0,0) -- (7.2,1.8) -- cycle    ;
%Straight Lines [id:da49083115372236863] 
\draw [color={rgb, 255:red, 155; green, 155; blue, 155 }  ,draw opacity=1 ] [dash pattern={on 0.84pt off 2.51pt}]  (11,368.6) -- (588.6,368.6) ;
%Shape: Circle [id:dp22945679605960345] 
\draw  [color={rgb, 255:red, 208; green, 2; blue, 27 }  ,draw opacity=1 ] (297.33,157.5) .. controls (297.33,152.25) and (301.59,148) .. (306.83,148) .. controls (312.08,148) and (316.33,152.25) .. (316.33,157.5) .. controls (316.33,162.75) and (312.08,167) .. (306.83,167) .. controls (301.59,167) and (297.33,162.75) .. (297.33,157.5) -- cycle ;
%Shape: Circle [id:dp08289021338463831] 
\draw  [color={rgb, 255:red, 208; green, 2; blue, 27 }  ,draw opacity=1 ] (342.67,157.5) .. controls (342.67,152.25) and (346.92,148) .. (352.17,148) .. controls (357.41,148) and (361.67,152.25) .. (361.67,157.5) .. controls (361.67,162.75) and (357.41,167) .. (352.17,167) .. controls (346.92,167) and (342.67,162.75) .. (342.67,157.5) -- cycle ;
%Straight Lines [id:da05165447422183944] 
\draw    (316.33,157.5) -- (340.67,157.5) ;
\draw [shift={(342.67,157.5)}, rotate = 180] [fill={rgb, 255:red, 0; green, 0; blue, 0 }  ][line width=0.08]  [draw opacity=0] (7.2,-1.8) -- (0,0) -- (7.2,1.8) -- cycle    ;
%Shape: Circle [id:dp2410606500988659] 
\draw  [color={rgb, 255:red, 208; green, 2; blue, 27 }  ,draw opacity=1 ] (297.3,180.83) .. controls (297.3,175.59) and (301.55,171.33) .. (306.8,171.33) .. controls (312.05,171.33) and (316.3,175.59) .. (316.3,180.83) .. controls (316.3,186.08) and (312.05,190.33) .. (306.8,190.33) .. controls (301.55,190.33) and (297.3,186.08) .. (297.3,180.83) -- cycle ;
%Shape: Circle [id:dp5486297370091895] 
\draw  [color={rgb, 255:red, 208; green, 2; blue, 27 }  ,draw opacity=1 ] (342.63,180.83) .. controls (342.63,175.59) and (346.89,171.33) .. (352.13,171.33) .. controls (357.38,171.33) and (361.63,175.59) .. (361.63,180.83) .. controls (361.63,186.08) and (357.38,190.33) .. (352.13,190.33) .. controls (346.89,190.33) and (342.63,186.08) .. (342.63,180.83) -- cycle ;
%Straight Lines [id:da0806724603215978] 
\draw    (316.3,180.83) -- (340.63,180.83) ;
\draw [shift={(342.63,180.83)}, rotate = 180] [fill={rgb, 255:red, 0; green, 0; blue, 0 }  ][line width=0.08]  [draw opacity=0] (7.2,-1.8) -- (0,0) -- (7.2,1.8) -- cycle    ;

%Shape: Circle [id:dp43039113781966143] 
\draw  [color={rgb, 255:red, 208; green, 2; blue, 27 }  ,draw opacity=1 ] (297.3,217.83) .. controls (297.3,212.59) and (301.55,208.33) .. (306.8,208.33) .. controls (312.05,208.33) and (316.3,212.59) .. (316.3,217.83) .. controls (316.3,223.08) and (312.05,227.33) .. (306.8,227.33) .. controls (301.55,227.33) and (297.3,223.08) .. (297.3,217.83) -- cycle ;
%Shape: Circle [id:dp5229014287691234] 
\draw  [color={rgb, 255:red, 208; green, 2; blue, 27 }  ,draw opacity=1 ] (342.63,217.83) .. controls (342.63,212.59) and (346.89,208.33) .. (352.13,208.33) .. controls (357.38,208.33) and (361.63,212.59) .. (361.63,217.83) .. controls (361.63,223.08) and (357.38,227.33) .. (352.13,227.33) .. controls (346.89,227.33) and (342.63,223.08) .. (342.63,217.83) -- cycle ;
%Straight Lines [id:da131633971935041] 
\draw    (316.3,217.83) -- (340.63,217.83) ;
\draw [shift={(342.63,217.83)}, rotate = 180] [fill={rgb, 255:red, 0; green, 0; blue, 0 }  ][line width=0.08]  [draw opacity=0] (7.2,-1.8) -- (0,0) -- (7.2,1.8) -- cycle    ;


%Shape: Circle [id:dp1179799709309528] 
\draw  [color={rgb, 255:red, 126; green, 211; blue, 33 }  ,draw opacity=1 ] (256.5,70.5) .. controls (256.5,65.25) and (260.75,61) .. (266,61) .. controls (271.25,61) and (275.5,65.25) .. (275.5,70.5) .. controls (275.5,75.75) and (271.25,80) .. (266,80) .. controls (260.75,80) and (256.5,75.75) .. (256.5,70.5) -- cycle ;
%Shape: Circle [id:dp4916621755630901] 
\draw  [color={rgb, 255:red, 126; green, 211; blue, 33 }  ,draw opacity=1 ] (256.5,93.8) .. controls (256.5,88.55) and (260.75,84.3) .. (266,84.3) .. controls (271.25,84.3) and (275.5,88.55) .. (275.5,93.8) .. controls (275.5,99.05) and (271.25,103.3) .. (266,103.3) .. controls (260.75,103.3) and (256.5,99.05) .. (256.5,93.8) -- cycle ;
%Shape: Circle [id:dp9076953584753158] 
\draw  [color={rgb, 255:red, 126; green, 211; blue, 33 }  ,draw opacity=1 ] (256.5,130.8) .. controls (256.5,125.55) and (260.75,121.3) .. (266,121.3) .. controls (271.25,121.3) and (275.5,125.55) .. (275.5,130.8) .. controls (275.5,136.05) and (271.25,140.3) .. (266,140.3) .. controls (260.75,140.3) and (256.5,136.05) .. (256.5,130.8) -- cycle ;
%Straight Lines [id:da9118419275275298] 
\draw    (272.75,77.75) -- (302.46,146.41) ;
\draw [shift={(303.25,148.25)}, rotate = 246.61] [fill={rgb, 255:red, 0; green, 0; blue, 0 }  ][line width=0.08]  [draw opacity=0] (7.2,-1.8) -- (0,0) -- (7.2,1.8) -- cycle    ;
%Straight Lines [id:da9676924899876029] 
\draw    (271,102.3) -- (299,171.89) ;
\draw [shift={(299.75,173.75)}, rotate = 248.08] [fill={rgb, 255:red, 0; green, 0; blue, 0 }  ][line width=0.08]  [draw opacity=0] (7.2,-1.8) -- (0,0) -- (7.2,1.8) -- cycle    ;
%Straight Lines [id:da8196893702332761] 
\draw    (272.25,138.25) -- (300.01,207.89) ;
\draw [shift={(300.75,209.75)}, rotate = 248.27] [fill={rgb, 255:red, 0; green, 0; blue, 0 }  ][line width=0.08]  [draw opacity=0] (7.2,-1.8) -- (0,0) -- (7.2,1.8) -- cycle    ;
%Shape: Circle [id:dp5522106490973764] 
\draw  [color={rgb, 255:red, 74; green, 144; blue, 226 }  ,draw opacity=1 ] (418.5,244.5) .. controls (418.5,239.25) and (422.75,235) .. (428,235) .. controls (433.25,235) and (437.5,239.25) .. (437.5,244.5) .. controls (437.5,249.75) and (433.25,254) .. (428,254) .. controls (422.75,254) and (418.5,249.75) .. (418.5,244.5) -- cycle ;
%Shape: Circle [id:dp9342115761880321] 
\draw  [color={rgb, 255:red, 74; green, 144; blue, 226 }  ,draw opacity=1 ] (418.5,267.8) .. controls (418.5,262.55) and (422.75,258.3) .. (428,258.3) .. controls (433.25,258.3) and (437.5,262.55) .. (437.5,267.8) .. controls (437.5,273.05) and (433.25,277.3) .. (428,277.3) .. controls (422.75,277.3) and (418.5,273.05) .. (418.5,267.8) -- cycle ;
%Shape: Circle [id:dp8459942019088817] 
\draw  [color={rgb, 255:red, 74; green, 144; blue, 226 }  ,draw opacity=1 ] (418.5,304.8) .. controls (418.5,299.55) and (422.75,295.3) .. (428,295.3) .. controls (433.25,295.3) and (437.5,299.55) .. (437.5,304.8) .. controls (437.5,310.05) and (433.25,314.3) .. (428,314.3) .. controls (422.75,314.3) and (418.5,310.05) .. (418.5,304.8) -- cycle ;
%Straight Lines [id:da09710004151743323] 
\draw    (361.1,160.6) -- (419.92,236.35) ;
\draw [shift={(421.14,237.93)}, rotate = 232.17] [fill={rgb, 255:red, 0; green, 0; blue, 0 }  ][line width=0.08]  [draw opacity=0] (7.2,-1.8) -- (0,0) -- (7.2,1.8) -- cycle    ;
%Straight Lines [id:da34635586705229726] 
\draw    (359.3,163.6) -- (419.79,259.1) ;
\draw [shift={(420.86,260.79)}, rotate = 237.65] [fill={rgb, 255:red, 0; green, 0; blue, 0 }  ][line width=0.08]  [draw opacity=0] (7.2,-1.8) -- (0,0) -- (7.2,1.8) -- cycle    ;
%Straight Lines [id:da4349576454406552] 
\draw    (357.1,165.8) -- (423.36,293.58) ;
\draw [shift={(424.29,295.36)}, rotate = 242.59] [fill={rgb, 255:red, 0; green, 0; blue, 0 }  ][line width=0.08]  [draw opacity=0] (7.2,-1.8) -- (0,0) -- (7.2,1.8) -- cycle    ;
%Straight Lines [id:da4563331916615285] 
\draw    (360.7,184.8) -- (417.42,239.68) ;
\draw [shift={(418.86,241.07)}, rotate = 224.06] [fill={rgb, 255:red, 0; green, 0; blue, 0 }  ][line width=0.08]  [draw opacity=0] (7.2,-1.8) -- (0,0) -- (7.2,1.8) -- cycle    ;
%Straight Lines [id:da061074266286135215] 
\draw    (358.7,187.6) -- (417.43,262.92) ;
\draw [shift={(418.66,264.5)}, rotate = 232.06] [fill={rgb, 255:red, 0; green, 0; blue, 0 }  ][line width=0.08]  [draw opacity=0] (7.2,-1.8) -- (0,0) -- (7.2,1.8) -- cycle    ;
%Straight Lines [id:da8260762281885707] 
\draw    (356.1,189.6) -- (419.83,296.5) ;
\draw [shift={(420.86,298.21)}, rotate = 239.2] [fill={rgb, 255:red, 0; green, 0; blue, 0 }  ][line width=0.08]  [draw opacity=0] (7.2,-1.8) -- (0,0) -- (7.2,1.8) -- cycle    ;
%Straight Lines [id:da4076931125124921] 
\draw    (361.1,221.4) -- (416.64,243.75) ;
\draw [shift={(418.5,244.5)}, rotate = 201.92] [fill={rgb, 255:red, 0; green, 0; blue, 0 }  ][line width=0.08]  [draw opacity=0] (7.2,-1.8) -- (0,0) -- (7.2,1.8) -- cycle    ;
%Straight Lines [id:da272324322985684] 
\draw    (359.1,224) -- (416.89,266.61) ;
\draw [shift={(418.5,267.8)}, rotate = 216.4] [fill={rgb, 255:red, 0; green, 0; blue, 0 }  ][line width=0.08]  [draw opacity=0] (7.2,-1.8) -- (0,0) -- (7.2,1.8) -- cycle    ;
%Straight Lines [id:da5942246569149254] 
\draw    (356.5,226.6) -- (417.57,299.54) ;
\draw [shift={(418.86,301.07)}, rotate = 230.06] [fill={rgb, 255:red, 0; green, 0; blue, 0 }  ][line width=0.08]  [draw opacity=0] (7.2,-1.8) -- (0,0) -- (7.2,1.8) -- cycle    ;
%Curve Lines [id:da4226750962481063] 
\draw    (257.5,75) .. controls (228.25,206.5) and (243.22,284.83) .. (417.9,247.8) ;
\draw [shift={(417.9,247.8)}, rotate = 168.55] [fill={rgb, 255:red, 0; green, 0; blue, 0 }  ][line width=0.08]  [draw opacity=0] (7.2,-1.8) -- (0,0) -- (7.2,1.8) -- cycle    ;
%Curve Lines [id:da7657131181018171] 
\draw    (258.88,100.63) .. controls (248.75,169.5) and (225.97,271.13) .. (418.38,269.13) ;
\draw [shift={(418.38,269.13)}, rotate = 180.33] [fill={rgb, 255:red, 0; green, 0; blue, 0 }  ][line width=0.08]  [draw opacity=0] (7.2,-1.8) -- (0,0) -- (7.2,1.8) -- cycle    ;
%Curve Lines [id:da7219861824522342] 
\draw    (261.17,139.5) .. controls (249.63,208.75) and (283.38,255.25) .. (418.07,303.43) ;
\draw [shift={(418.07,303.43)}, rotate = 200.08] [fill={rgb, 255:red, 0; green, 0; blue, 0 }  ][line width=0.08]  [draw opacity=0] (7.2,-1.8) -- (0,0) -- (7.2,1.8) -- cycle    ;
%Straight Lines [id:da6954685307859836] 
\draw    (196,243.5) -- (220.33,243.5) ;
\draw [shift={(222.33,243.5)}, rotate = 180] [fill={rgb, 255:red, 0; green, 0; blue, 0 }  ][line width=0.08]  [draw opacity=0] (7.2,-1.8) -- (0,0) -- (7.2,1.8) -- cycle    ;
%Shape: Circle [id:dp06599842740642958] 
\draw  [color={rgb, 255:red, 74; green, 144; blue, 226 }  ,draw opacity=1 ] (222.33,243.5) .. controls (222.33,238.25) and (226.59,234) .. (231.83,234) .. controls (237.08,234) and (241.33,238.25) .. (241.33,243.5) .. controls (241.33,248.75) and (237.08,253) .. (231.83,253) .. controls (226.59,253) and (222.33,248.75) .. (222.33,243.5) -- cycle ;
%Straight Lines [id:da8831164181259743] 
\draw    (196,266.8) -- (220.33,266.8) ;
\draw [shift={(222.33,266.8)}, rotate = 180] [fill={rgb, 255:red, 0; green, 0; blue, 0 }  ][line width=0.08]  [draw opacity=0] (7.2,-1.8) -- (0,0) -- (7.2,1.8) -- cycle    ;
%Shape: Circle [id:dp6632665263641222] 
\draw  [color={rgb, 255:red, 74; green, 144; blue, 226 }  ,draw opacity=1 ] (222.33,266.8) .. controls (222.33,261.55) and (226.59,257.3) .. (231.83,257.3) .. controls (237.08,257.3) and (241.33,261.55) .. (241.33,266.8) .. controls (241.33,272.05) and (237.08,276.3) .. (231.83,276.3) .. controls (226.59,276.3) and (222.33,272.05) .. (222.33,266.8) -- cycle ;
%Straight Lines [id:da1717231570739859] 
\draw    (196,303.8) -- (220.33,303.8) ;
\draw [shift={(222.33,303.8)}, rotate = 180] [fill={rgb, 255:red, 0; green, 0; blue, 0 }  ][line width=0.08]  [draw opacity=0] (7.2,-1.8) -- (0,0) -- (7.2,1.8) -- cycle    ;
%Shape: Circle [id:dp36553179592098783] 
\draw  [color={rgb, 255:red, 74; green, 144; blue, 226 }  ,draw opacity=1 ] (222.33,303.8) .. controls (222.33,298.55) and (226.59,294.3) .. (231.83,294.3) .. controls (237.08,294.3) and (241.33,298.55) .. (241.33,303.8) .. controls (241.33,309.05) and (237.08,313.3) .. (231.83,313.3) .. controls (226.59,313.3) and (222.33,309.05) .. (222.33,303.8) -- cycle ;
%Straight Lines [id:da9597099251108501] 
\draw    (437.5,244.5) -- (461.83,244.5) ;
\draw [shift={(463.83,244.5)}, rotate = 180] [fill={rgb, 255:red, 0; green, 0; blue, 0 }  ][line width=0.08]  [draw opacity=0] (7.2,-1.8) -- (0,0) -- (7.2,1.8) -- cycle    ;
%Straight Lines [id:da7937919892966077] 
\draw    (437.5,267.8) -- (461.83,267.8) ;
\draw [shift={(463.83,267.8)}, rotate = 180] [fill={rgb, 255:red, 0; green, 0; blue, 0 }  ][line width=0.08]  [draw opacity=0] (7.2,-1.8) -- (0,0) -- (7.2,1.8) -- cycle    ;
%Straight Lines [id:da2541460536893574] 
\draw    (437.5,304.8) -- (461.83,304.8) ;
\draw [shift={(463.83,304.8)}, rotate = 180] [fill={rgb, 255:red, 0; green, 0; blue, 0 }  ][line width=0.08]  [draw opacity=0] (7.2,-1.8) -- (0,0) -- (7.2,1.8) -- cycle    ;
%Shape: Circle [id:dp17368365059269686] 
\draw  [color={rgb, 255:red, 74; green, 144; blue, 226 }  ,draw opacity=1 ] (463.83,244.5) .. controls (463.83,239.25) and (468.09,235) .. (473.33,235) .. controls (478.58,235) and (482.83,239.25) .. (482.83,244.5) .. controls (482.83,249.75) and (478.58,254) .. (473.33,254) .. controls (468.09,254) and (463.83,249.75) .. (463.83,244.5) -- cycle ;
%Shape: Circle [id:dp2672571368525132] 
\draw  [color={rgb, 255:red, 74; green, 144; blue, 226 }  ,draw opacity=1 ] (463.83,267.8) .. controls (463.83,262.55) and (468.09,258.3) .. (473.33,258.3) .. controls (478.58,258.3) and (482.83,262.55) .. (482.83,267.8) .. controls (482.83,273.05) and (478.58,277.3) .. (473.33,277.3) .. controls (468.09,277.3) and (463.83,273.05) .. (463.83,267.8) -- cycle ;
%Shape: Circle [id:dp708525032532848] 
\draw  [color={rgb, 255:red, 74; green, 144; blue, 226 }  ,draw opacity=1 ] (463.83,304.8) .. controls (463.83,299.55) and (468.09,295.3) .. (473.33,295.3) .. controls (478.58,295.3) and (482.83,299.55) .. (482.83,304.8) .. controls (482.83,310.05) and (478.58,314.3) .. (473.33,314.3) .. controls (468.09,314.3) and (463.83,310.05) .. (463.83,304.8) -- cycle ;
%Shape: Circle [id:dp7255190009372188] 
\draw  [color={rgb, 255:red, 245; green, 166; blue, 35 }  ,draw opacity=1 ] (326.47,347.5) .. controls (326.47,342.25) and (330.72,338) .. (335.97,338) .. controls (341.21,338) and (345.47,342.25) .. (345.47,347.5) .. controls (345.47,352.75) and (341.21,357) .. (335.97,357) .. controls (330.72,357) and (326.47,352.75) .. (326.47,347.5) -- cycle ;
%Straight Lines [id:da23800986407359082] 
\draw [color={rgb, 255:red, 155; green, 155; blue, 155 }  ,draw opacity=1 ] [dash pattern={on 0.84pt off 2.51pt}]  (487.33,0.75) -- (487.33,314) ;
%Straight Lines [id:da5420007319356506] 
\draw [color={rgb, 255:red, 155; green, 155; blue, 155 }  ,draw opacity=1 ] [dash pattern={on 0.84pt off 2.51pt}]  (538.67,2.25) -- (538.67,314) ;
%Straight Lines [id:da638120539610346] 
\draw    (361.67,157.25) -- (499.8,157.25) ;
\draw [shift={(501.8,157.25)}, rotate = 180] [fill={rgb, 255:red, 0; green, 0; blue, 0 }  ][line width=0.08]  [draw opacity=0] (7.2,-1.8) -- (0,0) -- (7.2,1.8) -- cycle    ;
%Shape: Circle [id:dp6824735541696747] 
\draw  [color={rgb, 255:red, 245; green, 166; blue, 35 }  ,draw opacity=1 ] (562.47,183.83) .. controls (562.47,178.59) and (566.72,174.33) .. (571.97,174.33) .. controls (577.21,174.33) and (581.47,178.59) .. (581.47,183.83) .. controls (581.47,189.08) and (577.21,193.33) .. (571.97,193.33) .. controls (566.72,193.33) and (562.47,189.08) .. (562.47,183.83) -- cycle ;
%Shape: Circle [id:dp5156095412004806] 
\draw  [color={rgb, 255:red, 245; green, 166; blue, 35 }  ,draw opacity=1 ] (465.2,347.5) .. controls (465.2,342.25) and (469.45,338) .. (474.7,338) .. controls (479.95,338) and (484.2,342.25) .. (484.2,347.5) .. controls (484.2,352.75) and (479.95,357) .. (474.7,357) .. controls (469.45,357) and (465.2,352.75) .. (465.2,347.5) -- cycle ;
%Shape: Circle [id:dp6485856767533946] 
\draw  [color={rgb, 255:red, 245; green, 166; blue, 35 }  ,draw opacity=1 ] (515.73,347.5) .. controls (515.73,342.25) and (519.99,338) .. (525.23,338) .. controls (530.48,338) and (534.73,342.25) .. (534.73,347.5) .. controls (534.73,352.75) and (530.48,357) .. (525.23,357) .. controls (519.99,357) and (515.73,352.75) .. (515.73,347.5) -- cycle ;
%Curve Lines [id:da9247431008395106] 
\draw    (240,248.4) .. controls (272.34,258) and (307.69,283.29) .. (331.48,337.16) ;
\draw [shift={(332.2,338.8)}, rotate = 245.29] [fill={rgb, 255:red, 0; green, 0; blue, 0 }  ][line width=0.08]  [draw opacity=0] (7.2,-1.8) -- (0,0) -- (7.2,1.8) -- cycle    ;
%Curve Lines [id:da4147748923248351] 
\draw    (239.6,272.9) .. controls (253.86,274.48) and (303.3,287.67) .. (328.83,339.03) ;
\draw [shift={(329.6,340.6)}, rotate = 239.32] [fill={rgb, 255:red, 0; green, 0; blue, 0 }  ][line width=0.08]  [draw opacity=0] (7.2,-1.8) -- (0,0) -- (7.2,1.8) -- cycle    ;
%Curve Lines [id:da7153073087082051] 
\draw    (241.33,303.8) .. controls (251.89,303.34) and (297.28,295.95) .. (326.71,341.02) ;
\draw [shift={(327.6,342.4)}, rotate = 230.36] [fill={rgb, 255:red, 0; green, 0; blue, 0 }  ][line width=0.08]  [draw opacity=0] (7.2,-1.8) -- (0,0) -- (7.2,1.8) -- cycle    ;
%Curve Lines [id:da7753237943474862] 
\draw    (481.8,249.8) .. controls (506.4,318.4) and (491.4,353.4) .. (345.4,350.8) ;
\draw [shift={(345.4,350.8)}, rotate = 0.59] [fill={rgb, 255:red, 0; green, 0; blue, 0 }  ][line width=0.08]  [draw opacity=0] (7.2,-1.8) -- (0,0) -- (7.2,1.8) -- cycle    ;
%Curve Lines [id:da24332882355411245] 
\draw    (480.2,275) .. controls (505.41,326.04) and (450.73,345.31) .. (347.03,347.47) ;
\draw [shift={(345.47,347.5)}, rotate = 358.5] [fill={rgb, 255:red, 0; green, 0; blue, 0 }  ][line width=0.08]  [draw opacity=0] (7.2,-1.8) -- (0,0) -- (7.2,1.8) -- cycle    ;
%Curve Lines [id:da5506933564220804] 
\draw    (469,313.4) .. controls (456.07,327.13) and (431.84,339.67) .. (346.29,344.53) ;
\draw [shift={(345,344.6)}, rotate = 355.96] [fill={rgb, 255:red, 0; green, 0; blue, 0 }  ][line width=0.08]  [draw opacity=0] (7.2,-1.8) -- (0,0) -- (7.2,1.8) -- cycle    ;
%Curve Lines [id:da7099110881272985] 
\draw    (482.83,244.5) .. controls (508.8,286.17) and (501.78,309.14) .. (483.08,338.81) ;
\draw [shift={(482.22,340.17)}, rotate = 302] [fill={rgb, 255:red, 0; green, 0; blue, 0 }  ][line width=0.08]  [draw opacity=0] (7.2,-1.8) -- (0,0) -- (7.2,1.8) -- cycle    ;
%Curve Lines [id:da02238402771595016] 
\draw    (482.83,267.8) .. controls (501.32,284.94) and (497.02,308.45) .. (478.73,336.55) ;
\draw [shift={(477.89,337.83)}, rotate = 302.24] [fill={rgb, 255:red, 0; green, 0; blue, 0 }  ][line width=0.08]  [draw opacity=0] (7.2,-1.8) -- (0,0) -- (7.2,1.8) -- cycle    ;
%Straight Lines [id:da31851708212883434] 
\draw    (361.63,180.83) -- (499.4,180.83) ;
\draw [shift={(501.4,180.83)}, rotate = 180] [fill={rgb, 255:red, 0; green, 0; blue, 0 }  ][line width=0.08]  [draw opacity=0] (7.2,-1.8) -- (0,0) -- (7.2,1.8) -- cycle    ;
%Straight Lines [id:da09276398608555847] 
\draw    (361.63,217.83) -- (499,217.83) ;
\draw [shift={(501,217.83)}, rotate = 180] [fill={rgb, 255:red, 0; green, 0; blue, 0 }  ][line width=0.08]  [draw opacity=0] (7.2,-1.8) -- (0,0) -- (7.2,1.8) -- cycle    ;
%Curve Lines [id:da4867248015097767] 
\draw    (283.3,387.2) .. controls (185.13,363.32) and (16.94,265.49) .. (56.17,187.01) ;
\draw [shift={(56.78,185.83)}, rotate = 115.28] [fill={rgb, 255:red, 0; green, 0; blue, 0 }  ][line width=0.08]  [draw opacity=0] (7.2,-1.8) -- (0,0) -- (7.2,1.8) -- cycle    ;
%Curve Lines [id:da6225287245391646] 
\draw    (282.5,385) .. controls (196.37,359.06) and (62.83,273.36) .. (65.19,227.7) ;
\draw [shift={(65.3,226.33)}, rotate = 91.23] [fill={rgb, 255:red, 0; green, 0; blue, 0 }  ][line width=0.08]  [draw opacity=0] (7.2,-1.8) -- (0,0) -- (7.2,1.8) -- cycle    ;
%Curve Lines [id:da14670118842375524] 
\draw    (282.57,381.14) .. controls (218.29,332.93) and (97.5,301.74) .. (177.38,250.38) ;
\draw [shift={(178.6,249.6)}, rotate = 146.11] [fill={rgb, 255:red, 0; green, 0; blue, 0 }  ][line width=0.08]  [draw opacity=0] (7.2,-1.8) -- (0,0) -- (7.2,1.8) -- cycle    ;
%Curve Lines [id:da05931190530278174] 
\draw    (283.71,378.57) .. controls (221.1,326.6) and (124.53,301.36) .. (175.8,272.09) ;
\draw [shift={(177.4,271.2)}, rotate = 149.58] [fill={rgb, 255:red, 0; green, 0; blue, 0 }  ][line width=0.08]  [draw opacity=0] (7.2,-1.8) -- (0,0) -- (7.2,1.8) -- cycle    ;
%Curve Lines [id:da6126326050141437] 
\draw    (285.43,376.86) .. controls (253.91,343.76) and (221.34,332.54) .. (193.89,311.77) ;
\draw [shift={(192.64,310.82)}, rotate = 36.75] [fill={rgb, 255:red, 0; green, 0; blue, 0 }  ][line width=0.08]  [draw opacity=0] (7.2,-1.8) -- (0,0) -- (7.2,1.8) -- cycle    ;
%Curve Lines [id:da631546261097365] 
\draw    (289.71,374.86) .. controls (272.8,316.65) and (254.58,243.34) .. (297.19,162.47) ;
\draw [shift={(297.83,161.25)}, rotate = 116.83] [fill={rgb, 255:red, 0; green, 0; blue, 0 }  ][line width=0.08]  [draw opacity=0] (7.2,-1.8) -- (0,0) -- (7.2,1.8) -- cycle    ;
%Curve Lines [id:da6609630790590302] 
\draw    (292.1,374.1) .. controls (283.18,316.74) and (265.48,269.77) .. (297.84,186.02) ;
\draw [shift={(298.33,184.75)}, rotate = 110.48] [fill={rgb, 255:red, 0; green, 0; blue, 0 }  ][line width=0.08]  [draw opacity=0] (7.2,-1.8) -- (0,0) -- (7.2,1.8) -- cycle    ;
%Curve Lines [id:da6265302997948612] 
\draw    (294.5,374.08) .. controls (285.26,288.94) and (284.44,275.83) .. (298.57,225.67) ;
\draw [shift={(299,224.14)}, rotate = 105.68] [fill={rgb, 255:red, 0; green, 0; blue, 0 }  ][line width=0.08]  [draw opacity=0] (7.2,-1.8) -- (0,0) -- (7.2,1.8) -- cycle    ;
%Curve Lines [id:da038804301541383035] 
\draw    (300,378.43) .. controls (369.9,358.35) and (349.57,337.21) .. (417.87,250.92) ;
\draw [shift={(418.9,249.62)}, rotate = 128.06] [fill={rgb, 255:red, 0; green, 0; blue, 0 }  ][line width=0.08]  [draw opacity=0] (7.2,-1.8) -- (0,0) -- (7.2,1.8) -- cycle    ;
%Curve Lines [id:da8936189159998054] 
\draw    (301.29,381) .. controls (374.13,359.11) and (359.65,333.92) .. (418.3,273.34) ;
\draw [shift={(419.19,272.43)}, rotate = 133.69] [fill={rgb, 255:red, 0; green, 0; blue, 0 }  ][line width=0.08]  [draw opacity=0] (7.2,-1.8) -- (0,0) -- (7.2,1.8) -- cycle    ;
%Curve Lines [id:da3514522204329813] 
\draw    (301.6,383.6) .. controls (370.64,365.76) and (375.44,324.65) .. (416.83,306.45) ;
\draw [shift={(418.09,305.91)}, rotate = 155.81] [fill={rgb, 255:red, 0; green, 0; blue, 0 }  ][line width=0.08]  [draw opacity=0] (7.2,-1.8) -- (0,0) -- (7.2,1.8) -- cycle    ;
%Curve Lines [id:da10458715808840569] 
\draw [color={rgb, 255:red, 0; green, 0; blue, 0 }  ,draw opacity=1 ]   (535.2,309.6) .. controls (535.2,320.74) and (534.09,324.25) .. (530.49,338.35) ;
\draw [shift={(530.02,340.17)}, rotate = 284.41] [fill={rgb, 255:red, 0; green, 0; blue, 0 }  ,fill opacity=1 ][line width=0.08]  [draw opacity=0] (7.2,-1.8) -- (0,0) -- (7.2,1.8) -- cycle    ;
%Curve Lines [id:da493730763401113] 
\draw [color={rgb, 255:red, 0; green, 0; blue, 0 }  ,draw opacity=1 ]   (529.21,309.17) .. controls (529.95,316.68) and (527.21,326.93) .. (525.56,336.1) ;
\draw [shift={(525.23,338)}, rotate = 280.74] [fill={rgb, 255:red, 0; green, 0; blue, 0 }  ,fill opacity=1 ][line width=0.08]  [draw opacity=0] (7.2,-1.8) -- (0,0) -- (7.2,1.8) -- cycle    ;
%Curve Lines [id:da6471133794486161] 
\draw [color={rgb, 255:red, 0; green, 0; blue, 0 }  ,draw opacity=1 ]   (521.6,308.93) .. controls (523.14,321.16) and (522.8,320.93) .. (521,337.11) ;
\draw [shift={(520.8,338.93)}, rotate = 276.48] [fill={rgb, 255:red, 0; green, 0; blue, 0 }  ,fill opacity=1 ][line width=0.08]  [draw opacity=0] (7.2,-1.8) -- (0,0) -- (7.2,1.8) -- cycle    ;
%Curve Lines [id:da670832980921543] 
\draw    (536,157.73) .. controls (547.09,156.61) and (552.46,179.04) .. (560.75,185.27) ;
\draw [shift={(562.38,186.25)}, rotate = 225.65] [fill={rgb, 255:red, 0; green, 0; blue, 0 }  ][line width=0.08]  [draw opacity=0] (7.2,-1.8) -- (0,0) -- (7.2,1.8) -- cycle    ;
%Curve Lines [id:da02770294021289521] 
\draw    (536.55,180.36) .. controls (547.81,179.22) and (550.4,192.7) .. (562.8,191.08) ;
\draw [shift={(564.63,190.75)}, rotate = 183.71] [fill={rgb, 255:red, 0; green, 0; blue, 0 }  ][line width=0.08]  [draw opacity=0] (7.2,-1.8) -- (0,0) -- (7.2,1.8) -- cycle    ;
%Curve Lines [id:da4364925852621979] 
\draw    (536.36,218.18) .. controls (547.75,217.02) and (559.26,209.92) .. (567.61,194.33) ;
\draw [shift={(568.5,192.6)}, rotate = 121.91] [fill={rgb, 255:red, 0; green, 0; blue, 0 }  ][line width=0.08]  [draw opacity=0] (7.2,-1.8) -- (0,0) -- (7.2,1.8) -- cycle    ;
%Curve Lines [id:da13574990988185998] 
\draw    (299.57,389.14) .. controls (505.2,353.6) and (566,434.4) .. (571.97,193.33) ;
\draw [shift={(571.97,193.33)}, rotate = 91.42] [fill={rgb, 255:red, 0; green, 0; blue, 0 }  ][line width=0.08]  [draw opacity=0] (7.2,-1.8) -- (0,0) -- (7.2,1.8) -- cycle    ;
%Curve Lines [id:da8440950354449732] 
\draw    (33.43,66) .. controls (150.57,-6.57) and (536.87,-49.13) .. (578.2,176.2) ;
\draw [shift={(578.2,176.2)}, rotate = 257.65] [fill={rgb, 255:red, 0; green, 0; blue, 0 }  ][line width=0.08]  [draw opacity=0] (7.2,-1.8) -- (0,0) -- (7.2,1.8) -- cycle    ;
%Curve Lines [id:da3962909343156955] 
\draw    (30.86,85.43) .. controls (136.86,-0.29) and (513,-46.73) .. (575,174.6) ;
\draw [shift={(575,174.6)}, rotate = 252.47] [fill={rgb, 255:red, 0; green, 0; blue, 0 }  ][line width=0.08]  [draw opacity=0] (7.2,-1.8) -- (0,0) -- (7.2,1.8) -- cycle    ;
%Curve Lines [id:da5814632409959868] 
\draw    (30.29,122.57) .. controls (70.33,16.33) and (497,-53) .. (571.97,174.33) ;
\draw [shift={(571.97,174.33)}, rotate = 249.7] [fill={rgb, 255:red, 0; green, 0; blue, 0 }  ][line width=0.08]  [draw opacity=0] (7.2,-1.8) -- (0,0) -- (7.2,1.8) -- cycle    ;
%Curve Lines [id:da44873372603350736] 
\draw    (273.5,64) .. controls (417.19,-3.16) and (515.88,89.44) .. (564.64,175.82) ;
\draw [shift={(565.38,177.13)}, rotate = 240.32] [fill={rgb, 255:red, 0; green, 0; blue, 0 }  ][line width=0.08]  [draw opacity=0] (7.2,-1.8) -- (0,0) -- (7.2,1.8) -- cycle    ;
%Curve Lines [id:da8498920721940137] 
\draw    (275,89.25) .. controls (401.78,-12.24) and (506.78,88.15) .. (562.99,177.82) ;
\draw [shift={(563.83,179.17)}, rotate = 237.72] [fill={rgb, 255:red, 0; green, 0; blue, 0 }  ][line width=0.08]  [draw opacity=0] (7.2,-1.8) -- (0,0) -- (7.2,1.8) -- cycle    ;
%Curve Lines [id:da5382978880434821] 
\draw    (273,124) .. controls (371.51,-22.76) and (503.18,84.91) .. (561.75,180.68) ;
\draw [shift={(562.63,182.13)}, rotate = 238.26] [fill={rgb, 255:red, 0; green, 0; blue, 0 }  ][line width=0.08]  [draw opacity=0] (7.2,-1.8) -- (0,0) -- (7.2,1.8) -- cycle    ;

% Text Node
\draw (99,64.5) node [anchor=north west][inner sep=0.75pt]   [align=left] {Time slot $\displaystyle 1$};
% Text Node
\draw (345,63.5) node [anchor=north west][inner sep=0.75pt]   [align=left] {Time slot $\displaystyle 2$};
% Text Node
\draw (28,102) node [anchor=north west][inner sep=0.75pt]  [rotate=-90]  {$\cdots $};
% Text Node
\draw (90,189) node [anchor=north west][inner sep=0.75pt]  [rotate=-90]  {$\cdots $};
% Text Node
\draw (210,277) node [anchor=north west][inner sep=0.75pt]  [rotate=-90]  {$\cdots $};
% Text Node
\draw (333,189) node [anchor=north west][inner sep=0.75pt]  [rotate=-90]  {$\cdots $};
% Text Node
\draw (455,277) node [anchor=north west][inner sep=0.75pt]  [rotate=-90]  {$\cdots $};
% Text Node
\draw (493.6,342.2) node [anchor=north west][inner sep=0.75pt]    {$\cdots $};
% Text Node
\draw (508,177) node [anchor=north west][inner sep=0.75pt]    {$\cdots $};
% Text Node
\draw (18.07,62.33) node [anchor=north west][inner sep=0.75pt]  [font=\footnotesize]  {$\theta _{1}^{1}$};
% Text Node
\draw (17.6,85.4) node [anchor=north west][inner sep=0.75pt]  [font=\footnotesize]  {$\theta _{2}^{1}$};
% Text Node
\draw (17.8,122.2) node [anchor=north west][inner sep=0.75pt]  [font=\footnotesize]  {$\theta _{S}^{1}$};
% Text Node
\draw (57,148.4) node [anchor=north west][inner sep=0.75pt]  [font=\footnotesize]  {$\kappa _{1}^{1}$};
% Text Node
\draw (57.4,172) node [anchor=north west][inner sep=0.75pt]  [font=\footnotesize]  {$\kappa _{2}^{1}$};
% Text Node
\draw (57,209.2) node [anchor=north west][inner sep=0.75pt]  [font=\footnotesize]  {$\kappa _{S}^{1}$};
% Text Node
\draw (103.6,149.8) node [anchor=north west][inner sep=0.75pt]  [font=\footnotesize]  {$\rho _{1}^{1}$};
% Text Node
\draw (104,172.4) node [anchor=north west][inner sep=0.75pt]  [font=\footnotesize]  {$\rho _{2}^{1}$};
% Text Node
\draw (102.8,208.6) node [anchor=north west][inner sep=0.75pt]  [font=\footnotesize]  {$\rho _{S}^{1}$};
% Text Node
\draw (178.8,236.2) node [anchor=north west][inner sep=0.75pt]  [font=\footnotesize]  {$\lambda _{1}^{1}$};
% Text Node
\draw (178,259.4) node [anchor=north west][inner sep=0.75pt]  [font=\footnotesize]  {$\lambda _{2}^{1}$};
% Text Node
\draw (178.4,296.6) node [anchor=north west][inner sep=0.75pt]  [font=\footnotesize]  {$\lambda _{S}^{1}$};
% Text Node
\draw (224.4,235.4) node [anchor=north west][inner sep=0.75pt]  [font=\footnotesize]  {$\varepsilon _{1}^{1}$};
% Text Node
\draw (225.2,259) node [anchor=north west][inner sep=0.75pt]  [font=\footnotesize]  {$\varepsilon _{2}^{1}$};
% Text Node
\draw (224.4,295.8) node [anchor=north west][inner sep=0.75pt]  [font=\footnotesize]  {$\varepsilon _{S}^{1}$};
% Text Node
\draw (258.67,63) node [anchor=north west][inner sep=0.75pt]  [font=\footnotesize]  {$\theta _{1}^{2}$};
% Text Node
\draw (259.2,86.07) node [anchor=north west][inner sep=0.75pt]  [font=\footnotesize]  {$\theta _{2}^{2}$};
% Text Node
\draw (258.4,122.87) node [anchor=north west][inner sep=0.75pt]  [font=\footnotesize]  {$\theta _{S}^{2}$};
% Text Node
\draw (297.6,150.07) node [anchor=north west][inner sep=0.75pt]  [font=\footnotesize]  {$\kappa _{1}^{2}$};
% Text Node
\draw (298,173.67) node [anchor=north west][inner sep=0.75pt]  [font=\footnotesize]  {$\kappa _{2}^{2}$};
% Text Node
\draw (297.6,210.87) node [anchor=north west][inner sep=0.75pt]  [font=\footnotesize]  {$\kappa _{S}^{2}$};
% Text Node
\draw (345.2,150.47) node [anchor=north west][inner sep=0.75pt]  [font=\footnotesize]  {$\rho _{1}^{2}$};
% Text Node
\draw (345.6,173.07) node [anchor=north west][inner sep=0.75pt]  [font=\footnotesize]  {$\rho _{2}^{2}$};
% Text Node
\draw (344.4,209.27) node [anchor=north west][inner sep=0.75pt]  [font=\footnotesize]  {$\rho _{S}^{2}$};
% Text Node
\draw (420.4,236.87) node [anchor=north west][inner sep=0.75pt]  [font=\footnotesize]  {$\lambda _{1}^{2}$};
% Text Node
\draw (419.6,260.07) node [anchor=north west][inner sep=0.75pt]  [font=\footnotesize]  {$\lambda _{2}^{2}$};
% Text Node
\draw (420,297.27) node [anchor=north west][inner sep=0.75pt]  [font=\footnotesize]  {$\lambda _{S}^{2}$};
% Text Node
\draw (466,236.07) node [anchor=north west][inner sep=0.75pt]  [font=\footnotesize]  {$\varepsilon _{1}^{2}$};
% Text Node
\draw (466.8,259.67) node [anchor=north west][inner sep=0.75pt]  [font=\footnotesize]  {$\varepsilon _{2}^{2}$};
% Text Node
\draw (466,296.47) node [anchor=north west][inner sep=0.75pt]  [font=\footnotesize]  {$\varepsilon _{S}^{2}$};
% Text Node
\draw (566.8,178.87) node [anchor=north west][inner sep=0.75pt]  [font=\footnotesize]  {$\mu $};
% Text Node
\draw (288,378.07) node [anchor=north west][inner sep=0.75pt]  [font=\footnotesize]  {$\delta $};
% Text Node
\draw (330.8,343.8) node [anchor=north west][inner sep=0.75pt]  [font=\footnotesize]  {$\tau _{1}$};
% Text Node
\draw (469.2,343) node [anchor=north west][inner sep=0.75pt]  [font=\footnotesize]  {$\tau _{2}$};
% Text Node
\draw (518,343.4) node [anchor=north west][inner sep=0.75pt]  [font=\footnotesize]  {$\tau _{N}$};
% Text Node
\draw (548,339) node [anchor=north west][inner sep=0.75pt]   [align=left] {Tasks};


\end{tikzpicture}




    \end{center}
    \caption{The graph for which finding the min cost flow gives the minimized CF in the edge computing network.}\label{fig:networkflow}
%\vspace*{10pt}
%\hrule height 0.03cm 
\end{figure*}

% % \SetCoordinates[xAngle=0,yAngle=60,yLength=1,xLength=1]

\begin{figure*}[t]
    \begin{center}


\begin{tikzpicture}[multilayer=3d]
\SetLayerDistance{4}

\begin{Layer}[layer=1]
%   \draw[orange,very thick] (0,0) rectangle (2.5,1);
  \draw[step=.5, orange,draw opacity=.5] (0,0) grid (10,4);
  \node at (0,0)[below right,orange]{Layer 1};
\end{Layer}

\begin{Layer}[layer=2]
%   \draw[orange,very thick] (0,0) rectangle (2.5,1);
  \draw[step=.5, orange,draw opacity=.5] (0,0) grid (10,4);
  \node at (0,0)[below right,orange]{Layer 2};
\end{Layer}

\Vertex[x=0.5,y=.5,label = $\theta^t_s$, fontsize=\normalsize, size = .7, layer=2, color = green!50]{theta_1_1}
\Vertex[x=0.5,y=2.5,label = $\theta^t_s$, fontsize=\normalsize, size = .7, layer=2, color = green!50]{theta_1_2}
\Vertex[x=0.5,y=3.5,label = $\theta^t_s$, fontsize=\normalsize, size = .7, layer=2, color = green!50]{theta_1_S}

\Vertex[x=0.5,y=.5,label = $\theta^t_s$, fontsize=\normalsize, size = .7, layer=2, color = green!50]{theta_2_1}
\Vertex[x=0.5,y=2.5,label = $\theta^t_s$, fontsize=\normalsize, size = .7, layer=2, color = green!50]{theta_2_2}
\Vertex[x=0.5,y=3.5,label = $\theta^t_s$, fontsize=\normalsize, size = .7, layer=2, color = green!50]{theta_2_S}

% \Vertex[x=0.5,y=.5,label = $\theta^t_s$, fontsize=\normalsize, size = .7, layer=2, color = green]{theta_1_1}
% \Vertex[x=0.5,y=2.5,label = $\theta^t_s$, fontsize=\normalsize, size = .7, layer=2, color = green]{theta_2_1}
% \Vertex[x=0.5,y=3.5,label = $\theta^t_s$, fontsize=\normalsize, size = .7, layer=2, color = green]{theta_S_1}

\Vertex[x=0.5,y=.5,label = $\kappa^t_s$, fontsize=\normalsize, size = .7, layer=1, color = red!50]{kappa_1_1}
\Vertex[x=0.5,y=2.5,label = $\kappa^t_s$, fontsize=\normalsize, size = .7, layer=1, color = red!50]{kappa_1_2}
\Vertex[x=0.5,y=3.5,label = $\kappa^t_s$, fontsize=\normalsize, size = .7, layer=1, color = red!50]{kappa_1_S}

\Vertex[x=2.5,y=.5,label = $\theta^t_s$, fontsize=\normalsize, size = .7, layer=1, color = red!50]{rho_1_1}
\Vertex[x=2.5,y=2.5,label = $\theta^t_s$, fontsize=\normalsize, size = .7, layer=1, color = red!50]{rho_1_2}
\Vertex[x=2.5,y=3.5,label = $\theta^t_s$, fontsize=\normalsize, size = .7, layer=1, color = red!50]{rho_1_S}


% \Edge[bend=60](C)(B)
\Edge[Direct](theta_1_1)(kappa_1_1)
% \Edge(C)(C)
% \Edge[style=dashed](B)(C)

\Edge[Direct](kappa_1_1)(rho_1_1)


\end{tikzpicture}


    \end{center}
    \caption{The graph for which finding the min cost flow gives the minimized CF in the edge computing network.}\label{fig:networkflow}
\end{figure*}

\begin{figure*}[t]
    \begin{center}








\tikzset{every picture/.style={line width=0.75pt}} %set default line width to 0.75pt        

\begin{tikzpicture}[x=0.75pt,y=0.75pt,yscale=-1,xscale=1]
%uncomment if require: \path (0,453); %set diagram left start at 0, and has height of 453
\clip (0,20) rectangle (590, 430);

%Shape: Rectangle [id:dp9902399099893548] 
\draw  [color={rgb, 255:red, 0; green, 0; blue, 0 }  ,draw opacity=1 ][fill={rgb, 255:red, 255; green, 255; blue, 255 }  ,fill opacity=0.8 ][dash pattern={on 0.84pt off 2.51pt}] (495.67,30.5) -- (542,30.5) -- (542,342.5) -- (495.67,342.5) -- cycle ;
%Shape: Rectangle [id:dp3122464555551674] 
\draw  [color={rgb, 255:red, 0; green, 0; blue, 0 }  ,draw opacity=1 ][fill={rgb, 255:red, 255; green, 255; blue, 255 }  ,fill opacity=0.8 ][dash pattern={on 0.84pt off 2.51pt}] (254.33,30) -- (484.67,30) -- (484.67,342.5) -- (254.33,342.5) -- cycle ;
%Curve Lines [id:da038804301541383035] 
\draw    (300,404.43) .. controls (369.9,384.35) and (349.57,363.21) .. (417.87,276.92) ;
\draw [shift={(418.9,275.62)}, rotate = 128.06] [fill={rgb, 255:red, 0; green, 0; blue, 0 }  ][line width=0.08]  [draw opacity=0] (7.2,-1.8) -- (0,0) -- (7.2,1.8) -- cycle    ;
%Curve Lines [id:da8936189159998054] 
\draw    (301.29,407) .. controls (374.13,385.11) and (359.65,359.92) .. (418.3,299.34) ;
\draw [shift={(419.19,298.43)}, rotate = 133.69] [fill={rgb, 255:red, 0; green, 0; blue, 0 }  ][line width=0.08]  [draw opacity=0] (7.2,-1.8) -- (0,0) -- (7.2,1.8) -- cycle    ;
%Curve Lines [id:da3514522204329813] 
\draw    (301.6,409.6) .. controls (370.64,391.76) and (375.44,350.65) .. (416.83,332.45) ;
\draw [shift={(418.09,331.91)}, rotate = 155.81] [fill={rgb, 255:red, 0; green, 0; blue, 0 }  ][line width=0.08]  [draw opacity=0] (7.2,-1.8) -- (0,0) -- (7.2,1.8) -- cycle    ;
%Curve Lines [id:da13574990988185998] 
\draw    (299.57,415.14) .. controls (512.33,403.83) and (566,460.4) .. (571.97,219.33) ;
\draw [shift={(571.97,219.33)}, rotate = 91.42] [fill={rgb, 255:red, 0; green, 0; blue, 0 }  ][line width=0.08]  [draw opacity=0] (7.2,-1.8) -- (0,0) -- (7.2,1.8) -- cycle    ;
%Shape: Rectangle [id:dp6941285509047341] 
\draw  [color={rgb, 255:red, 0; green, 0; blue, 0 }  ,draw opacity=1 ][fill={rgb, 255:red, 255; green, 255; blue, 255 }  ,fill opacity=0.8 ][dash pattern={on 0.84pt off 2.51pt}] (312.5,355.08) -- (587.56,355.08) -- (587.56,388.33) -- (312.5,388.33) -- cycle ;
%Shape: Rectangle [id:dp3460157697146131] 
\draw  [color={rgb, 255:red, 0; green, 0; blue, 0 }  ,draw opacity=1 ][fill={rgb, 255:red, 255; green, 255; blue, 255 }  ,fill opacity=0.8 ][dash pattern={on 0.84pt off 2.51pt}] (13.13,31) -- (242.4,31) -- (242.4,342.9) -- (13.13,342.9) -- cycle ;
%Shape: Circle [id:dp593919605681948] 
\draw  [color={rgb, 255:red, 208; green, 2; blue, 27 }  ,draw opacity=1 ][fill={rgb, 255:red, 255; green, 255; blue, 255 }  ,fill opacity=1 ] (55.83,182.33) .. controls (55.83,177.09) and (60.09,172.83) .. (65.33,172.83) .. controls (70.58,172.83) and (74.83,177.09) .. (74.83,182.33) .. controls (74.83,187.58) and (70.58,191.83) .. (65.33,191.83) .. controls (60.09,191.83) and (55.83,187.58) .. (55.83,182.33) -- cycle ;
%Shape: Circle [id:dp4139267441151957] 
\draw  [color={rgb, 255:red, 208; green, 2; blue, 27 }  ,draw opacity=1 ][fill={rgb, 255:red, 255; green, 255; blue, 255 }  ,fill opacity=1 ] (101.17,182.33) .. controls (101.17,177.09) and (105.42,172.83) .. (110.67,172.83) .. controls (115.91,172.83) and (120.17,177.09) .. (120.17,182.33) .. controls (120.17,187.58) and (115.91,191.83) .. (110.67,191.83) .. controls (105.42,191.83) and (101.17,187.58) .. (101.17,182.33) -- cycle ;
%Straight Lines [id:da08608125520746612] 
\draw    (74.83,182.33) -- (99.17,182.33) ;
\draw [shift={(101.17,182.33)}, rotate = 180] [fill={rgb, 255:red, 0; green, 0; blue, 0 }  ][line width=0.08]  [draw opacity=0] (7.2,-1.8) -- (0,0) -- (7.2,1.8) -- cycle    ;
%Shape: Circle [id:dp25420348525468484] 
\draw  [color={rgb, 255:red, 208; green, 2; blue, 27 }  ,draw opacity=1 ][fill={rgb, 255:red, 255; green, 255; blue, 255 }  ,fill opacity=1 ] (55.8,205.83) .. controls (55.8,200.59) and (60.05,196.33) .. (65.3,196.33) .. controls (70.55,196.33) and (74.8,200.59) .. (74.8,205.83) .. controls (74.8,211.08) and (70.55,215.33) .. (65.3,215.33) .. controls (60.05,215.33) and (55.8,211.08) .. (55.8,205.83) -- cycle ;
%Shape: Circle [id:dp9166573217183533] 
\draw  [color={rgb, 255:red, 208; green, 2; blue, 27 }  ,draw opacity=1 ][fill={rgb, 255:red, 255; green, 255; blue, 255 }  ,fill opacity=1 ] (101.13,205.83) .. controls (101.13,200.59) and (105.39,196.33) .. (110.63,196.33) .. controls (115.88,196.33) and (120.13,200.59) .. (120.13,205.83) .. controls (120.13,211.08) and (115.88,215.33) .. (110.63,215.33) .. controls (105.39,215.33) and (101.13,211.08) .. (101.13,205.83) -- cycle ;
%Straight Lines [id:da6142941969360483] 
\draw    (74.8,205.83) -- (99.13,205.83) ;
\draw [shift={(101.13,205.83)}, rotate = 180] [fill={rgb, 255:red, 0; green, 0; blue, 0 }  ][line width=0.08]  [draw opacity=0] (7.2,-1.8) -- (0,0) -- (7.2,1.8) -- cycle    ;
%Shape: Circle [id:dp2558096751159191] 
\draw  [color={rgb, 255:red, 208; green, 2; blue, 27 }  ,draw opacity=1 ][fill={rgb, 255:red, 255; green, 255; blue, 255 }  ,fill opacity=1 ] (55.8,242.67) .. controls (55.8,237.42) and (60.05,233.17) .. (65.3,233.17) .. controls (70.55,233.17) and (74.8,237.42) .. (74.8,242.67) .. controls (74.8,247.91) and (70.55,252.17) .. (65.3,252.17) .. controls (60.05,252.17) and (55.8,247.91) .. (55.8,242.67) -- cycle ;
%Shape: Circle [id:dp9463226177671344] 
\draw  [color={rgb, 255:red, 208; green, 2; blue, 27 }  ,draw opacity=1 ][fill={rgb, 255:red, 255; green, 255; blue, 255 }  ,fill opacity=1 ] (101.13,242.67) .. controls (101.13,237.42) and (105.39,233.17) .. (110.63,233.17) .. controls (115.88,233.17) and (120.13,237.42) .. (120.13,242.67) .. controls (120.13,247.91) and (115.88,252.17) .. (110.63,252.17) .. controls (105.39,252.17) and (101.13,247.91) .. (101.13,242.67) -- cycle ;
%Straight Lines [id:da680976523746148] 
\draw    (74.8,242.67) -- (99.13,242.67) ;
\draw [shift={(101.13,242.67)}, rotate = 180] [fill={rgb, 255:red, 0; green, 0; blue, 0 }  ][line width=0.08]  [draw opacity=0] (7.2,-1.8) -- (0,0) -- (7.2,1.8) -- cycle    ;
%Shape: Circle [id:dp44439668591732095] 
\draw  [color={rgb, 255:red, 126; green, 211; blue, 33 }  ,draw opacity=1 ][fill={rgb, 255:red, 255; green, 255; blue, 255 }  ,fill opacity=1 ] (15,95.5) .. controls (15,90.25) and (19.25,86) .. (24.5,86) .. controls (29.75,86) and (34,90.25) .. (34,95.5) .. controls (34,100.75) and (29.75,105) .. (24.5,105) .. controls (19.25,105) and (15,100.75) .. (15,95.5) -- cycle ;
%Shape: Circle [id:dp9621130956289814] 
\draw  [color={rgb, 255:red, 126; green, 211; blue, 33 }  ,draw opacity=1 ][fill={rgb, 255:red, 255; green, 255; blue, 255 }  ,fill opacity=1 ] (15,118.8) .. controls (15,113.55) and (19.25,109.3) .. (24.5,109.3) .. controls (29.75,109.3) and (34,113.55) .. (34,118.8) .. controls (34,124.05) and (29.75,128.3) .. (24.5,128.3) .. controls (19.25,128.3) and (15,124.05) .. (15,118.8) -- cycle ;
%Shape: Circle [id:dp6789068245318313] 
\draw  [color={rgb, 255:red, 126; green, 211; blue, 33 }  ,draw opacity=1 ][fill={rgb, 255:red, 255; green, 255; blue, 255 }  ,fill opacity=1 ] (15,155.8) .. controls (15,150.55) and (19.25,146.3) .. (24.5,146.3) .. controls (29.75,146.3) and (34,150.55) .. (34,155.8) .. controls (34,161.05) and (29.75,165.3) .. (24.5,165.3) .. controls (19.25,165.3) and (15,161.05) .. (15,155.8) -- cycle ;
%Straight Lines [id:da9900400553640198] 
\draw    (31.25,102.75) -- (60.96,171.41) ;
\draw [shift={(61.75,173.25)}, rotate = 246.61] [fill={rgb, 255:red, 0; green, 0; blue, 0 }  ][line width=0.08]  [draw opacity=0] (7.2,-1.8) -- (0,0) -- (7.2,1.8) -- cycle    ;
%Straight Lines [id:da6873568632040385] 
\draw    (29.5,127.3) -- (57.5,196.89) ;
\draw [shift={(58.25,198.75)}, rotate = 248.08] [fill={rgb, 255:red, 0; green, 0; blue, 0 }  ][line width=0.08]  [draw opacity=0] (7.2,-1.8) -- (0,0) -- (7.2,1.8) -- cycle    ;
%Straight Lines [id:da43712383842750646] 
\draw    (30.75,163.25) -- (58.51,232.89) ;
\draw [shift={(59.25,234.75)}, rotate = 248.27] [fill={rgb, 255:red, 0; green, 0; blue, 0 }  ][line width=0.08]  [draw opacity=0] (7.2,-1.8) -- (0,0) -- (7.2,1.8) -- cycle    ;
%Curve Lines [id:da6823635267305639] 
\draw    (284.3,415.2) .. controls (-13.67,427.61) and (4.87,335.7) .. (56.75,187.25) ;
\draw [shift={(56.75,187.25)}, rotate = 109.22] [fill={rgb, 255:red, 0; green, 0; blue, 0 }  ][line width=0.08]  [draw opacity=0] (7.2,-1.8) -- (0,0) -- (7.2,1.8) -- cycle    ;
%Shape: Circle [id:dp09915510148134388] 
\draw  [color={rgb, 255:red, 74; green, 144; blue, 226 }  ,draw opacity=1 ][fill={rgb, 255:red, 255; green, 255; blue, 255 }  ,fill opacity=1 ] (177,269.5) .. controls (177,264.25) and (181.25,260) .. (186.5,260) .. controls (191.75,260) and (196,264.25) .. (196,269.5) .. controls (196,274.75) and (191.75,279) .. (186.5,279) .. controls (181.25,279) and (177,274.75) .. (177,269.5) -- cycle ;
%Shape: Circle [id:dp2106611699536416] 
\draw  [color={rgb, 255:red, 74; green, 144; blue, 226 }  ,draw opacity=1 ][fill={rgb, 255:red, 255; green, 255; blue, 255 }  ,fill opacity=1 ] (177,292.8) .. controls (177,287.55) and (181.25,283.3) .. (186.5,283.3) .. controls (191.75,283.3) and (196,287.55) .. (196,292.8) .. controls (196,298.05) and (191.75,302.3) .. (186.5,302.3) .. controls (181.25,302.3) and (177,298.05) .. (177,292.8) -- cycle ;
%Shape: Circle [id:dp25074856576749416] 
\draw  [color={rgb, 255:red, 74; green, 144; blue, 226 }  ,draw opacity=1 ][fill={rgb, 255:red, 255; green, 255; blue, 255 }  ,fill opacity=1 ] (177,329.8) .. controls (177,324.55) and (181.25,320.3) .. (186.5,320.3) .. controls (191.75,320.3) and (196,324.55) .. (196,329.8) .. controls (196,335.05) and (191.75,339.3) .. (186.5,339.3) .. controls (181.25,339.3) and (177,335.05) .. (177,329.8) -- cycle ;
%Straight Lines [id:da748472758265778] 
\draw    (119.6,185.6) -- (178.42,261.35) ;
\draw [shift={(179.64,262.93)}, rotate = 232.17] [fill={rgb, 255:red, 0; green, 0; blue, 0 }  ][line width=0.08]  [draw opacity=0] (7.2,-1.8) -- (0,0) -- (7.2,1.8) -- cycle    ;
%Straight Lines [id:da3965030230559381] 
\draw    (117.8,188.6) -- (178.29,284.1) ;
\draw [shift={(179.36,285.79)}, rotate = 237.65] [fill={rgb, 255:red, 0; green, 0; blue, 0 }  ][line width=0.08]  [draw opacity=0] (7.2,-1.8) -- (0,0) -- (7.2,1.8) -- cycle    ;
%Straight Lines [id:da8255076351126915] 
\draw    (115.6,190.8) -- (181.86,318.58) ;
\draw [shift={(182.79,320.36)}, rotate = 242.59] [fill={rgb, 255:red, 0; green, 0; blue, 0 }  ][line width=0.08]  [draw opacity=0] (7.2,-1.8) -- (0,0) -- (7.2,1.8) -- cycle    ;
%Straight Lines [id:da31689046304044277] 
\draw    (119.2,209.8) -- (175.92,264.68) ;
\draw [shift={(177.36,266.07)}, rotate = 224.06] [fill={rgb, 255:red, 0; green, 0; blue, 0 }  ][line width=0.08]  [draw opacity=0] (7.2,-1.8) -- (0,0) -- (7.2,1.8) -- cycle    ;
%Straight Lines [id:da8323923675685014] 
\draw    (117.2,212.6) -- (175.93,287.92) ;
\draw [shift={(177.16,289.5)}, rotate = 232.06] [fill={rgb, 255:red, 0; green, 0; blue, 0 }  ][line width=0.08]  [draw opacity=0] (7.2,-1.8) -- (0,0) -- (7.2,1.8) -- cycle    ;
%Straight Lines [id:da2609478512157011] 
\draw    (114.6,214.6) -- (178.33,321.5) ;
\draw [shift={(179.36,323.21)}, rotate = 239.2] [fill={rgb, 255:red, 0; green, 0; blue, 0 }  ][line width=0.08]  [draw opacity=0] (7.2,-1.8) -- (0,0) -- (7.2,1.8) -- cycle    ;
%Straight Lines [id:da21284023528449292] 
\draw    (119.6,246.4) -- (175.14,268.75) ;
\draw [shift={(177,269.5)}, rotate = 201.92] [fill={rgb, 255:red, 0; green, 0; blue, 0 }  ][line width=0.08]  [draw opacity=0] (7.2,-1.8) -- (0,0) -- (7.2,1.8) -- cycle    ;
%Straight Lines [id:da5547102757175062] 
\draw    (117.6,249) -- (175.39,291.61) ;
\draw [shift={(177,292.8)}, rotate = 216.4] [fill={rgb, 255:red, 0; green, 0; blue, 0 }  ][line width=0.08]  [draw opacity=0] (7.2,-1.8) -- (0,0) -- (7.2,1.8) -- cycle    ;
%Straight Lines [id:da1210776062190595] 
\draw    (115,251.6) -- (176.07,324.54) ;
\draw [shift={(177.36,326.07)}, rotate = 230.06] [fill={rgb, 255:red, 0; green, 0; blue, 0 }  ][line width=0.08]  [draw opacity=0] (7.2,-1.8) -- (0,0) -- (7.2,1.8) -- cycle    ;
%Shape: Circle [id:dp3111818739060366] 
\draw  [color={rgb, 255:red, 126; green, 211; blue, 33 }  ,draw opacity=1 ][fill={rgb, 255:red, 255; green, 255; blue, 255 }  ,fill opacity=1 ] (282.6,409.6) .. controls (282.6,404.35) and (286.85,400.1) .. (292.1,400.1) .. controls (297.35,400.1) and (301.6,404.35) .. (301.6,409.6) .. controls (301.6,414.85) and (297.35,419.1) .. (292.1,419.1) .. controls (286.85,419.1) and (282.6,414.85) .. (282.6,409.6) -- cycle ;
%Straight Lines [id:da7422309951426038] 
\draw    (120.17,182.5) -- (295.17,182.5) ;
\draw [shift={(297.17,182.5)}, rotate = 180] [fill={rgb, 255:red, 0; green, 0; blue, 0 }  ][line width=0.08]  [draw opacity=0] (7.2,-1.8) -- (0,0) -- (7.2,1.8) -- cycle    ;
%Straight Lines [id:da311090791695219] 
\draw    (120.13,205.83) -- (295.13,205.83) ;
\draw [shift={(297.13,205.83)}, rotate = 180] [fill={rgb, 255:red, 0; green, 0; blue, 0 }  ][line width=0.08]  [draw opacity=0] (7.2,-1.8) -- (0,0) -- (7.2,1.8) -- cycle    ;
%Straight Lines [id:da9404033824837681] 
\draw    (120.13,242.83) -- (295.13,242.83) ;
\draw [shift={(297.13,242.83)}, rotate = 180] [fill={rgb, 255:red, 0; green, 0; blue, 0 }  ][line width=0.08]  [draw opacity=0] (7.2,-1.8) -- (0,0) -- (7.2,1.8) -- cycle    ;
%Curve Lines [id:da2362736060765973] 
\draw    (16,99.67) .. controls (-7.33,195.75) and (-2.33,317.75) .. (176.4,272.47) ;
\draw [shift={(176.4,272.47)}, rotate = 166.92] [fill={rgb, 255:red, 0; green, 0; blue, 0 }  ][line width=0.08]  [draw opacity=0] (7.2,-1.8) -- (0,0) -- (7.2,1.8) -- cycle    ;
%Curve Lines [id:da08786040586631416] 
\draw    (17.38,125.63) .. controls (10.67,161.75) and (-16.44,301.17) .. (176.88,294.13) ;
\draw [shift={(176.88,294.13)}, rotate = 180.48] [fill={rgb, 255:red, 0; green, 0; blue, 0 }  ][line width=0.08]  [draw opacity=0] (7.2,-1.8) -- (0,0) -- (7.2,1.8) -- cycle    ;
%Curve Lines [id:da6253702613941365] 
\draw    (18.83,164) .. controls (12.7,232.41) and (61.38,293.02) .. (174.86,327.9) ;
\draw [shift={(176.57,328.43)}, rotate = 197.52] [fill={rgb, 255:red, 0; green, 0; blue, 0 }  ][line width=0.08]  [draw opacity=0] (7.2,-1.8) -- (0,0) -- (7.2,1.8) -- cycle    ;
%Shape: Circle [id:dp9172527974861038] 
\draw  [color={rgb, 255:red, 208; green, 2; blue, 27 }  ,draw opacity=1 ][fill={rgb, 255:red, 255; green, 255; blue, 255 }  ,fill opacity=1 ] (297.33,183.5) .. controls (297.33,178.25) and (301.59,174) .. (306.83,174) .. controls (312.08,174) and (316.33,178.25) .. (316.33,183.5) .. controls (316.33,188.75) and (312.08,193) .. (306.83,193) .. controls (301.59,193) and (297.33,188.75) .. (297.33,183.5) -- cycle ;
%Shape: Circle [id:dp5231178422085889] 
\draw  [color={rgb, 255:red, 208; green, 2; blue, 27 }  ,draw opacity=1 ][fill={rgb, 255:red, 255; green, 255; blue, 255 }  ,fill opacity=1 ] (342.67,183.5) .. controls (342.67,178.25) and (346.92,174) .. (352.17,174) .. controls (357.41,174) and (361.67,178.25) .. (361.67,183.5) .. controls (361.67,188.75) and (357.41,193) .. (352.17,193) .. controls (346.92,193) and (342.67,188.75) .. (342.67,183.5) -- cycle ;
%Straight Lines [id:da5968784915078358] 
\draw    (316.33,183.5) -- (340.67,183.5) ;
\draw [shift={(342.67,183.5)}, rotate = 180] [fill={rgb, 255:red, 0; green, 0; blue, 0 }  ][line width=0.08]  [draw opacity=0] (7.2,-1.8) -- (0,0) -- (7.2,1.8) -- cycle    ;
%Shape: Circle [id:dp873871893479828] 
\draw  [color={rgb, 255:red, 208; green, 2; blue, 27 }  ,draw opacity=1 ][fill={rgb, 255:red, 255; green, 255; blue, 255 }  ,fill opacity=1 ] (297.3,206.83) .. controls (297.3,201.59) and (301.55,197.33) .. (306.8,197.33) .. controls (312.05,197.33) and (316.3,201.59) .. (316.3,206.83) .. controls (316.3,212.08) and (312.05,216.33) .. (306.8,216.33) .. controls (301.55,216.33) and (297.3,212.08) .. (297.3,206.83) -- cycle ;
%Shape: Circle [id:dp14264306094926993] 
\draw  [color={rgb, 255:red, 208; green, 2; blue, 27 }  ,draw opacity=1 ][fill={rgb, 255:red, 255; green, 255; blue, 255 }  ,fill opacity=1 ] (342.63,206.83) .. controls (342.63,201.59) and (346.89,197.33) .. (352.13,197.33) .. controls (357.38,197.33) and (361.63,201.59) .. (361.63,206.83) .. controls (361.63,212.08) and (357.38,216.33) .. (352.13,216.33) .. controls (346.89,216.33) and (342.63,212.08) .. (342.63,206.83) -- cycle ;
%Straight Lines [id:da8117714174485189] 
\draw    (316.3,206.83) -- (340.63,206.83) ;
\draw [shift={(342.63,206.83)}, rotate = 180] [fill={rgb, 255:red, 0; green, 0; blue, 0 }  ][line width=0.08]  [draw opacity=0] (7.2,-1.8) -- (0,0) -- (7.2,1.8) -- cycle    ;
%Shape: Circle [id:dp9728559309176834] 
\draw  [color={rgb, 255:red, 208; green, 2; blue, 27 }  ,draw opacity=1 ][fill={rgb, 255:red, 255; green, 255; blue, 255 }  ,fill opacity=1 ] (297.3,243.83) .. controls (297.3,238.59) and (301.55,234.33) .. (306.8,234.33) .. controls (312.05,234.33) and (316.3,238.59) .. (316.3,243.83) .. controls (316.3,249.08) and (312.05,253.33) .. (306.8,253.33) .. controls (301.55,253.33) and (297.3,249.08) .. (297.3,243.83) -- cycle ;
%Shape: Circle [id:dp9833656880931798] 
\draw  [color={rgb, 255:red, 208; green, 2; blue, 27 }  ,draw opacity=1 ][fill={rgb, 255:red, 255; green, 255; blue, 255 }  ,fill opacity=1 ] (342.63,243.83) .. controls (342.63,238.59) and (346.89,234.33) .. (352.13,234.33) .. controls (357.38,234.33) and (361.63,238.59) .. (361.63,243.83) .. controls (361.63,249.08) and (357.38,253.33) .. (352.13,253.33) .. controls (346.89,253.33) and (342.63,249.08) .. (342.63,243.83) -- cycle ;
%Straight Lines [id:da5568476376308586] 
\draw    (316.3,243.83) -- (340.63,243.83) ;
\draw [shift={(342.63,243.83)}, rotate = 180] [fill={rgb, 255:red, 0; green, 0; blue, 0 }  ][line width=0.08]  [draw opacity=0] (7.2,-1.8) -- (0,0) -- (7.2,1.8) -- cycle    ;
%Shape: Circle [id:dp1179799709309528] 
\draw  [color={rgb, 255:red, 126; green, 211; blue, 33 }  ,draw opacity=1 ] (256.5,96.5) .. controls (256.5,91.25) and (260.75,87) .. (266,87) .. controls (271.25,87) and (275.5,91.25) .. (275.5,96.5) .. controls (275.5,101.75) and (271.25,106) .. (266,106) .. controls (260.75,106) and (256.5,101.75) .. (256.5,96.5) -- cycle ;
%Shape: Circle [id:dp4916621755630901] 
\draw  [color={rgb, 255:red, 126; green, 211; blue, 33 }  ,draw opacity=1 ] (256.5,119.8) .. controls (256.5,114.55) and (260.75,110.3) .. (266,110.3) .. controls (271.25,110.3) and (275.5,114.55) .. (275.5,119.8) .. controls (275.5,125.05) and (271.25,129.3) .. (266,129.3) .. controls (260.75,129.3) and (256.5,125.05) .. (256.5,119.8) -- cycle ;
%Shape: Circle [id:dp9076953584753158] 
\draw  [color={rgb, 255:red, 126; green, 211; blue, 33 }  ,draw opacity=1 ] (256.5,156.8) .. controls (256.5,151.55) and (260.75,147.3) .. (266,147.3) .. controls (271.25,147.3) and (275.5,151.55) .. (275.5,156.8) .. controls (275.5,162.05) and (271.25,166.3) .. (266,166.3) .. controls (260.75,166.3) and (256.5,162.05) .. (256.5,156.8) -- cycle ;
%Straight Lines [id:da9118419275275298] 
\draw    (272.75,103.75) -- (302.46,172.41) ;
\draw [shift={(303.25,174.25)}, rotate = 246.61] [fill={rgb, 255:red, 0; green, 0; blue, 0 }  ][line width=0.08]  [draw opacity=0] (7.2,-1.8) -- (0,0) -- (7.2,1.8) -- cycle    ;
%Straight Lines [id:da9676924899876029] 
\draw    (271,128.3) -- (299,197.89) ;
\draw [shift={(299.75,199.75)}, rotate = 248.08] [fill={rgb, 255:red, 0; green, 0; blue, 0 }  ][line width=0.08]  [draw opacity=0] (7.2,-1.8) -- (0,0) -- (7.2,1.8) -- cycle    ;
%Straight Lines [id:da8196893702332761] 
\draw    (272.25,164.25) -- (300.01,233.89) ;
\draw [shift={(300.75,235.75)}, rotate = 248.27] [fill={rgb, 255:red, 0; green, 0; blue, 0 }  ][line width=0.08]  [draw opacity=0] (7.2,-1.8) -- (0,0) -- (7.2,1.8) -- cycle    ;
%Shape: Circle [id:dp5522106490973764] 
\draw  [color={rgb, 255:red, 74; green, 144; blue, 226 }  ,draw opacity=1 ][fill={rgb, 255:red, 255; green, 255; blue, 255 }  ,fill opacity=1 ] (418.5,270.5) .. controls (418.5,265.25) and (422.75,261) .. (428,261) .. controls (433.25,261) and (437.5,265.25) .. (437.5,270.5) .. controls (437.5,275.75) and (433.25,280) .. (428,280) .. controls (422.75,280) and (418.5,275.75) .. (418.5,270.5) -- cycle ;
%Shape: Circle [id:dp9342115761880321] 
\draw  [color={rgb, 255:red, 74; green, 144; blue, 226 }  ,draw opacity=1 ][fill={rgb, 255:red, 255; green, 255; blue, 255 }  ,fill opacity=1 ] (418.5,293.8) .. controls (418.5,288.55) and (422.75,284.3) .. (428,284.3) .. controls (433.25,284.3) and (437.5,288.55) .. (437.5,293.8) .. controls (437.5,299.05) and (433.25,303.3) .. (428,303.3) .. controls (422.75,303.3) and (418.5,299.05) .. (418.5,293.8) -- cycle ;
%Shape: Circle [id:dp8459942019088817] 
\draw  [color={rgb, 255:red, 74; green, 144; blue, 226 }  ,draw opacity=1 ][fill={rgb, 255:red, 255; green, 255; blue, 255 }  ,fill opacity=1 ] (418.5,330.8) .. controls (418.5,325.55) and (422.75,321.3) .. (428,321.3) .. controls (433.25,321.3) and (437.5,325.55) .. (437.5,330.8) .. controls (437.5,336.05) and (433.25,340.3) .. (428,340.3) .. controls (422.75,340.3) and (418.5,336.05) .. (418.5,330.8) -- cycle ;
%Straight Lines [id:da09710004151743323] 
\draw    (361.1,186.6) -- (419.92,262.35) ;
\draw [shift={(421.14,263.93)}, rotate = 232.17] [fill={rgb, 255:red, 0; green, 0; blue, 0 }  ][line width=0.08]  [draw opacity=0] (7.2,-1.8) -- (0,0) -- (7.2,1.8) -- cycle    ;
%Straight Lines [id:da34635586705229726] 
\draw    (359.3,189.6) -- (419.79,285.1) ;
\draw [shift={(420.86,286.79)}, rotate = 237.65] [fill={rgb, 255:red, 0; green, 0; blue, 0 }  ][line width=0.08]  [draw opacity=0] (7.2,-1.8) -- (0,0) -- (7.2,1.8) -- cycle    ;
%Straight Lines [id:da4349576454406552] 
\draw    (357.1,191.8) -- (423.36,319.58) ;
\draw [shift={(424.29,321.36)}, rotate = 242.59] [fill={rgb, 255:red, 0; green, 0; blue, 0 }  ][line width=0.08]  [draw opacity=0] (7.2,-1.8) -- (0,0) -- (7.2,1.8) -- cycle    ;
%Straight Lines [id:da4563331916615285] 
\draw    (360.7,210.8) -- (417.42,265.68) ;
\draw [shift={(418.86,267.07)}, rotate = 224.06] [fill={rgb, 255:red, 0; green, 0; blue, 0 }  ][line width=0.08]  [draw opacity=0] (7.2,-1.8) -- (0,0) -- (7.2,1.8) -- cycle    ;
%Straight Lines [id:da061074266286135215] 
\draw    (358.7,213.6) -- (417.43,288.92) ;
\draw [shift={(418.66,290.5)}, rotate = 232.06] [fill={rgb, 255:red, 0; green, 0; blue, 0 }  ][line width=0.08]  [draw opacity=0] (7.2,-1.8) -- (0,0) -- (7.2,1.8) -- cycle    ;
%Straight Lines [id:da8260762281885707] 
\draw    (356.1,215.6) -- (419.83,322.5) ;
\draw [shift={(420.86,324.21)}, rotate = 239.2] [fill={rgb, 255:red, 0; green, 0; blue, 0 }  ][line width=0.08]  [draw opacity=0] (7.2,-1.8) -- (0,0) -- (7.2,1.8) -- cycle    ;
%Straight Lines [id:da4076931125124921] 
\draw    (361.1,247.4) -- (416.64,269.75) ;
\draw [shift={(418.5,270.5)}, rotate = 201.92] [fill={rgb, 255:red, 0; green, 0; blue, 0 }  ][line width=0.08]  [draw opacity=0] (7.2,-1.8) -- (0,0) -- (7.2,1.8) -- cycle    ;
%Straight Lines [id:da272324322985684] 
\draw    (359.1,250) -- (416.89,292.61) ;
\draw [shift={(418.5,293.8)}, rotate = 216.4] [fill={rgb, 255:red, 0; green, 0; blue, 0 }  ][line width=0.08]  [draw opacity=0] (7.2,-1.8) -- (0,0) -- (7.2,1.8) -- cycle    ;
%Straight Lines [id:da5942246569149254] 
\draw    (356.5,252.6) -- (417.57,325.54) ;
\draw [shift={(418.86,327.07)}, rotate = 230.06] [fill={rgb, 255:red, 0; green, 0; blue, 0 }  ][line width=0.08]  [draw opacity=0] (7.2,-1.8) -- (0,0) -- (7.2,1.8) -- cycle    ;
%Curve Lines [id:da4226750962481063] 
\draw    (257.5,101) .. controls (236.67,179.75) and (243.67,295.75) .. (417.9,273.8) ;
\draw [shift={(417.9,273.8)}, rotate = 173.98] [fill={rgb, 255:red, 0; green, 0; blue, 0 }  ][line width=0.08]  [draw opacity=0] (7.2,-1.8) -- (0,0) -- (7.2,1.8) -- cycle    ;
%Curve Lines [id:da7657131181018171] 
\draw    (258.88,126.63) .. controls (253.17,154.75) and (228.67,291.25) .. (418.38,295.13) ;
\draw [shift={(418.38,295.13)}, rotate = 184.2] [fill={rgb, 255:red, 0; green, 0; blue, 0 }  ][line width=0.08]  [draw opacity=0] (7.2,-1.8) -- (0,0) -- (7.2,1.8) -- cycle    ;
%Curve Lines [id:da7219861824522342] 
\draw    (261.17,165.5) .. controls (264.67,241.75) and (283.38,281.25) .. (418.07,329.43) ;
\draw [shift={(418.07,329.43)}, rotate = 200.06] [fill={rgb, 255:red, 0; green, 0; blue, 0 }  ][line width=0.08]  [draw opacity=0] (7.2,-1.8) -- (0,0) -- (7.2,1.8) -- cycle    ;
%Straight Lines [id:da6954685307859836] 
\draw    (196,269.5) -- (220.33,269.5) ;
\draw [shift={(222.33,269.5)}, rotate = 180] [fill={rgb, 255:red, 0; green, 0; blue, 0 }  ][line width=0.08]  [draw opacity=0] (7.2,-1.8) -- (0,0) -- (7.2,1.8) -- cycle    ;
%Shape: Circle [id:dp06599842740642958] 
\draw  [color={rgb, 255:red, 74; green, 144; blue, 226 }  ,draw opacity=1 ][fill={rgb, 255:red, 255; green, 255; blue, 255 }  ,fill opacity=1 ] (222.33,269.5) .. controls (222.33,264.25) and (226.59,260) .. (231.83,260) .. controls (237.08,260) and (241.33,264.25) .. (241.33,269.5) .. controls (241.33,274.75) and (237.08,279) .. (231.83,279) .. controls (226.59,279) and (222.33,274.75) .. (222.33,269.5) -- cycle ;
%Straight Lines [id:da8831164181259743] 
\draw    (196,292.8) -- (220.33,292.8) ;
\draw [shift={(222.33,292.8)}, rotate = 180] [fill={rgb, 255:red, 0; green, 0; blue, 0 }  ][line width=0.08]  [draw opacity=0] (7.2,-1.8) -- (0,0) -- (7.2,1.8) -- cycle    ;
%Shape: Circle [id:dp6632665263641222] 
\draw  [color={rgb, 255:red, 74; green, 144; blue, 226 }  ,draw opacity=1 ][fill={rgb, 255:red, 255; green, 255; blue, 255 }  ,fill opacity=1 ] (222.33,292.8) .. controls (222.33,287.55) and (226.59,283.3) .. (231.83,283.3) .. controls (237.08,283.3) and (241.33,287.55) .. (241.33,292.8) .. controls (241.33,298.05) and (237.08,302.3) .. (231.83,302.3) .. controls (226.59,302.3) and (222.33,298.05) .. (222.33,292.8) -- cycle ;
%Straight Lines [id:da1717231570739859] 
\draw    (196,329.8) -- (220.33,329.8) ;
\draw [shift={(222.33,329.8)}, rotate = 180] [fill={rgb, 255:red, 0; green, 0; blue, 0 }  ][line width=0.08]  [draw opacity=0] (7.2,-1.8) -- (0,0) -- (7.2,1.8) -- cycle    ;
%Shape: Circle [id:dp36553179592098783] 
\draw  [color={rgb, 255:red, 74; green, 144; blue, 226 }  ,draw opacity=1 ][fill={rgb, 255:red, 255; green, 255; blue, 255 }  ,fill opacity=1 ] (222.33,329.8) .. controls (222.33,324.55) and (226.59,320.3) .. (231.83,320.3) .. controls (237.08,320.3) and (241.33,324.55) .. (241.33,329.8) .. controls (241.33,335.05) and (237.08,339.3) .. (231.83,339.3) .. controls (226.59,339.3) and (222.33,335.05) .. (222.33,329.8) -- cycle ;
%Straight Lines [id:da9597099251108501] 
\draw    (437.5,270.5) -- (461.83,270.5) ;
\draw [shift={(463.83,270.5)}, rotate = 180] [fill={rgb, 255:red, 0; green, 0; blue, 0 }  ][line width=0.08]  [draw opacity=0] (7.2,-1.8) -- (0,0) -- (7.2,1.8) -- cycle    ;
%Straight Lines [id:da7937919892966077] 
\draw    (437.5,293.8) -- (461.83,293.8) ;
\draw [shift={(463.83,293.8)}, rotate = 180] [fill={rgb, 255:red, 0; green, 0; blue, 0 }  ][line width=0.08]  [draw opacity=0] (7.2,-1.8) -- (0,0) -- (7.2,1.8) -- cycle    ;
%Straight Lines [id:da2541460536893574] 
\draw    (437.5,330.8) -- (461.83,330.8) ;
\draw [shift={(463.83,330.8)}, rotate = 180] [fill={rgb, 255:red, 0; green, 0; blue, 0 }  ][line width=0.08]  [draw opacity=0] (7.2,-1.8) -- (0,0) -- (7.2,1.8) -- cycle    ;
%Shape: Circle [id:dp17368365059269686] 
\draw  [color={rgb, 255:red, 74; green, 144; blue, 226 }  ,draw opacity=1 ][fill={rgb, 255:red, 255; green, 255; blue, 255 }  ,fill opacity=1 ] (463.83,270.5) .. controls (463.83,265.25) and (468.09,261) .. (473.33,261) .. controls (478.58,261) and (482.83,265.25) .. (482.83,270.5) .. controls (482.83,275.75) and (478.58,280) .. (473.33,280) .. controls (468.09,280) and (463.83,275.75) .. (463.83,270.5) -- cycle ;
%Shape: Circle [id:dp2672571368525132] 
\draw  [color={rgb, 255:red, 74; green, 144; blue, 226 }  ,draw opacity=1 ][fill={rgb, 255:red, 255; green, 255; blue, 255 }  ,fill opacity=1 ] (463.83,293.8) .. controls (463.83,288.55) and (468.09,284.3) .. (473.33,284.3) .. controls (478.58,284.3) and (482.83,288.55) .. (482.83,293.8) .. controls (482.83,299.05) and (478.58,303.3) .. (473.33,303.3) .. controls (468.09,303.3) and (463.83,299.05) .. (463.83,293.8) -- cycle ;
%Shape: Circle [id:dp708525032532848] 
\draw  [color={rgb, 255:red, 74; green, 144; blue, 226 }  ,draw opacity=1 ][fill={rgb, 255:red, 255; green, 255; blue, 255 }  ,fill opacity=1 ] (463.83,330.8) .. controls (463.83,325.55) and (468.09,321.3) .. (473.33,321.3) .. controls (478.58,321.3) and (482.83,325.55) .. (482.83,330.8) .. controls (482.83,336.05) and (478.58,340.3) .. (473.33,340.3) .. controls (468.09,340.3) and (463.83,336.05) .. (463.83,330.8) -- cycle ;
%Shape: Circle [id:dp7255190009372188] 
\draw  [color={rgb, 255:red, 245; green, 166; blue, 35 }  ,draw opacity=1 ][fill={rgb, 255:red, 255; green, 255; blue, 255 }  ,fill opacity=1 ] (326.47,373.5) .. controls (326.47,368.25) and (330.72,364) .. (335.97,364) .. controls (341.21,364) and (345.47,368.25) .. (345.47,373.5) .. controls (345.47,378.75) and (341.21,383) .. (335.97,383) .. controls (330.72,383) and (326.47,378.75) .. (326.47,373.5) -- cycle ;
%Straight Lines [id:da638120539610346] 
\draw    (361.67,183.25) -- (499.8,183.25) ;
\draw [shift={(501.8,183.25)}, rotate = 180] [fill={rgb, 255:red, 0; green, 0; blue, 0 }  ][line width=0.08]  [draw opacity=0] (7.2,-1.8) -- (0,0) -- (7.2,1.8) -- cycle    ;
%Shape: Circle [id:dp6824735541696747] 
\draw  [color={rgb, 255:red, 245; green, 166; blue, 35 }  ,draw opacity=1 ][fill={rgb, 255:red, 255; green, 255; blue, 255 }  ,fill opacity=1 ] (562.47,209.83) .. controls (562.47,204.59) and (566.72,200.33) .. (571.97,200.33) .. controls (577.21,200.33) and (581.47,204.59) .. (581.47,209.83) .. controls (581.47,215.08) and (577.21,219.33) .. (571.97,219.33) .. controls (566.72,219.33) and (562.47,215.08) .. (562.47,209.83) -- cycle ;
%Shape: Circle [id:dp5156095412004806] 
\draw  [color={rgb, 255:red, 245; green, 166; blue, 35 }  ,draw opacity=1 ][fill={rgb, 255:red, 255; green, 255; blue, 255 }  ,fill opacity=1 ] (465.2,373.5) .. controls (465.2,368.25) and (469.45,364) .. (474.7,364) .. controls (479.95,364) and (484.2,368.25) .. (484.2,373.5) .. controls (484.2,378.75) and (479.95,383) .. (474.7,383) .. controls (469.45,383) and (465.2,378.75) .. (465.2,373.5) -- cycle ;
%Shape: Circle [id:dp6485856767533946] 
\draw  [color={rgb, 255:red, 245; green, 166; blue, 35 }  ,draw opacity=1 ][fill={rgb, 255:red, 255; green, 255; blue, 255 }  ,fill opacity=1 ] (515.73,373.5) .. controls (515.73,368.25) and (519.99,364) .. (525.23,364) .. controls (530.48,364) and (534.73,368.25) .. (534.73,373.5) .. controls (534.73,378.75) and (530.48,383) .. (525.23,383) .. controls (519.99,383) and (515.73,378.75) .. (515.73,373.5) -- cycle ;
%Curve Lines [id:da9247431008395106] 
\draw    (240,274.4) .. controls (258.86,357.64) and (240.08,385.38) .. (325.3,375.22) ;
\draw [shift={(326.59,375.06)}, rotate = 173.55] [fill={rgb, 255:red, 0; green, 0; blue, 0 }  ][line width=0.08]  [draw opacity=0] (7.2,-1.8) -- (0,0) -- (7.2,1.8) -- cycle    ;
%Curve Lines [id:da4147748923248351] 
\draw    (234.58,302.06) .. controls (256.47,328.37) and (218.58,397.93) .. (324.98,377.61) ;
\draw [shift={(326.6,377.3)}, rotate = 170.91] [fill={rgb, 255:red, 0; green, 0; blue, 0 }  ][line width=0.08]  [draw opacity=0] (7.2,-1.8) -- (0,0) -- (7.2,1.8) -- cycle    ;
%Curve Lines [id:da7153073087082051] 
\draw    (235.67,338.94) .. controls (232.13,359.95) and (257.59,402.51) .. (326.93,379.42) ;
\draw [shift={(327.97,379.06)}, rotate = 165.15] [fill={rgb, 255:red, 0; green, 0; blue, 0 }  ][line width=0.08]  [draw opacity=0] (7.2,-1.8) -- (0,0) -- (7.2,1.8) -- cycle    ;
%Curve Lines [id:da7753237943474862] 
\draw    (481.8,275.8) .. controls (506.4,344.4) and (491.4,379.4) .. (345.4,376.8) ;
\draw [shift={(345.4,376.8)}, rotate = 0.59] [fill={rgb, 255:red, 0; green, 0; blue, 0 }  ][line width=0.08]  [draw opacity=0] (7.2,-1.8) -- (0,0) -- (7.2,1.8) -- cycle    ;
%Curve Lines [id:da24332882355411245] 
\draw    (480.2,301) .. controls (505.41,352.04) and (450.73,371.31) .. (347.03,373.47) ;
\draw [shift={(345.47,373.5)}, rotate = 358.5] [fill={rgb, 255:red, 0; green, 0; blue, 0 }  ][line width=0.08]  [draw opacity=0] (7.2,-1.8) -- (0,0) -- (7.2,1.8) -- cycle    ;
%Curve Lines [id:da7099110881272985] 
\draw    (482.83,270.5) .. controls (508.8,312.17) and (501.78,335.14) .. (483.08,364.81) ;
\draw [shift={(482.22,366.17)}, rotate = 302] [fill={rgb, 255:red, 0; green, 0; blue, 0 }  ][line width=0.08]  [draw opacity=0] (7.2,-1.8) -- (0,0) -- (7.2,1.8) -- cycle    ;
%Curve Lines [id:da02238402771595016] 
\draw    (482.83,293.8) .. controls (501.32,310.94) and (497.02,334.45) .. (478.73,362.55) ;
\draw [shift={(477.89,363.83)}, rotate = 302.24] [fill={rgb, 255:red, 0; green, 0; blue, 0 }  ][line width=0.08]  [draw opacity=0] (7.2,-1.8) -- (0,0) -- (7.2,1.8) -- cycle    ;
%Straight Lines [id:da31851708212883434] 
\draw    (361.63,206.83) -- (499.4,206.83) ;
\draw [shift={(501.4,206.83)}, rotate = 180] [fill={rgb, 255:red, 0; green, 0; blue, 0 }  ][line width=0.08]  [draw opacity=0] (7.2,-1.8) -- (0,0) -- (7.2,1.8) -- cycle    ;
%Straight Lines [id:da09276398608555847] 
\draw    (361.63,243.83) -- (499,243.83) ;
\draw [shift={(501,243.83)}, rotate = 180] [fill={rgb, 255:red, 0; green, 0; blue, 0 }  ][line width=0.08]  [draw opacity=0] (7.2,-1.8) -- (0,0) -- (7.2,1.8) -- cycle    ;
%Curve Lines [id:da4867248015097767] 
\draw    (283.3,413.2) .. controls (-23.46,421.57) and (23.85,316.01) .. (56.29,213.38) ;
\draw [shift={(56.78,211.83)}, rotate = 107.54] [fill={rgb, 255:red, 0; green, 0; blue, 0 }  ][line width=0.08]  [draw opacity=0] (7.2,-1.8) -- (0,0) -- (7.2,1.8) -- cycle    ;
%Curve Lines [id:da6225287245391646] 
\draw    (282.75,411.04) .. controls (-29.45,414.6) and (41.79,293.36) .. (56.44,248.24) ;
\draw [shift={(56.87,246.9)}, rotate = 107.6] [fill={rgb, 255:red, 0; green, 0; blue, 0 }  ][line width=0.08]  [draw opacity=0] (7.2,-1.8) -- (0,0) -- (7.2,1.8) -- cycle    ;
%Curve Lines [id:da14670118842375524] 
\draw    (282.57,407.14) .. controls (71.4,376.63) and (43.8,317.83) .. (178.6,275.6) ;
\draw [shift={(178.6,275.6)}, rotate = 162.34] [fill={rgb, 255:red, 0; green, 0; blue, 0 }  ][line width=0.08]  [draw opacity=0] (7.2,-1.8) -- (0,0) -- (7.2,1.8) -- cycle    ;
%Curve Lines [id:da05931190530278174] 
\draw    (283.71,404.57) .. controls (26.29,358.86) and (96.68,307.15) .. (176.2,297.34) ;
\draw [shift={(177.4,297.2)}, rotate = 172.97] [fill={rgb, 255:red, 0; green, 0; blue, 0 }  ][line width=0.08]  [draw opacity=0] (7.2,-1.8) -- (0,0) -- (7.2,1.8) -- cycle    ;
%Curve Lines [id:da6126326050141437] 
\draw    (285.43,402.86) .. controls (45.01,352.88) and (102.42,311.45) .. (175.5,331.53) ;
\draw [shift={(176.6,331.83)}, rotate = 195.36] [fill={rgb, 255:red, 0; green, 0; blue, 0 }  ][line width=0.08]  [draw opacity=0] (7.2,-1.8) -- (0,0) -- (7.2,1.8) -- cycle    ;
%Curve Lines [id:da631546261097365] 
\draw    (289.71,400.86) .. controls (272.8,342.65) and (254.58,269.34) .. (297.19,188.47) ;
\draw [shift={(297.83,187.25)}, rotate = 116.83] [fill={rgb, 255:red, 0; green, 0; blue, 0 }  ][line width=0.08]  [draw opacity=0] (7.2,-1.8) -- (0,0) -- (7.2,1.8) -- cycle    ;
%Curve Lines [id:da6609630790590302] 
\draw    (292.1,400.1) .. controls (283.18,342.74) and (265.48,295.77) .. (297.84,212.02) ;
\draw [shift={(298.33,210.75)}, rotate = 110.48] [fill={rgb, 255:red, 0; green, 0; blue, 0 }  ][line width=0.08]  [draw opacity=0] (7.2,-1.8) -- (0,0) -- (7.2,1.8) -- cycle    ;
%Curve Lines [id:da6265302997948612] 
\draw    (294.5,400.08) .. controls (285.26,314.94) and (284.44,301.83) .. (298.57,251.67) ;
\draw [shift={(299,250.14)}, rotate = 105.68] [fill={rgb, 255:red, 0; green, 0; blue, 0 }  ][line width=0.08]  [draw opacity=0] (7.2,-1.8) -- (0,0) -- (7.2,1.8) -- cycle    ;
%Curve Lines [id:da10458715808840569] 
\draw [color={rgb, 255:red, 0; green, 0; blue, 0 }  ,draw opacity=1 ]   (535.2,335.6) .. controls (535.2,346.74) and (534.09,350.25) .. (530.49,364.35) ;
\draw [shift={(530.02,366.17)}, rotate = 284.41] [fill={rgb, 255:red, 0; green, 0; blue, 0 }  ,fill opacity=1 ][line width=0.08]  [draw opacity=0] (7.2,-1.8) -- (0,0) -- (7.2,1.8) -- cycle    ;
%Curve Lines [id:da493730763401113] 
\draw [color={rgb, 255:red, 0; green, 0; blue, 0 }  ,draw opacity=1 ]   (529.21,335.17) .. controls (529.95,342.68) and (527.21,352.93) .. (525.56,362.1) ;
\draw [shift={(525.23,364)}, rotate = 280.74] [fill={rgb, 255:red, 0; green, 0; blue, 0 }  ,fill opacity=1 ][line width=0.08]  [draw opacity=0] (7.2,-1.8) -- (0,0) -- (7.2,1.8) -- cycle    ;
%Curve Lines [id:da6471133794486161] 
\draw [color={rgb, 255:red, 0; green, 0; blue, 0 }  ,draw opacity=1 ]   (521.6,334.93) .. controls (523.14,347.16) and (522.8,346.93) .. (521,363.11) ;
\draw [shift={(520.8,364.93)}, rotate = 276.48] [fill={rgb, 255:red, 0; green, 0; blue, 0 }  ,fill opacity=1 ][line width=0.08]  [draw opacity=0] (7.2,-1.8) -- (0,0) -- (7.2,1.8) -- cycle    ;
%Curve Lines [id:da670832980921543] 
\draw    (536,183.73) .. controls (547.09,182.61) and (552.46,205.04) .. (560.75,211.27) ;
\draw [shift={(562.38,212.25)}, rotate = 225.65] [fill={rgb, 255:red, 0; green, 0; blue, 0 }  ][line width=0.08]  [draw opacity=0] (7.2,-1.8) -- (0,0) -- (7.2,1.8) -- cycle    ;
%Curve Lines [id:da02770294021289521] 
\draw    (536.55,206.36) .. controls (547.81,205.22) and (550.4,218.7) .. (562.8,217.08) ;
\draw [shift={(564.63,216.75)}, rotate = 183.71] [fill={rgb, 255:red, 0; green, 0; blue, 0 }  ][line width=0.08]  [draw opacity=0] (7.2,-1.8) -- (0,0) -- (7.2,1.8) -- cycle    ;
%Curve Lines [id:da4364925852621979] 
\draw    (536.36,244.18) .. controls (547.75,243.02) and (559.26,235.92) .. (567.61,220.33) ;
\draw [shift={(568.5,218.6)}, rotate = 121.91] [fill={rgb, 255:red, 0; green, 0; blue, 0 }  ][line width=0.08]  [draw opacity=0] (7.2,-1.8) -- (0,0) -- (7.2,1.8) -- cycle    ;
%Curve Lines [id:da8440950354449732] 
\draw    (33.43,92) .. controls (150.57,19.43) and (523,-25) .. (578.2,202.2) ;
\draw [shift={(578.2,202.2)}, rotate = 254.51] [fill={rgb, 255:red, 0; green, 0; blue, 0 }  ][line width=0.08]  [draw opacity=0] (7.2,-1.8) -- (0,0) -- (7.2,1.8) -- cycle    ;
%Curve Lines [id:da3962909343156955] 
\draw    (30.86,111.43) .. controls (136.86,25.71) and (513,-20.73) .. (575,200.6) ;
\draw [shift={(575,200.6)}, rotate = 252.47] [fill={rgb, 255:red, 0; green, 0; blue, 0 }  ][line width=0.08]  [draw opacity=0] (7.2,-1.8) -- (0,0) -- (7.2,1.8) -- cycle    ;
%Curve Lines [id:da5814632409959868] 
\draw    (30.29,148.57) .. controls (70.33,42.33) and (504,-30) .. (571.97,200.33) ;
\draw [shift={(571.97,200.33)}, rotate = 251.46] [fill={rgb, 255:red, 0; green, 0; blue, 0 }  ][line width=0.08]  [draw opacity=0] (7.2,-1.8) -- (0,0) -- (7.2,1.8) -- cycle    ;
%Curve Lines [id:da44873372603350736] 
\draw    (273.5,90) .. controls (417.19,22.84) and (515.88,115.44) .. (564.64,201.82) ;
\draw [shift={(565.38,203.13)}, rotate = 240.32] [fill={rgb, 255:red, 0; green, 0; blue, 0 }  ][line width=0.08]  [draw opacity=0] (7.2,-1.8) -- (0,0) -- (7.2,1.8) -- cycle    ;
%Curve Lines [id:da8498920721940137] 
\draw    (275,115.25) .. controls (401.78,13.76) and (506.78,114.15) .. (562.99,203.82) ;
\draw [shift={(563.83,205.17)}, rotate = 237.72] [fill={rgb, 255:red, 0; green, 0; blue, 0 }  ][line width=0.08]  [draw opacity=0] (7.2,-1.8) -- (0,0) -- (7.2,1.8) -- cycle    ;
%Curve Lines [id:da5382978880434821] 
\draw    (273,150) .. controls (371.51,3.24) and (503.18,110.91) .. (561.75,206.68) ;
\draw [shift={(562.63,208.13)}, rotate = 238.26] [fill={rgb, 255:red, 0; green, 0; blue, 0 }  ][line width=0.08]  [draw opacity=0] (7.2,-1.8) -- (0,0) -- (7.2,1.8) -- cycle    ;
%Curve Lines [id:da5506933564220804] 
\draw    (469,339.4) .. controls (456.07,353.13) and (431.84,365.67) .. (346.29,370.53) ;
\draw [shift={(345,370.6)}, rotate = 355.96] [fill={rgb, 255:red, 0; green, 0; blue, 0 }  ][line width=0.08]  [draw opacity=0] (7.2,-1.8) -- (0,0) -- (7.2,1.8) -- cycle    ;
%Curve Lines [id:da4798017392113958] 
\draw    (241.36,270.59) .. controls (244.82,272.41) and (249.55,277.14) .. (253.73,282.23) ;
%Curve Lines [id:da22264778478756986] 
\draw    (241.18,268.05) .. controls (245.73,268.41) and (249.36,269.86) .. (253.91,274.05) ;
%Curve Lines [id:da11521960149110244] 
\draw    (240.08,264.36) .. controls (244.62,264.73) and (249.17,265.09) .. (253.71,269.27) ;



% Text Node
\draw (99, 90) node [anchor=north west][inner sep=0.75pt]   [align=left] {Time slot $\displaystyle 1$};
% Text Node
\draw (345, 90) node [anchor=north west][inner sep=0.75pt]   [align=left] {Time slot $\displaystyle 2$};
% Text Node
\draw (28, 128) node [anchor=north west][inner sep=0.75pt]  [rotate=-90]  {$\cdots $};
% Text Node
\draw (270, 128) node [anchor=north west][inner sep=0.75pt]  [rotate=-90]  {$\cdots $};
% Text Node
\draw (90,215) node [anchor=north west][inner sep=0.75pt]  [rotate=-90]  {$\cdots $};
% Text Node
\draw (210,303) node [anchor=north west][inner sep=0.75pt]  [rotate=-90]  {$\cdots $};
% Text Node
\draw (333,215) node [anchor=north west][inner sep=0.75pt]  [rotate=-90]  {$\cdots $};
% Text Node
\draw (455,303) node [anchor=north west][inner sep=0.75pt]  [rotate=-90]  {$\cdots $};
% Text Node
\draw (493.6,368.2) node [anchor=north west][inner sep=0.75pt]    {$\cdots $};
% Text Node
\draw (508,203) node [anchor=north west][inner sep=0.75pt]    {$\cdots $};
% Text Node
\draw (17.5, 88) node [anchor=north west][inner sep=0.75pt]  [font=\footnotesize]  {$\theta _{1}^{1}$};
% Text Node
\draw (17.5, 111) node [anchor=north west][inner sep=0.75pt]  [font=\footnotesize]  {$\theta _{2}^{1}$};
% Text Node
\draw (17.5, 148) node [anchor=north west][inner sep=0.75pt]  [font=\footnotesize]  {$\theta _{S}^{1}$};
% Text Node
\draw (57,174.4) node [anchor=north west][inner sep=0.75pt]  [font=\footnotesize]  {$\kappa _{1}^{1}$};
% Text Node
\draw (57.4,198) node [anchor=north west][inner sep=0.75pt]  [font=\footnotesize]  {$\kappa _{2}^{1}$};
% Text Node
\draw (57,235.2) node [anchor=north west][inner sep=0.75pt]  [font=\footnotesize]  {$\kappa _{S}^{1}$};
% Text Node
\draw (103.6,175.8) node [anchor=north west][inner sep=0.75pt]  [font=\footnotesize]  {$\rho _{1}^{1}$};
% Text Node
\draw (104,198.4) node [anchor=north west][inner sep=0.75pt]  [font=\footnotesize]  {$\rho _{2}^{1}$};
% Text Node
\draw (102.8,234.6) node [anchor=north west][inner sep=0.75pt]  [font=\footnotesize]  {$\rho _{S}^{1}$};
% Text Node
\draw (178.8,262.2) node [anchor=north west][inner sep=0.75pt]  [font=\footnotesize]  {$\lambda _{1}^{1}$};
% Text Node
\draw (178,285.4) node [anchor=north west][inner sep=0.75pt]  [font=\footnotesize]  {$\lambda _{2}^{1}$};
% Text Node
\draw (178.4,322.6) node [anchor=north west][inner sep=0.75pt]  [font=\footnotesize]  {$\lambda _{S}^{1}$};
% Text Node
\draw (224.4,261.4) node [anchor=north west][inner sep=0.75pt]  [font=\footnotesize]  {$\varepsilon _{1}^{1}$};
% Text Node
\draw (225.2,285) node [anchor=north west][inner sep=0.75pt]  [font=\footnotesize]  {$\varepsilon _{2}^{1}$};
% Text Node
\draw (224.4,321.8) node [anchor=north west][inner sep=0.75pt]  [font=\footnotesize]  {$\varepsilon _{S}^{1}$};
% Text Node
\draw (258.67,89) node [anchor=north west][inner sep=0.75pt]  [font=\footnotesize]  {$\theta _{1}^{2}$};
% Text Node
\draw (259.2,112.07) node [anchor=north west][inner sep=0.75pt]  [font=\footnotesize]  {$\theta _{2}^{2}$};
% Text Node
\draw (258.4,148.87) node [anchor=north west][inner sep=0.75pt]  [font=\footnotesize]  {$\theta _{S}^{2}$};
% Text Node
\draw (297.6,176.07) node [anchor=north west][inner sep=0.75pt]  [font=\footnotesize]  {$\kappa _{1}^{2}$};
% Text Node
\draw (298,199.67) node [anchor=north west][inner sep=0.75pt]  [font=\footnotesize]  {$\kappa _{2}^{2}$};
% Text Node
\draw (297.6,236.87) node [anchor=north west][inner sep=0.75pt]  [font=\footnotesize]  {$\kappa _{S}^{2}$};
% Text Node
\draw (345.2,176.47) node [anchor=north west][inner sep=0.75pt]  [font=\footnotesize]  {$\rho _{1}^{2}$};
% Text Node
\draw (345.6,199.07) node [anchor=north west][inner sep=0.75pt]  [font=\footnotesize]  {$\rho _{2}^{2}$};
% Text Node
\draw (344.4,235.27) node [anchor=north west][inner sep=0.75pt]  [font=\footnotesize]  {$\rho _{S}^{2}$};
% Text Node
\draw (420.4,262.87) node [anchor=north west][inner sep=0.75pt]  [font=\footnotesize]  {$\lambda _{1}^{2}$};
% Text Node
\draw (419.6,286.07) node [anchor=north west][inner sep=0.75pt]  [font=\footnotesize]  {$\lambda _{2}^{2}$};
% Text Node
\draw (420,323.27) node [anchor=north west][inner sep=0.75pt]  [font=\footnotesize]  {$\lambda _{S}^{2}$};
% Text Node
\draw (466,262.07) node [anchor=north west][inner sep=0.75pt]  [font=\footnotesize]  {$\varepsilon _{1}^{2}$};
% Text Node
\draw (466.8,285.67) node [anchor=north west][inner sep=0.75pt]  [font=\footnotesize]  {$\varepsilon _{2}^{2}$};
% Text Node
\draw (466,322.47) node [anchor=north west][inner sep=0.75pt]  [font=\footnotesize]  {$\varepsilon _{S}^{2}$};
% Text Node
\draw (566.8,204.87) node [anchor=north west][inner sep=0.75pt]  [font=\footnotesize]  {$\mu $};
% Text Node
\draw (288,404.07) node [anchor=north west][inner sep=0.75pt]  [font=\footnotesize]  {$\delta $};
% Text Node
\draw (330.8,369.8) node [anchor=north west][inner sep=0.75pt]  [font=\footnotesize]  {$\tau _{1}$};
% Text Node
\draw (469.2,369) node [anchor=north west][inner sep=0.75pt]  [font=\footnotesize]  {$\tau _{2}$};
% Text Node
\draw (518,369.4) node [anchor=north west][inner sep=0.75pt]  [font=\footnotesize]  {$\tau _{N}$};
% Text Node
\draw (544.67, 367) node [anchor=north west][inner sep=0.75pt]   [align=left] {Tasks};
% Text Node
% \draw (508, 20) node [anchor=north west][inner sep=0.75pt]    {$\cdots $};




\end{tikzpicture}


    \end{center}
    \caption{The graph for which finding the min cost flow gives the minimized CF in the edge computing network.}\label{fig:networkflow}
\end{figure*}







Although the CF minimization problem is formulated as ILP in \eqref{formulation}, it corresponds to a minimum-cost flow problem that can be solved in polynomial time. We show the corresponding graph in Fig. \ref{fig:networkflow}.

\subsection{Overall Introduction}
A minimum-cost flow problem is an optimization problem to find the cheapest possible way of sending a certain amount of flow from supply node(s) to demand node(s) in a directed graph. In the graph, every arcs has two attributes: 1) the per-unit cost and 2) the capacity of sending flow across itself. The flows in the graph satisfy:
\begin{enumerate}
    \item Balance constraint: the sum of the flow through arcs directed toward a node plus that node's supply, if any, equals the sum of the flow through arcs directed away from that node plus that node's demand, if any. 
    \item Capacity constraint: The flow on every arc of a flow network is no more than the arc's capacity.
\end{enumerate}

In our graph shown in Fig \ref{fig:networkflow}, the green and the yellow nodes in are supply and demand nodes, respectively, and the other colored nodes are all transshipment nodes that neither supply nor demand. Overall, green node $\theta$, red $(\kappa, \rho)$-pairs, blue $(\lambda, \varepsilon)$-pairs, and yellow $\tau$-nodes represent the grid, batteries, servers, and tasks, respectively. The flows in the graph can be regarded as energy transfer, and the per-unit costs of arcs are used for accounting for the CF.

Note that the per-unit costs and capacities of arcs default to zero and infinite if we do not specify them, respectively. we use $(a\rightarrow b)$ to represent the arc from node $a$ to node $b$.

\subsection{The Entities in a Time Slots} \label{sub-A}

There are $T$ parts named by Time Slot $t$ $(t\in \mathcal{T})$ in the graph. Looking into a part of them, there are three types of entities that are $\theta$-nodes, $(\kappa, \rho)$-pairs, and $(\lambda, \varepsilon)$-pairs:
\begin{enumerate}
    \item Green node $\theta^{t}_{s}$ represents renewable source $s$ in time slot $t$. Its quantity supplied is set to be $R_{st}$, i.e., the amount of energy available from source $s$ in time slot $t$. The meaning of the two flows in arcs $(\theta^{t}_{s}\rightarrow \lambda^{t}_{s})$ and $(\theta^{t}_{s}\rightarrow \kappa^{t}_{s})$ is the same as that of the variables $z_{st}$ and $u_{st}$, respectively. By graph construction, the two flows follow constraint \eqref{c5}. Note that node $\theta^{t}_{s}$ also connects to yellow demand node $\mu$ used for receiving the surplus flows (i.e., surplus energy) in the graph to guarantee the feasibility of the minimum-cost flow problem.
    \item A red pair $(\kappa^{t}_{s}, \rho^{t}_{s})$ represents battery $s$ in time slot $t$. The capacity of arc $(\kappa^{t}_{s}\rightarrow \rho^{t}_{s})$ is set to be the capacity of battery $s$, i.e., $L$. The flows in arcs $(\delta \rightarrow \kappa^{t}_{s})$, $(\theta^{t}_{s} \rightarrow \kappa^{t}_{s})$, and $(\rho^{t-1}_{s}\rightarrow \kappa^{t}_{s})$ are equivalent to variables $u_{st}$, $v_{st}$, and $w_{s(t-1)}$, respectively, and they follow constraint \eqref{c2} by their flow balance constraint.
    \item A blue pair $(\lambda^{t}_{s}, \varepsilon^{t}_{s})$ represents server $s$ in time slot $t$. The flows in arcs $(\delta \rightarrow \lambda^{t}_{s})$, $(\rho^{t}_{s^\prime} \rightarrow \lambda^{t}_{s})$, and $(\theta^{t}_{s} \rightarrow \lambda^{t}_{s} )$ represent variables $x_{nst}$, $y_{nss^\prime t}$, and $z_{nst}$, respectively. The flows are subject to constraint \eqref{c1} by flow balance constraint. In addition, the capacity of arc $(\lambda^{t}_{s}\rightarrow\varepsilon^{t}_{s})$ is set to be server capacity $H$, so the corresponding capacity constraint implies constraint \eqref{c4}. Note that the flow through arc $(\rho^{t}_{s^\prime}\rightarrow\lambda^{t}_{s})$ represents that battery $s^\prime$ provides energy to server $s$ in time slot $t$, so the per-unit cost of the arc is set to be $\beta_{s^\prime s} I_{st}$, and it accounts for the losses in task offloading. Clearly, the total flow cost in these arcs equals to \eqref{lo}.
\end{enumerate}

\subsection{Battery Evolution}
Between any two neighboring parts Time slots $t$ and $t+1$ in the graph, there are $S$ arcs, i.e., $(\rho^{t}_{s}\rightarrow\kappa^{t+1}_{s})$ $\forall s\in \mathcal{S}$. The arcs mean the evolution of the batteries between time slots $t$ and $t+1$. According to flow balance constraint in networkflow graph, node $\kappa^{t+1}_{s}$ will receive flows that are from node $\rho^{t}_{s}$ except those to node $\lambda^{t}_{s}$, i.e., this flow balance constraint derives constraint \eqref{c3}.

\subsection{The Grids}

The multiple local grids are merged to be a green node $\delta$ but we can easily identify the different grids by the arcs to the nodes corresponding to different batteries and servers:
\begin{enumerate}
    \item Flows can be sent through arc $(\delta \rightarrow\kappa^{t}_{s})$, and this represents that local grid $s$ charge battery $s$ in time slot $t$. Clearly, its per-unit cost is $I_{st}$ that accounts for CF in battery charging, and the total flow cost in these arcs is the same as \eqref{bc}.
    \item We use the flow through arc $(\delta \rightarrow \lambda^{t}_{s})$ to represent that local grid $s$ provides energy to server $s$ in time slot $t$, so the cost of arc $(\delta \rightarrow \lambda^{t}_{s})$ is set to be $I_{st}$. The per-unit cost is used for accounting for CF in task completion, and the sum of flow cost in these arcs is equivalent to \eqref{th}.
\end{enumerate}
Another arc $(\delta \rightarrow \mu)$ is used for sending the surplus flows from the grid. The quantity supplied of node $\delta$ is set to be large enough, e.g., $N$. 

\subsection{The Tasks} \label{sub-D}

In the part named by Task, a yellow demand node $\tau_n$ represents task $n$. The quantity demanded of any node $\tau_n$ is one. The arcs between $\varepsilon$-nodes and $\tau$-nodes are constructed by the information of all the tasks' tuples. Specifically, arc $(\varepsilon^{t}_{s}\rightarrow\tau_{n})$ represents that server $s$ can complete task $n$ in time slot $t$ ($o_n \leq t \leq d_n$). Obviously, the flow through this arc represent variable $\pi_{nst}$, and the balance constraint for the arc are equivalent to constraint \eqref{c0}.

Regarding the per-unit cost of the arcs, we use them to account for the CF in task offloading. Therefore, the per-unit cost of arc $(\varepsilon^{t}_{s}\rightarrow\tau_n)$ is $\alpha_{s_{n}s}I_{s_{n}t}$, and the sum of flow cost in these arcs is equivalent to \eqref{tt}.

\subsection{Algorithms and Complexity}

A minimum-cost flow problem can be solved to the optimum by existing algorithms in polynomial time \cite{10.5555/137406}. So, we have the following theorem with respect to the complexity of our optimization approach.

\begin{theorem}
CF minimization problem \eqref{formulation} can be solved in polynomial time.
\end{theorem}
\begin{proof}
The optimization approach to problem \eqref{formulation} is that 
\begin{enumerate}
    \item we construct the corresponding graph by the details in Section \ref{sub-A} - \ref{sub-D};
    \item we obtain the corresponding solution to problem \eqref{formulation} by, for example, the network simplex algorithm. 
\end{enumerate}
Obviously, the corresponding graph have in total $\mathcal{O}(ST + N)$ nodes and $\mathcal{O}(S^2 +ST + NST)$ arcs, so the graph constructing can be completed in polynomial time. In addition, the minimum-cost flow problem can be solved in polynomial time. Hence the conclusion.
\end{proof}






\section{Performance Evaluation}

In this section, we use a 24-hour CI data set of Sweden, Germany, and Poland from \cite{ElectricityMaps}, a part of which is shown in \ref{tab:data}, for performance evaluation. In our simulation, the length of a time slot is an hour. The amount of renewable energy follows a binomial distribution $B(5, 0.5)$ during daytime (7 am to 7 pm), and is zero otherwise. All the tasks' $o_{n}$, $d_n$ ($o_n \leq d_n$), and $s_n$ all follow a discrete uniform distribution. In addition, we set $\mathcal{S}_n = \mathcal{S}$ $\forall n \in \mathcal{N}$, i.e., a task can be offloaded to any server. Other simulation parameters are listed in Table \ref{tab:parameters}. Note that the values of CF in our results are all normalized by $E$. 

\begin{table}[h]
    \caption{\label{tab:parameters}Simulation Parameters.}
    \begin{center}
        \begin{threeparttable}[b]
            \begin{tabular}{*{2}{lr}}
                \toprule
                \midrule
                {\bf Parameter} & {\bf Value}\\
                \midrule
                 The number of time slots ($T$) & $24$ \\
                 The number of servers ($S$) & $3$\\
                 The number of tasks ($N$) & $100$\\
                 The energy for transferring one task ($\alpha$) & $0.1$\\
                 Loss for transferring one energy unit ($\beta$) & $0.2$\\
                \bottomrule
            \end{tabular}
        \end{threeparttable}
    \end{center}
\end{table}

\begin{figure}[t]
	\begin{center}

    \pgfplotstableread[col sep=comma,header=true]{%
    y,x,myvalue
    0 ,  7 ,  27464.1
    5 ,  7 ,  14408.1
    10 ,  7 ,  9417.0
    15 ,  7 ,  7024.5
    20 ,  7 ,  6030.4
    25 ,  7 ,  5610.0
    30 ,  7 ,  5333.9
    0 ,  8 ,  24431.0
    5 ,  8 ,  13741.2
    10 ,  8 ,  9063.1
    15 ,  8 ,  6639.3
    20 ,  8 ,  5808.4
    25 ,  8 ,  5314.4
    30 ,  8 ,  5094.4
    0 ,  9 ,  22348.9
    5 ,  9 ,  13279.9
    10 ,  9 ,  8967.8
    15 ,  9 ,  6499.8
    20 ,  9 ,  5657.5
    25 ,  9 ,  5182.6
    30 ,  9 ,  4924.0
    0 ,  10 ,  20266.8
    5 ,  10 ,  12849.5
    10 ,  10 ,  8885.2
    15 ,  10 ,  6434.8
    20 ,  10 ,  5536.5
    25 ,  10 ,  5050.8
    30 ,  10 ,  4829.8
    0 ,  11 ,  19292.6
    5 ,  11 ,  12518.0
    10 ,  11 ,  8802.6
    15 ,  11 ,  6402.3
    20 ,  11 ,  5443.5
    25 ,  11 ,  4975.0
    30 ,  11 ,  4819.0
    0 ,  12 ,  18799.8
    5 ,  12 ,  12376.8
    10 ,  12 ,  8720.0
    15 ,  12 ,  6402.3
    20 ,  12 ,  5378.5
    25 ,  12 ,  4910.0
    30 ,  12 ,  4808.2
    0 ,  13 ,  18452.7
    5 ,  13 ,  12268.8
    10 ,  13 ,  8637.4
    15 ,  13 ,  6402.3
    20 ,  13 ,  5313.5
    25 ,  13 ,  4877.5
    30 ,  13 ,  4797.4
    }{\datatable}
    %
    %\pgfplotstablesort[col sep=comma,header=true]\resulttable{\datatable}
    \pgfplotstablesort[create on use/sortkey/.style={
            create col/assign/.code={%
                \edef\entry{{\thisrow{x}}{\thisrow{y}}{\thisrow{myvalue}}}%
                \pgfkeyslet{/pgfplots/table/create col/next content}\entry
            }
        },
        sort key=sortkey,
        sort cmp={%
            iflessthan/.code args={#1#2#3#4}{%
                \edef\temp{#1#2}%
                \expandafter\pgfplotsmulticmpthree\temp\do{#3}{#4}%
            },
        },
        sort,
        columns/Mtx/.style={string type},
        columns/Kind/.style={string type},]\resulttable{\datatable}
    
    \begin{tikzpicture}%[x={(0.866cm,-0.5cm)},y={(0.866cm,0.5cm)},z={(0cm,1 cm)}]
        \pgfplotsset{set layers}
        \begin{axis}[% from section 4.6.4 of the pgfplotsmanual
                view={120}{40},
                width=0.5*\textwidth,
                height=0.4375*\textwidth,
                z buffer=none,
                xmin=6.5, xmax=13,
                ymin=-1, ymax=30,
                zmin=0, zmax=56000,
                enlargelimits=upper,
                ztick={25000, 50000},
                zticklabels={2.5, 5}, % here one has to "cheat"
                % meaning that one has to put labels which are the actual value 
                % divided by 2. This is because the bars will be centered at these
                % values
                xtick=data,
                extra tick style={grid=major},
                ytick=data,
                grid=minor,
                xlabel={$H$},
                ylabel={$L$},
                zlabel={Total CF},
                minor tick num=1,
                point meta=explicit,
                colormap name=viridis,
                scatter/use mapped color={
                    draw=mapped color,fill=mapped color!60},
                execute at begin plot={}            
                ]
        \path let \p1=($(axis cs:0,0,1)-(axis cs:0,0,0)$) in 
        \pgfextra{\pgfmathsetmacro{\conv}{2*\y1}
        \ifx\gconv\conv
        \else
        \xdef\gconv{\conv}
        \typeout{Please\space recompile\space the\space file!}
        \fi     
                };  
        \path let \p1=($(axis cs:1,0,0)-(axis cs:0,0,0)$) in 
        \pgfextra{\pgfmathsetmacro{\convx}{veclen(\x1,\y1)}
        \typeout{One\space unit\space in\space x\space 
                direction\space is\space\convx pt}
                };                  
        \path let \p1=($(axis cs:0,1,0)-(axis cs:0,0,0)$) in 
        \pgfextra{\pgfmathsetmacro{\convy}{veclen(\x1,\y1)}
        \typeout{One\space unit\space in\space y\space 
                direction\space is\space\convy pt}
                };                  
        \addplot3 [visualization depends on={
        \gconv*z \as \myz}, % you may have to recompile to get the prefactor right
        scatter/@pre marker code/.append style={/pgfplots/cube/size z=\myz},%
        scatter/@pre marker code/.append style={/pgfplots/cube/size x=11.66135pt},%
        scatter/@pre marker code/.append style={/pgfplots/cube/size y=9.10493pt},%
        scatter,only marks,
        mark=cube*,mark size=5,opacity=1]
         table[x expr={\thisrow{x}},y expr={\thisrow{y}},z
         expr={1*\thisrow{myvalue}},
         meta expr={-1*\thisrow{x}}
                ] \resulttable;
            \end{axis}
        \makeatletter
        \immediate\write\@mainaux{\xdef\string\gconv{\gconv}\relax}
        \makeatother
    
    \end{tikzpicture}

	\end{center}
	\caption{The 3D histogram shows the total CF under varying batter capacity $L$ and server capacity $H$.}\label{fig:result1}
\end{figure}


Fig. \ref{fig:result1} shows the performance results of our proposed scheme with respect to battery capacity $L$ and server capacity $H$. As we can see, there is a dramatic reduction of CF when the battery capacity goes from zero (i.e., no battery) to $5$ units. Thus the importance of battery (and BMS) for cresting low-CF energy buffer is apparent. When the battery capacity $L$ is greater than $15$, the total CF is hardly improved because this capacity level can store all the energy required by all the tasks. In addition, when the server capacity increases, the total CF decreases as more tasks can be offloaded to servers in low-CI spaces.

We are curious about the impact of task offloading and energy sharing, respectively, hence we compare the performance of four schemes:
\begin{itemize}
    \item S1: The is the proposed scheme considering both task offloading and energy sharing via BMS in the network.
    \item S2: This scheme allows task offloading but the BMS is disabled, i.e., a battery can only provide energy to the local server.
    \item S3: This is the opposite to S2, namely energy sharing is enabled, but the tasks can not be offloaded among the servers.
    \item S4: In the last scheme, task offloading and energy sharing are both disabled; this corresponds to the most basic benchmark for comparison.
\end{itemize}
In the simulation, we obtain the performance of the four schemes by solving the corresponding minimum-cost flow problems. Figs. \ref{fig:result2} and \ref{fig:result3} show the performance results with respect to batter capacity $L$ and server capacity $H$, where we use w/ and w/o to represent with and without, respectively. Overall, compared with the conventional network (scheme S4), task offloading and energy sharing can help significantly reduce the total CF up to $83.3\%$ by the observation in Fig.\ref{fig:result3} when $L = 5$ and $H = 16$.

\begin{figure}
%	\setlength\abovecaptionskip{0.5\baselineskip}
%  	\setlength\abovecaptionskip{0.75\baselineskip}
        \begin{center}
    		\begin{tikzpicture}
        		\begin{axis}[
        		%	scaled ticks=true, 
        		%	scaled y ticks={real:1e3},
        			scaled y ticks=base 10:-3,
        		    %title={},
        		    xlabel={The battery capacity $L$},
        		    ylabel={Total CF},
        		    xmin=1, xmax=10,
        		    ymin=0, ymax=22000,
        		  %  xtick={25, 50, 75, 100, 125, 150},
        		%	xticklabels=\empty,
        %		    ytick={0, 1000, 2000, 3000},
        		%	point meta=y *10^3, % the displayed number
        		    legend pos=north east,
        		    %ymajorgrids=true,    
        		    %xmajorgrids=true,
        		    grid style=densely dashed,
        		    tick label style={font=\scriptsize},
        		    label style={font=\small},
        		    legend style={font=\scriptsize},
        		]
        		
            		\addplot[ color=red, mark=square, line width=0.8pt]     
            		coordinates { 
            		( 1 , 9342.1 )
                    ( 2 , 7706.7 )
                    ( 3 , 6087.9 )
                    ( 4 , 4689.1 )
                    ( 5 , 3379.7 )
                    ( 6 , 2966.0 )
                    ( 7 , 2809.3 )
                    ( 8 , 2674.2 )
                    ( 9 , 2539.1 )
                    ( 10 , 2425.4 )
            		};
            		%
            		\addplot[ color=blue, mark=o, dotted, mark options={solid}, line width=0.8pt]     
            		coordinates {
                    ( 1 , 16053.6 )
                    ( 2 , 13501.4 )
                    ( 3 , 11056.8 )
                    ( 4 , 8895.2 )
                    ( 5 , 6815.0 )
                    ( 6 , 5303.2 )
                    ( 7 , 4006.2 )
                    ( 8 , 3679.6 )
                    ( 9 , 3582.6 )
                    ( 10 , 3496.2 )
            		};
            		
            		\addplot[ color=black, mark=triangle, densely dashed, mark options={solid}, line width=0.8pt]
            		coordinates {
                    ( 1 , 10218.3 )
                    ( 2 , 9378.0 )
                    ( 3 , 8537.7 )
                    ( 4 , 7735.5 )
                    ( 5 , 6933.3 )
                    ( 6 , 6190.4 )
                    ( 7 , 5490.7 )
                    ( 8 , 4812.6 )
                    ( 9 , 4245.1 )
                    ( 10 , 3682.4 )
            		};
            		
            		\addplot[ color=brown, mark=x, dash dot, mark options={solid}, line width=0.8pt]
            		coordinates {
                    ( 1 , 18807.0 )
                    ( 2 , 17852.0 )
                    ( 3 , 16923.0 )
                    ( 4 , 15994.0 )
                    ( 5 , 15065.0 )
                    ( 6 , 14683.0 )
                    ( 7 , 14301.0 )
                    ( 8 , 13919.0 )
                    ( 9 , 13537.0 )
                    ( 10 , 13155.0 )
            		};

            		\legend{w/ offloading and BMS, w/ BMS w/o offloading, w/ offloading w/o BMS, w/o offloading and BMS}
        		
        		\end{axis}
    		\end{tikzpicture}
    		
        \end{center}
    \caption{Caption.}\label{fig:result3}
\end{figure}


\begin{figure}
%	\setlength\abovecaptionskip{0.5\baselineskip}
%  	\setlength\abovecaptionskip{0.75\baselineskip}
        \begin{center}
    		\begin{tikzpicture}
        		\begin{axis}[
        		%	scaled ticks=true, 
        		%	scaled y ticks={real:1e3},
        			scaled y ticks=base 10:-3,
        		    %title={},
        		    xlabel={The server capacity $H$},
        		    ylabel={Total CF},
        		    xmin=8, xmax=17,
        		  %  ymin=0, ymax=3500,
        		  %  xtick={25, 50, 75, 100, 125, 150},
        		%	xticklabels=\empty,
        %		    ytick={0, 1000, 2000, 3000},
        		%	point meta=y *10^3, % the displayed number
        		  %  legend pos=north east,
        		    legend style={at={(0.48, 0.75)},anchor=west},
        		    %ymajorgrids=true,    
        		    %xmajorgrids=true,
        		    grid style=densely dashed,
        		    tick label style={font=\scriptsize},
        		    label style={font=\small},
        		    legend style={font=\scriptsize},
        		]
        		
            		\addplot[ color=red, mark=square, line width=0.8pt]     
            		coordinates { 
                    ( 8 , 4949.3 )
                    ( 9 , 4163.9 )
                    ( 10 , 3379.7 )
                    ( 11 , 3005.4 )
                    ( 12 , 2878.1 )
                    ( 13 , 2767.4 )
                    ( 14 , 2656.7 )
                    ( 15 , 2564.0 )
                    ( 16 , 2528.3 )
                    ( 17 , 2521.1 )
            		};
            		%
            		\addplot[ color=blue, mark=o, dotted, mark options={solid}, line width=0.8pt]     
            		coordinates {
                    ( 8 , 6818.0 )
                    ( 9 , 6816.0 )
                    ( 10 , 6815.0 )
                    ( 11 , 6815.0 )
                    ( 12 , 6815.0 )
                    ( 13 , 6815.0 )
                    ( 14 , 6815.0 )
                    ( 15 , 6815.0 )
                    ( 16 , 6815.0 )
                    ( 17 , 6815.0 )
            		};
            		
            		\addplot[ color=black, mark=triangle, densely dashed, mark options={solid}, line width=0.8pt]
            		coordinates {
                    ( 8 , 9506.6 )
                    ( 9 , 8014.9 )
                    ( 10 , 6933.3 )
                    ( 11 , 5927.9 )
                    ( 12 , 5033.1 )
                    ( 13 , 4248.9 )
                    ( 14 , 3464.7 )
                    ( 15 , 2766.0 )
                    ( 16 , 2528.3 )
                    ( 17 , 2521.1 )
            		};
            		
            		\addplot[ color=brown, mark=x, dashed, mark options={solid}, line width=0.8pt]
            		coordinates {
                    ( 8 , 15091.0 )
                    ( 9 , 15065.0 )
                    ( 10 , 15065.0 )
                    ( 11 , 15065.0 )
                    ( 12 , 15065.0 )
                    ( 13 , 15065.0 )
                    ( 14 , 15065.0 )
                    ( 15 , 15065.0 )
                    ( 16 , 15065.0 )
                    ( 17 , 15065.0 )
            		};

            		\legend{w/ offloading and BMS, w/ BMS w/o offloading, w/ offloading w/o BMS, w/o offloading and BMS}
        		
        		\end{axis}
    		\end{tikzpicture}
    		
        \end{center}
    \caption{Caption.}\label{fig:result3}
\end{figure}


In Fig. \ref{fig:result2}, the comparison between S1 and S2 reveals that, when the batteries are low-capacitied and high-capacitied, the energy sharing is useless. When the low case, this is because there is less surplus energy for sharing when the low case. For the high case, the tasks can be offloaded to servers with the local battery has stored enough low-CI energy. A similar conclusion from the results of S1 and S2 in Fig. \ref{fig:result3} is that energy sharing is not crucial when server capacity is large enough. In the case of high-capacitied servers, the tasks can offloaded as much as possible to low-CI space such that we do not need energy sharing. 

In addition, the curves of S2 and S3 in Fig. \ref{fig:result2} shows that task offloading and energy sharing almost give the same effect when battery capacity is high enough. This can be expected as the two operations that offloading a task to low-CI space and obtaining energy from low-CI space are in fact similar. So, we only need to consider one of the two operations in the case of high-capacitied batteries and servers.

The performance of S3 and S4 in Fig. \ref{fig:result3} hardly change with increasing server capacity $H$, and this observation shows that considering only task scheduling without task offloading leads to a limited reduction of CF. This is because the difference of CI in temporal dimension is typically not so large as that in spatial dimension. In addition, the performance of S1 and S2 in Fig. \ref{fig:result3} shows that energy sharing can help reduce significantly the total CF when the capacities of the batteries and the servers are both moderate size, for example, $L = 5$ and $H = 10$. 

\section{Conclusion} \label{Sec:conclusion}
We have considered a CF minimization problem for an edge computing network with renewable energy and energy sharing by joint task offloading and energy sharing. We formulated the problem as an integer linear programming model. Making the use of the structure of the problem, we have proposed a graph-based reformulation that can help solve the problem to the optimum in polynomial time. The numerical results demonstrated the potential of the use of spatial and temporal information of CI for reducing the CF in the network, and joint task offloading and energy sharing can help significantly reduce CF. Moreover, our observation have found that we can achieve the potential if only improve the energy storage or computing capability instead of both.

\bibliographystyle{IEEEtran}
\bibliography{mybibtex}


\end{document}
